%!TEX program = xelatex
% 完整编译: xelatex -> biber/bibtex -> xelatex -> xelatex
\documentclass[lang=cn,11pt,a4paper]{elegantpaper}

\title{最优化小论文}
\author{22120307 陈景龙}
\institute{北京交通大学}

% \version{0.10}
\date{\zhtoday}

\usepackage{algorithm}
\usepackage{algorithmic}
\usepackage{siunitx}

% 本文档命令
\usepackage{array}
\newcommand{\ccr}[1]{\makecell{{\color{#1}\rule{1cm}{1cm}}}}
\addbibresource[location=local]{./reference.bib} % 参考文献,不要删除

% 定义数学命令
\newcommand{\subject}{s.t.}
\newcommand{\E}{\mathbb{E}}
\newcommand{\I}{\mathbb{I}}

\begin{document}

\maketitle

\begin{abstract}
    寻求NP-Hard问题的较优算法是一重要内容,其中,与独立集相关的问题较为常见。本文将从独立集的性质入手,介绍基于多种思想的几类独立集算法。对于一般图,本文在搜索算法的基础上,提出了动态规划的优化算法,同时分析并测试了几种算法在随机数据下的运行效率;对于特殊图,本文介绍了两种针对特殊图的算法思想,提出了求解“k-仙人图”的独立集问题的算法思想及其扩展应用。

    \keywords{优化问题,独立集算法,NP-Hard}
\end{abstract}

\newpage
\definecolor{winered}{rgb}{0,0,0} 
\tableofcontents
\newpage

\section{前言}
不少与独立集有关的问题(最大独立集、最大带权独立集、独立集计数等问题)都是图论中经典的NP-Hard问题,在信息学竞赛中广泛出现,然而解决独立集问题的算法效率通常不高。由此,本人对此类问题进行了更深入的研究,希望能用更加高效的方法解决此类问题。

本文的研究分为两部分,第一部分介绍了两种求解一般图的独立集问题的算法:基于极大独立集搜索的独立集算法和基于动态规划的独立集算法,第二部分介绍了两类特殊图的独立集算法,分别基于图匹配思想和图上的阶段划分思想。

\section{定义及约定}
独立集问题有多种形式。为了方便描述,以下给出一些定义。
\begin{definition}
    对于无向图 $G = (V, E)$ 和点 $u, v \in V$,若 $(u, v) \in E$,则称 $u, v$ 相邻(adjacent);定义点 $v \in V$ 的领域(neighborhood)为 $V$ 中与 $v$ 相邻的结点集合,记为 $N(v)$;另外,$N_G(v)$ 表示 $v$ 在图 $G$ 中的领域。
\end{definition}

\begin{definition}
    点 $v$ 的度(degree) $\deg(v)$ 定义为 $N(v)$ 的大小,即 $\deg(v) = |N(v)|$;另外,$\deg_G(v)$ 表示 $v$ 在图 $G$ 中的度。
\end{definition}

\begin{definition}
    无向图 $G = (V, E)$ 的一个独立集(independent set)定义为 $V$ 的一个子集,满足子集中的结点两两不相邻。形式化地,$I$ 是 $G$ 的一个独立集,当且仅当 $I \subseteq V$ 且 $\forall u, v \in I$,$(u, v) \notin E$。
\end{definition}

\begin{definition}
    无向图 $G = (V, E)$ 的一个最大独立集(maximum independent set)是指 $G$ 中所含结点数 $|I|$ 最多的独立集 $I$。
\end{definition}

\begin{definition}
    无向图 $G = (V, E)$ 的独立数(independence number)\footnote{导出子图有点导出子图和边导出子图两种,本文中均指前者。}定义为 $G$ 的最大独立集 $I$ 所含的结点数$|I|$,记为 $\alpha(G)$。
\end{definition}

\begin{definition}
    无向图 $G = (V, E)$ 在 $S \subseteq V$ 上的导出子图(induced subgraph)定义为以 $S$ 为点集,两端点都在 $S$ 内的边为边集构成的图,记为 $G[S]$。
\end{definition}

本文中的所有问题均以最大独立集问题为例,即对于给定的无向图 $G = (V, E)$,找出 $G$ 的一个最大独立集 $I$。

为了方便起见,约定 $n$ 为图 $G$ 的阶数(即顶点数),$m$ 为图 $G$ 的边数,即 $n = |V|$,$m = |E|$。


\section{一般图的独立集问题}
目前,解决一般图的大多数独立集相关的问题都不存在多项式时间的算法,只能用复杂度较优的指数级算法。

事实上,已有不少理论复杂度十分优秀的求图的最大独立集的算法,能够快速计算出上百阶的无向图的最大独立集\footnote{Robson提出了一种复杂度约 $O(1.1888^n)$的最大独立集算法},但这些算法实现往往过于复杂,难以应用。笔者选择了一些相对高效又较易于实现的算法进行了研宄。

\subsection{基于极大独立集搜索的独立集算法}
\subsubsection{朴素的搜索算法}
最朴素的搜索算法非常简单:用深度优先搜素枚举 $V$ 的子集 $I \subseteq V$,即按一定顺序枚举每个点以 $v \in V$ 是否属于 $I$,一旦存在 $(u, v) \in E$ 使得 $u, v \in I$,就回溯。输出枚举的所有独立集 $I$ 中, $|I|$ 最大的一个。该算法的复杂度为 $O(2^nm)$,效率太低。

针对最大独立集这一问题,可以进行一些剪枝。例如:
\begin{enumerate}
    \item 若 $\deg(v) = 0$,则不存在与 $v$ 关联的边,故总可以令 $v \in I$。
    \item 若 $\deg(v) = 1$,考虑唯一的与 $v$ 关联的结点 $u$,若 $u \notin I$,则总可以令 $v \in I$;否则,从 $I$ 中删去 $u$ 并加入 $v$,$I$ 的大小不变。因此总可以令 $v \in I$。
    \item 搜索时记录当前搜到的独立集的大小的最大值 $a$,记 $P$ 为 $V-I$ 中不与 $I$ 中结点相邻的点集,当 $|I| + |P| \le a$ 时可进行最优性剪枝。
\end{enumerate}

然而加入这些剪枝之后,复杂度仍然很高,难以接受。

\subsubsection{极大独立集与Bron-Kerbosch算法}
朴素的搜索算法效率太低,有没什么好的方法米优化呢?我们提出极大独立集的概念:
\begin{definition}
    无向图 $G=(V, E)$ 的一个极大独立集(maximal independent set)是指 $G$ 的一个独立集 $I$,满足对于任意的结点 $v \in V - I$,点集 $I + \{v\}$ 不是独立集。
\end{definition}

通常情况下,一个图的极大独立集个数比独立集个数少得多。例如 50 个结点构成的链有32,951,280,099 个独立集,却只有1,221,537 个极大独立集。另外,不少有关独立集的组合优化问题都可以只考虑极大独立集,最大独立集问题就是这样一个例子:
\begin{theorem}
    每个最大独立集都是极大独立集。
\end{theorem}
\begin{proof}
    设 $I$ 是一个最大独立集。对于任意的 $v \in V - I$,假如 $I + \{v\}$ 是独立集,因为 $|I + \{v\}| = |I| + 1 > |I|$,所以 $I + \{v\}$ 是一个比 $I$ 更大的独立集,也就是说,$I$ 不是最大独立集,与假设矛盾。所以 $I$ 一定是一个极大独立集。
\end{proof}

因此,如果能找出 $G$ 的所有极大独立集,就能找出 $G$ 的最大独立集。

求极大独立集的算法有很多,其中Bron-Kerbosch算法是实现较为简洁的一种。下面介绍求极大独立集的Bron-Kerbosch算法。

Bron-Kerbosch算法可以对任意的无向图G求出其所有的极大独立集,其基本思想仍然是搜索优化。伪代码如下\footnote{Bron-Kerbosch算法有多种实现方式,本文介绍其中的一种。}:

\begin{algorithm} 
	\caption{BronKerbosch($R, P, X$)}\label{alg1} 
	\begin{algorithmic}[1]
        \IF{$P = X = \varnothing$}
            \PRINT $R$
        \ENDIF
        \STATE{选择结点 $u \in P \cup X$,使得 $|P \cap (\{u\}\cup N(u))|$ 最小}
        \FOR{$\forall v \in P \cap (\{u\} \cup N(u))$} 
            \STATE{BronKerbosch($P \cup \{v\}, P - (\{v\} \cup N(v)), X - (\{v\} \cup N(v))$)} 
            \STATE{$P \gets P - \{v\}$}
            \STATE{$X \gets X \cup \{v\}$}
        \ENDFOR
	\end{algorithmic} 
\end{algorithm}

调用 $\mathrm{BronKerbosch}(R, P, X)$ 函数,将输出 $G$ 的所有包含 $R$ 中的所有结点、$P$ 中的任意多个结点且不包含 $X$ 中的结点的所有极大独立集。调用 $\mathrm{BronKerbosch}(\varnothing, V, \varnothing)$ 即可得到 $G$ 的所有极大独立集。

实现时,集合可以用压位的方法存储,即用一个大小为 $\left \lceil \frac{n}{64} \right \rceil $ 的64位整数数组 $A$ 存一个大小为 $n$ 的集合 $A^{'}$,$x \in A^{'}$ 当且仅当 $A[\left \lceil \frac{x}{64} \right \rceil ] \text{AND} 2^{x \mod 64} > 0$($A$ 数组下标从0开始)。因为独立集问题中,图的阶数 $n$ 不会很大,所以压位的数组大小可以近似认为是一个常数。

上述算法的最关键之处在于Pivoting。算法过程中,有一步是选择结点 $u \in P \cup X$,使得 $|P \cap \left(\{u\} \cup N(u)\right)|$ 最小,$u$ 称为Pivot结点。之后枚举 $\{u\}\cup N(u)$ 中,属于独立集 $R$ 的第一个结点 $v$。这就是Pivoting的过程。

如果直按搜索极大独立集的话,效率是很低的,因为会搜到很多不是极大的独立集。例如当 $G$ 为 $n$ 阶零图\footnote{零图定义为没有边的图,即 $G = (V, E)$为零图当且仅当 $E = \varnothing$。}时,显然 $V$ 是 $G$ 的唯一的极大独立集,然而朴素的搜索枚举了某个结点不属于极大独立集时,尽管不可能搜出极大独立集,但算法还会继续搜索下去,浪费了大量时间。Pivoting的正确性基于以下定理:
\begin{theorem}
    对于无向图 $G=(V, E)$和 $v \in V$,$G$ 的任意极大独立集 $I$满足 $I \cap(\{v\} \cup N(v)) \neq \varnothing$。
\end{theorem}
\begin{proof}
    证明假设存在极大独立集 $I$,满足 $I \cap(\{v\} \cup N(v)) = \varnothing$,则对于任意 $u \in I$,$u \notin \{v\} \cup N(v)$,即 $u \neq v$ 且 $(u, v) \neq E$。

    因此 $I \cup \{v\}$ 也是一个独立集,且 $I \subsetneq I \cup \{v\}$,这说明 $I$ 不是极大独立集,矛盾。
\end{proof}

这样,我们就证明了该定理的正确性。尽管这样仍然会搜到一些不是极大的独立集,但这样的集合显然少了很多。

\subsubsection{极大独立集的个数}
之前我们只是感性地认识了极大独立集比较少,这里将给出Bron-Kerbosch算法的递归次数上界:
\begin{theorem}
    Bron-Kerbosch算法的递归调用次数为 $O(3^{\frac{n}{3}})$。
\end{theorem}

由此可以得到推论:
\begin{theorem}
    $n$ 阶无向图的极大独立集个数为 $O(3^{\frac{n}{3}})$。
\end{theorem}

这个上界是很容易达到的,构造 $\left\lfloor \frac{n}{3} \right\rfloor$ 个相互独立的三元环即可。但在图随机生成的情况下,这个上界是很不满的。为了说明这一点,笔者对随机图的极大独立集个数进行了研究。

从边数为 $m$ 的 $n$ 阶简单无向图中随机生成一个图 $G = (V, E)$,记 $G$ 的极大独立集个数 $x$,即
\[x=\sum_{S \subseteq V}\mathbb{I}(S \text { is a maximal independent set })\]
其中 $\mathbb{I}(\cdot)$ 是示性函数。

考虑计算 $x$ 的期望值 $\E(x)$。$S$ 是 $G$ 的极大独立集的条件为:
\begin{itemize}
    \item $S$ 是独立集,即对于任意的 $u, v \in S$,$(u, v) \notin E$;
    \item 对于任意的 $v \in V - S$,$V + \{v\}$ 不是独立集,即 $V - S$ 中的每个点至少与 $S$ 中的一个点相邻。
\end{itemize}

用容斥原理,枚举 $k$ 个 $V - S$ 中的点不与 $S$ 中的点相邻。记 $i = |S|$,则剩下 $n - i - k$ 个点可以和 $S$ 中的点连边,以及 $V - S$ 中任意两点(一共 $\frac{1}{2}(n - i)(n - i - 1)$ 对点)可以连边。 则满足 $S$ 是极大独立集的图 $G$ 个数为
\[\sum_{k=0}^{n-i}(-1)^{k}\binom{n - i}{k}\binom{(n-i-k) i+\frac{(n-i)(n-i-1)}{2}}{m}.\]

由于边数为 $m$ 的 $n$ 阶简单图共有 $\binom{\frac{n(n - 1)}{2}}{m}$ 个,故有
\begin{align*}
    E(x)&=\sum_{S \subseteq V} P(\I(S \text { is a maximal independent set })) \\
    &=\sum_{i=0}^{n} \sum_{S \subseteq V,|S|=i}\binom{\frac{n(n - 1)}{2}}{m}^{-1} \sum_{k=0}^{n-i}(-1)^{k}\binom{n - i}{k}\binom{(n-i-k) i+\frac{(n-i)(n-i-1)}{2}}{m}\\
    &=\binom{\frac{n(n - 1)}{2}}{m}^{-1} \sum_{i=0}^{n}\binom{n}{i}\sum_{k=0}^{n-i}(-1)^{k}\binom{n - i}{k}\binom{(n-i-k) i+\frac{(n-i)(n-i-1)}{2}}{m}
\end{align*}

下面给出了一些计算结果(四舍五入):
\begin{center}
    \begin{table}[!h]
        \centering
        \begin{tabular}{c|c|c|c|c|c}
            \hline 
            $E(x)$ & $m = n$ & $m = \left\lfloor \sqrt{3}n \right\rfloor$ & $m = 2n$ & $m = 3n$ & $m = \frac{n^2}{4}$ \\
            \hline 
            $n = 20$  & 84 & 101 & 99 & 81 & 49 \\
            \hline 
            $n = 30$  & 706 & 933 & 909 & 691 & 157 \\
            \hline 
            $n = 40$  & $5.95 \times 10^3$ & $8.67 \times 10^3$ & $8.40 \times 10^3$ & $5.88 \times 10^3$ & 403 \\
            \hline 
            $n = 50$  & $5.02 \times 10^4$ & $8.07 \times 10^4$ & $7.76 \times 10^4$ & $5.01 \times 10^4$ & 891 \\
            \hline 
            $n = 60$  & $4.23 \times 10^5$ & $7.51 \times 10^5$ & $7.18 \times 10^5$ & $4.27 \times 10^5$ & 1779 \\
            \hline 
            $n = 70$  & $3.57 \times 10^6$ & $6.99 \times 10^6$ & $6.64 \times 10^6$ & $3.63 \times 10^6$ & 3291 \\
            \hline 
            $n = 80$  & $3.01 \times 10^7$ & $6.51 \times 10^7$ & $6.14 \times 10^7$ & $3.10 \times 10^7$ & 5730 \\
            \hline 
            $n = 90$  & $2.54 \times 10^8$ & $6.06 \times 10^8$ & $5.68 \times 10^8$ & $2.64 \times 10^8$ & 9506 \\
            \hline 
            $n = 100$  & $2.15 \times 10^9$ & $5.64 \times 10^9$ & $5.25 \times 10^9$ & $2.25 \times 10^9$ & 15154 \\
            \hline
        \end{tabular}            
    \end{table}
\end{center}
    
可见随机情况下,极大独立集的个数远少于 $3^{\frac{n}{3}}$。另外,当 $m$ 接近 $\sqrt{3}n$ 时,独立集个数 $x$ 的期望值 $\E(x)$ 最大,而过于稠密的图,独立集个数相当少。

根据该结论,还可以进一步得出,$x \ge k \cdot \E(x)$ 的概率不超过 $\frac{1}{k}$,所以在大多数情况下随机图的极大独立集个数不会比期望值大太多。

值得注意的是,Bron-Kerbosch算法复杂度不和极大独立集个数直接相关,所以用极大独立集个数的期望值分析Bron-Kerbosch算法的期望运行时间并不准确;事实上,存在复杂度和极大独立集个数直按相关的极大独立集搜索算法。

\subsubsection{应用}
介绍了极大独立集的性质及算法之后,我们来看看它有哪些应用。

\begin{example}[图的3-染色问题]
    给定 $n$ 阶简单无向图 $G = (V, E)$,用三种颜色对 $V$ 中的结点进行染色,使得每条边 $(u, v) \in E$ 的两端点 $u, v$ 颜色不同。满足 $n \le 40$。
\end{example}

图的3-染色问题也是著名的 NP-Hard 问题。朴素的算法是每次枚举一个与已确定颜色的结点相邻的结点颜色,需要枚举 $O(2^n)$ 中情况,无法通过 $n = 40$ 的数据。如何才能更加高效地求解?

先给出一个定理:
\begin{theorem}
    无向图 $G=(V, E)$ 能够3-染色的充要条件是 $G$ 存在一个极大独立集 $I$,使得图 $G - I$ 是二分图\footnote{无向图 $G=(V, E)$ 是二分图 (bipartite graph) 定义为可以将 $V$ 划分为两个集合 $S$ 和 $V - S$,使得每条边的两个端点不在同一个集合内,即 $\forall (u, v) \in E$,$u \in S, v \in V - S$ 或 $u \in V - S, v \in S$。}。
\end{theorem}
\begin{proof}
    (充分性)设 $I$ 为 $G$ 的一个极大独立集,且 $G - I$ 为二分图,根据二分图的性质,存在点集 $X \subseteq V - I$,记 $Y = V - I - X$,使得对任意 $u, v \in X$ 或 $u, v \in Y$,有 $(u, v) \notin E$。

    因为 $I$ 是 $G$ 的独立集,所以 $\forall u, v \in I$,有 $(u, v) \notin E$。

    因此将 $I$ 中的点染色为1,$X$ 中的点染色为2,$Y$ 中的点染色为3是一种合法方案。

    (必要性)设 $G = (V, E)$ 能够3-染色,记 $X, Y, Z$ 分别为染颜色1, 2, 3的点集。

    由定义,对任意 $(u, v) \in E$,结点 $u, v$ 不属于这三个集合中的同一个集合,因此 $X, Y, Z$ 都是独立集。

    如果 $X$ 不是极大独立集,则存在以 $v \in V - X$,使 $X + \{v\}$ 是独立集。将 $v$ 加入点集 $X$,同时,若 $v \in Y$,则将 $v$ 从 $Y$ 中删去;否则 $v \in Z$,将 $v$ 从 $Z$ 中删去。重复此过程直至 $X$ 是极大独立集为止。

    显然此时 $Y, Z$ 仍然是 $G$ 的独立集,即对于 $u, v \in Y$ 或 $u, v \in Z$,有 $(u, v) \notin E$,故 $G[Y \cup Z]$ 是二分图。问题得证。
\end{proof}

判断图是否为二分图以及将其进行染色可以在 $O(m)$ 的时间内解决。因此只需用枚举图 $G$ 的所有极大独立集 $I$,然后判断图 $G - I$ 是否为二分图:
\begin{enumerate}
    \item 若对所有的极大独立集 $I$,图 $G - I$ 都不是二分图,则图 $G$ 不能3-染色。
    \item 若存在一个极大独立集 $I$,使得图 $G - I$ 是二分图,则图 $G$ 能3-染色:将 $I$ 中的结点用颜色 $I$ 染色,$G - I$ 用颜色2和3进行二分图染色即可。
\end{enumerate}

本题中,由于 $G$ 是简单图,$m = O(n^2)$,所以该算法时间复杂度为 $O(3^{\frac{n}{3}n^2})$。

\begin{example}[小Q运动季测试点10]\label{example2}
    给定一个 $n$ 元一次同余方程组
    \[\begin{cases}
        a_{0,0} x_{0}+a_{0,1} x_{1}+\ldots+a_{0, n-1} x_{n-1} \equiv c_{0} \quad \left(\bmod\ b_{0}\right) \\
        a_{1,0} x_{0}+a_{1,1} x_{1}+\ldots+a_{1, n-1} x_{n-1} \equiv c_{1} \quad\left(\bmod\ b_{1}\right) \\
        \ldots & \\
        a_{m-1,0} x_{0}+a_{m-1,1} x_{1}+\ldots+a_{m-1, n-1} x_{n-1} \equiv c_{m-1} \quad\left(\bmod\ b_{m-1}\right)
    \end{cases}\]
\end{example}
求一组解 $(x_1, x_2, \dots, x_n)$ 满足尽量多的方程。

本例中仅讨论测试点10。该测试点中,通过建立图论模型,将每个方程看成一个点,相互冲突的方程间连一条边,可以转化为点数 $n = 90$,边数 $m = 223$ 的无向图的最大独立集问题。由于具体转化过程超出了本文的范围,故略去。

用Bron-Kerbosch算法搜出所有极大独立集,输出其中最大的一个即可。这样做的效率如何呢?

笔者将朴素搜索算法Simple-Search和基于极大独立集的搜索算法Maximal-Search进行比较,两个算法仅使用了最基本的剪枝:将剩余的点全部加入 $I$ 都不大于当前搜到的点集,得到的点集大小都不超过当前搜到的最大的独立集,则剪枝。由于仅仅测试的是Maximal- Search是否有比Simple-Search更优的运行效率,这里并没有加入更多依赖问题性质的剪枝。

对于Simple-Search,笔者的程序经过运行若干小时,仍然只能得到大小为了33的独立集,并且程序未能结束。然而对于Maximal-Search,笔者的程序仅用不到1min就得到了一组大小为 34 的独立集,仅用3min就证明,图的独立数确实为 34。可见用极大独立集进行搜索确实能大幅提高运行效率。

通过加入更多的剪枝优化,可以进一步缩短算法运行时间。

\subsection{基于动态规划的独立集算法}
动态规划是一种高效、灵活的处理问题的方法。在独立集问题中,动态规划不仅能求解最优化类问题(如最大独立集、最大权独立集),还能求解计数类问题(如独立集计数)。下面仍以最大独立集问题为例,但为了体现动态规划的通用性,接下来的讨论将不加入最优性剪枝等仅针对最优化问题的剪枝。

\subsubsection{算法}
如果要使用动态规划求解独立集问题,就需要将问题化为规模更小的子问题。对于独立集,我们有以下两个定理:
\begin{theorem}
    对于无向图 $G = (V, E)$ 和 $V^{'} \subseteq V$,则对于任意 $I \subseteq V^{'}$,$I$ 是 $G$ 的独立集当且仅当 $I$ 是 $G[V^{'}]$ 的独立集。
\end{theorem}
\begin{proof}
    (充分性)当 $I$ 是 $G[V^{'}]$ 的独立集时,对于任意 $(u, v) \in E$,若 $u, v \in V^{'}$,显然 $u, v$ 不同时属于 $I$;若 $u, v$ 有一个不属于 $V^{'}$,不妨设 $u \notin V^{'}$,那么 $u \notin I$。因此 $I$ 是 $G$ 的独立集。

    (必要性)当 $I$ 是 $G$ 的独立集时,由于 $G[V^{'}]$ 的每条边都属于 $G$,故 $G[V^{'}]$ 的每条边至少有一个端点不属于 $I$。因此 $I$ 是 $G[V^{'}]$ 的独立集。
\end{proof}

\begin{theorem}
    对于无向图 $G = (V, E)$ 和 $v \in V$,若 $I \subseteq V$ 且 $v \in I$,则 $I$ 是 $G$ 的独立集当且仅当 $I - \{v\}$ 是 $G[V - \{v\} - N(v)]$ 的独立集。
\end{theorem}
\begin{proof}
    记 $T = V - \{v\} - N(v)$。

    (充分性)当 $I$ 是 $G$的独立集时,$N(v) \cap I = \varnothing$,所以 $I - \{v\} \subseteq T$,因为 $I$ 在 $G$ 中任意两点不相邻,且 $I - \{v\} \subseteq I$,$G[T] \subseteq G$,所以 $I - \{v\}$ 是 $G[T]$ 的独立集;

    (必要性)当 $I - \{v\}$ 是 $G[T]$ 的独立集时,因为 $I - \{v\} \subseteq V - \{v\} - N(v)$,所以 $N(v) \cap I = \varnothing$, 又因为 $G$ 比 $G[T]$ 多的边均与 $v$ 或 $N(v)$ 中的结点相邻,所以 $I$ 是 $G$ 的独立集。
\end{proof}

根据以上两个定理,我们可以用状态压缩的动态规划(DP)对于任意的无向图 $G= (V, E)$ 求出 $G$ 的独立数 $\alpha(G)$。

对点集 $S \subseteq V$,定义 $f(S)$ 为 $S$ 在 $G$ 上的导出子图的独立数,即 $f(S) = \alpha(G[S])$,显然 $f(\varnothing) = 0$。

考虑 $S \neq \varnothing$ 的情況。任取以 $v \in S$,考虑一个点集 $I \subseteq S$。若 $v \notin I$,则 $I$ 是 $G[S]$ 的独立集当且仅当 $I$ 是 $G[S - \{v\}]$ 的独立集:若 $v \in I$,则 $I$ 是 $G[S]$ 的独立集当且仅当 $I - \{v\}$ 是 $G[S - \{v\} - N(v)]$ 的独立集。由此可得:
\begin{equation}\label{eq3-1}
    f(S)=\begin{cases}
        0, & S=\varnothing \\
        \max \{f(S-\{v\}), f(S-\{v\}-N(v))+1\}, \forall v \in S, & S \neq \varnothing
    \end{cases}
\end{equation}

实现时,将图中结点编号为 $0, 1, \dots, n - 1$,结点 $v$ 可以选取 $S$ 中编号最大的点。同样可以使用压位技巧来存集合 $S$。另外,计算 $f$ 可以使用记忆化搜索,状态可以用Hash Table来存储。

该算法不仅能求出独立数,还能求出一个最大独立集,见以下伪代码($f$ 函数为依照(\ref{eq3-1})式定义的记忆化搜索函数):

\begin{algorithm} 
	\caption{Subset-Dynamic-Programming}\label{alg2} 
	\begin{algorithmic}[1]
        \STATE{$S = V$}
        \STATE{$I = \varnothing$}
        \WHILE{$S \neq \varnothing$}
            \STATE{令 $v$ 为 $S$ 中编号最大的点}
            \IF{$f(S - \{v\} > f(S - \{v\} - N(v))) + 1$}
                \STATE{$S \gets S - \{v\}$}
            \ELSE
                \STATE{$S \gets S - \{v\} - N(v)$}
                \STATE{$I \gets I + \{v\}$}
            \ENDIF
        \ENDWHILE
        \RETURN{$I = 0$}
	\end{algorithmic} 
\end{algorithm}

如果直接实现,复杂度为 $O(2^{\frac{n}{2}})$(设Hash Table的的单次操作时间为 $O(1)$,因为在前 $\frac{n}{2}$ 层递归中,每层递归最多2个分支,而递归超过 $\frac{n}{2}$ 层之后,$S$ 中只包含编号前 $\frac{n}{2}$ 的结点, 从而总复杂度为 $O(2^{\frac{n}{2}})$。

对于任意的 $n$,都能构造出使得该算法复杂度为 $\Theta(2^{\frac{n}{2}})$ 的图 $G$,方法是:将结点0和 $\left\lfloor \frac{n}{2} \right\rfloor$ 连边,结点1和 $\left\lfloor \frac{n}{2} \right\rfloor + 1$ 连边,$\cdots\cdots$,结点 $\left\lfloor \frac{n}{2} \right\rfloor - 1$ 和 $2\left\lfloor \frac{n}{2} \right\rfloor - 1$ 连边。这样递归的前 $\left\lfloor \frac{n}{2} \right\rfloor$ 层每层都有2个分支,且所有的分支都不同。

\begin{example}[团的计数]
    给定无向简单图 $G = (V, E)$,求 $G$ 有多少个团。一个团定义为一个点集 $S \subseteq V$,满足 $S$ 中任意两点都有边相连。$n \le 50$。
\end{example}

记 $\bar{G}$ 为 $G$ 的补图,不难发现,$S$ 是 $G$ 的团当且仅当 $S$ 是 $\bar{G}$ 的独立集。

这是因为,如果 $S$ 是 $G$ 的团,那么 $S$ 中的点在 $G$ 中两两相邻,故在 $\bar{G}$ 中两两不相邻, 即 $S$ 是 $\bar{G}$ 的独立集。反之,如果 $S$ 不是 $G$ 的团,即存在两点 $u, v$ 在 $G$ 中不相邻,则 $u, v$ 在 $\bar{G}$ 中相邻,也就是说,$S$ 不是 $\bar{G}$ 的独立集。

因此,问题转化为独立集计数问题——求G的独立集个数。显然这类计数问题无法用搜索优化的策略,不过,使用上述的Subset-Dynamic- Programming算法即可在 $O(2^{\frac{n}{2}})$ 时间内解决问题。

\subsubsection{效率优化}
经过实际测试,Subset-Dynamic-Programming算法运行效率不如优化的搜索算法。为什么?考虑最大独立集问题,该算法的最坏复杂度为 $O(2^{\frac{n}{2}})$,但搜索剪枝时可以直接处理度为1的结点,从而只需要 $O(n)$ 的时间。是否可以类比优化搜索算法,用一些“剪枝”来优化上述DP呢?

答案是肯定的。不过笔者并不打算重复之前的优化——既然用了基于DP的算法,就应当研究更加通用的优化。例如在求最大权独立集时,直接处理度为1的结点是错误的。经过笔者研究,以下优化可以大大提高DP的效率:
\begin{enumerate}
    \item 在状态转移方程中,结点 $v$ 不取 $S$ 中编号最大的点,而取 $G[S]$(注意是导出子图,而不是原图)中度数最大的点;
    \item 当图 $G[S]$ 不连通时,记每个连通块的点集分别为 $S_1, S_2, \dots, S_k$,由于每个连通块是独立的,可以转化为规模更小的子问题解决:
    \[f(S) = \sum_{i = 1}^kf(S_i)\]
    \item 当图 $G[S]$ 不含环时,可以改用树形DP求解。
\end{enumerate}

优化后的DP算法记作Optimized-Subset-Dynamic-Programming。笔者无法分析该算法在最坏情况下的复杂度是多少,但通过对随机图的测试,Optimized-Subset-Dynamic-Programming比之前的Subset-Dynamic-Prograuming快得多,其中优化1、2效果明显,尤其对于较稀疏的图。这是因为较稀疏的图在不断删去度数大的点时,导出子图 $G[S]$ 很容易不连通。

\begin{example}[小Q运动季测试点10]
    题意见例题\ref{example2}
\end{example}

上文己经提到了用Maximal-Search求得该问题的最优解所需的时间。现在我们尝试使用基于动态规划的独立集算法 Subset-Dynamic-Programming,遗憾的是,测试表明,这个算法运行效率并不高,并且由于状态数过多,空问消耗都无法接受。

我们再试一试Optimized-Subset-Dynamic-Programming。令人惊讶的是,这个算法的运行效率非常高——经过笔者测试,该算法只用了 1s 就得到了最优解(大小为 34 的独立集)!并且这个算法运行过程中没有使用依赖任何问题的特殊性的优化(如最优性剪枝)。可见在稀疏图上,Optimized-Subset-Dynamic-Programming 的确是一个优秀的算法。

\subsubsection{与搜索算法的联系}
事实上,如果把这个算法的记忆化去掉(即不用Hash Table存储 $f$ 值),就是一个带优化的搜索算法。这样的搜索算法(记为Optimized-Search)仍然比朴素的搜索算法Simple- Search快,但慢于Optimized-Subset-Dynamic-Programming。

Optimized-Subset-Dynamic-Programming结合了搜素和动态规划的优化思想,不仅有比较强的通用性,实现难度也很小。

然而Optimized-Subset-Dynamic-Programming有一定的缺陷:空间复杂度比较大。而搜索算法Optimized-Search的空间是多项式级别的,支持运行较长时间。因此可以只记忆化较小的 $S$ 的 $f(S)$ 值,剩余部分采用搜索的方法,这样就能在较低的空间需求下解决问题了。

\subsubsection{测试与对比}
笔者在研究出上述动态规划算法及优化之后,将该算法(及优化后的算法)和之前的基于搜索的算法进行了实现,并且用随机图测试了这些算法的期望运行效率。下表的第一行中,$n, m$ 代表 $n$ 阶 $m$ 边随机图,最后一列 $90, 223$ 代表WC2013 《小Q运动季》的测试点10对应的图。表格内的时间代表算法在对应的图上的期望运行时间估计值,“-”表示运行时间过长,未测出。

\begin{center}
    \begin{table}[!h]
        \centering
        \begin{tabular}{c|c|c|c|c}
            \hline
            算法 & $40, 60$ & $50, 85$ & $60, 120$ & $90, 223$ \\
            \hline
            Simple-Search & 2s & - & - & - \\
            \hline
            Maximal-Search & $<0.01$s & $<0.1$s & 1s & - \\
            \hline
            Subset-Dynamic-Programming & $<0.01$s & $<0.1$s & 1s & - \\
            \hline
            Optimized-Subset-Dynamic-Programming-1 & $<0.01$s & $<0.01$s & $<0.01$s & 1s \\
            \hline
            Optimized-Subset-Dynamic-Programming-2 & $<0.01$s & $<0.01$s & $<0.01$s & 1s \\
            \hline
        \end{tabular}            
    \end{table}
\end{center}

Maximal-Search和 Subset-Dynamic-Programming分别采用了 “搜索剪枝”和“记忆化” 的优化,其效果比较接近,Optimized-Subset-Dynamic-Programming则结合了两者的优势, 效率严格高于这两个算法。值得一提的是,Optimized-Subset-Dynaric-Programming-1和Optimized- Subset-Dynamic-Programming-2的区别在于后者加入了优化3(转为树形DP),尽管优化3看起来很高效——把指数级的问题用线性时间解決,但经过测试,两者运行效率几乎无差别。

\section{特殊图的独立集问题}
上文介绍了解决一般图的独立集的基本思想(搜索优化以及动态规划),这些方法复杂度均为指数级。本节中,我们将进一步探讨特殊图的独立集问题——当图本身具有一定特殊性质时,能否用多项式复杂度解决同样的问题?

\subsection{基于图匹配思想的最大独立集算法}
\subsubsection{二分图的最大独立集}
二分图的最大独立集是一个经典问题。我们有以下定理:
\begin{theorem}
    对于 $n$ 阶二分图 $G$,$\alpha(G) = n - v(G)$,其中 $\alpha(G)$,$v(G)$ 分别为图 $G$ 的独立数和匹配数。
\end{theorem}

该定理的证明可以在很多材料中找到,故证明略。

用匈牙利算法或网络流求出二分图 $G =(X, Y, E)$的一个匹配数以 $v(G)$,即可得到 $G$ 的独立数 $\alpha(G)$。另外,用网络流建图后求最小割可以得到一个最大独立集 $I$。

这个算法只能解決最优化类的独立集问题,不能解决更加复杂的问题(妇计数或带有其它限制等)。

\subsubsection{无爪图的最大独立集}
二分图的最大独立集给了我们一个思路求图的最大匹配时,可以通过找增广路不断增加匹配大小,那么求其它图的最大独立集能否也采用增广的方式?遗憾的是,在任意图上,两个独立集的对称差\footnote{集合 $A, B$ 的对称差 $A \Delta B = \{x \mid \I(x \in A) \neq \I(x \in B)\}$}的导出子图不一定是若干条路径或环,所以并不能用找增广路的方法求最大独立集。

不过,在一种特殊的图上,这种算法是可行的。
\begin{definition}
    无爪图(claw-fiee sraph)定义为所有导出子图都不是 $K_{1, 3}$ 的无向图。其中 $K_{1, 3}$ 称为爪(claw),即两部分别含有1个点和3个点的完全二分图。
\end{definition}

无爪图的最大独立集可以用类似一般图匹配的算法来求解。该算法依赖于以下定理:

\begin{theorem}
    设 $I_1, I_2$ 为无爪图 $G$ 的两个独立集,则 $G[I_1\Delta I_2]$ 的每一个连通块都是一条简单路径或简单环。
\end{theorem}
\begin{proof}
    设 $v \in I_1$,则 $N(v) \cap I_1 = \varnothing$,在 $G[I_1\Delta I_2]$ 中,$v$ 的度数为 $|N(v) \cap I_2|$。

    假设存在三个不同的点 $v_1, v_2, v_3 \in N(v) \cap I_2$,因为 $I_2$ 是独立集,所以 $v_1, v_2, v_3$ 两两不相邻,因此 $G[\{v, v_1, v_2, v_3\}]$ 是一个爪,矛盾。

    因此 $|N(v) \cap I_2| \le 2$,即 $G[I_1\Delta I_2]$ 的所有点度数均不超过2,定理得证。
\end{proof}

注意到如果 $|I_1| < |I_2|$,那么 $G[I_1\Delta I_2]$ 必然存在一个连通块 $C$,满足连通块中属于 $I_2$ 的结点比属于 $I_1$ 的结点多。由于 $C$ 中属于 $I_1, I_2$ 的结点交替出现,当 $C$ 为简单环时,$C$ 中属于 $I_1, I_2$ 的结点一样多,而当 $C$ 为简单路径时,$C$ 中属于 $I_1, I_2$ 的结点个数相差1,故必然存在一条路径满足属于 $I_2$ 的点数比属于 $I_1$ 的点数多 1。我们把 $C$ 称为 $I_1$ 的增广路 (augment path)。

这样,就可以类比一般图最大匹配的算法,用增广路算法求无爪图 $G=(V, E)$ 的最大独立集:初始时令 $I = \varnothing$,每次从一个点出发找一条增广路,然后将增广路上的点状态取反,即:原来不属于独立集的点加入独立集,原来属于独立集的点从独立集中删去。该算法的实现类似Edmonds的带花树算法。

\subsection{基于图上阶段划分思想的最大独立集算法}
\subsubsection{分层图上的动态规划}
对于图 $G = (V, E)$,将点集 $V$ 划分为 $k$ 个不相交的集合 $V_1, V_2, \cdots, V_k$,使得对任意 $u \in V_i, v \in V_j$,若 $|i - j| > 1$,则 $(u, v) \notin E$,则称集合序列 $<V_1, V_2, \cdots, V_l>$ 是 $G$ 的一个分层。如果每个 $V_i$ 中的结点都不多,那么可以按 $V_1, V_2, \cdots, V_k$ 顺序进行决策,在每个阶段只需状压一个层的选取情况即可,效率远高于一般图中的对整个图状压DP。

记 $f(i, S)$ 为图 $G[V_1 \cup V_2 \cup \cdots \cup V_i]$ 中包含 $S$ 为子集的最大的独立集,状态转移方程如下:
\[f(i, S)=\left\{\begin{array}{ll}
    -\infty, & S \text { is not independent, } \\
    0, & i=0, \\
    \max \left\{f\left(i-1, S^{\prime}\right) \mid S^{\prime} \subseteq V_{i-1}, S \cup S^{\prime} \text { is independent }\right\}+\left|S^{\prime}\right|, & \text { otherwise }
\end{array}\right.\]

$G$ 的独立数为所有 $G[V_k]$ 的独立集 $I$ 中 $f(k, I)$ 的最大值。该过程同样能求出一个 $G$ 的最大独立集,方法和之前类似,这里不再赘述。

这个算法的时间复杂度为 $O(\sum_{i = 1}^{k - 1}2^{|V_i| + |V_{i + 1|}})$,不过在大多数情况下,每个 $V_i$ 内的独立集个数并不多,实际的时间效率远高于理论上界。

\subsubsection{“k-仙人图”上的动态规划}
\begin{example}
    给定简单无向图 $G = (V, E)$,保证每条边属于且仅属于一个简单环,求 $G$ 的独立数。$|V| < 50,000,\ |E| \le 60,000$。
\end{example}

如果 $G$ 不连通,那么求出 $G$ 的每个连通块的独立数并求和即可。下文假设 $G$ 是连通图。每条边最多属于一个简单环的简单连通图称为“仙人掌”,它有什么特殊的性质呢?

我们不妨大胆尝试一下——任选一个 $r \in V$,以 $r$ 为根对图 $G$ 进行深度优先搜索 (depth-frst search,DFS),得到一个深度优先搜素树(DFS树)$T=(V,E_T)$。显然 $T$ 是 $G$ 的一个生成树。定义树边为属于 $E_T$ 的边,非树边为不属于 $E_T$ 的边(即属于 $E - E_T$ 的边)。

DFS树有一个很重要的性质:
\begin{theorem}
    对任意非树边 $e = (u, v)$,在 $T$ 中或者 $u$ 是 $v$ 的祖先,或者 $v$ 是$u$ 的祖先。
\end{theorem}
\begin{proof}
    设 $e = (u, v)$ 为非树边,且 $u$ 比 $v$ 先访问到,则访问到$u$ 时,假如 $v$ 不在以 $u$ 为根的子树内,那么枚举到 $u$ 的出边 $e$ 时,$v$ 未被访问,因此下一步将沿着边 $e$ 访问到 $v$,从而 $e \in E_T$,与假设矛盾,从而 $u$ 是 $v$ 的祖先。同理,若 $v$ 比 $u$ 先访问到,则 $v$ 是 $u$ 的祖先。
\end{proof}

对于非树边 $e = (u, v)$,若树边 $e^{'}$ 在树 $T$ 中从 $u$ 到 $v$ 的简单路径上,则称树边 $e^{'}$ 被非树边 $e$ 覆盖(cover)。对于仙人掌,我们有:

\begin{theorem}
    每条树边最多被一条非树边覆盖。
\end{theorem}
\begin{proof}
    假设一条树边 $e$ 被两条非树边 $(u_1, v_1)$,$(u_2, v_2)$ 覆盖,则 $(u_1, v_1)$ 和 $T$ 中 $v_1$ 到 $u_1$ 的简单路径构成一个简单环 $C_1$,$(u_2, v_2)$ 和 $T$ 中 $v_2$ 到 $u_2$ 的简单路径构成一个简单环 $C_2$,而 $e$ 同时属于 $C_1$ 和 $C_2$,与仙人掌的定义矛盾。因此 $e$ 最多被一条非树边覆盖。
\end{proof}

这个性质使得我们可以对树 $T$ 进行树形动态规划。初步的想法是:记 $f(i, 0)$ 为 $T$ 中以结点 $i \in V$ 为根的子树内最大的独立集大小,$f(i, 1)$ 为 $i$ 的父结点属于独立集的情况下,$i$ 子树内最大的独立集大小,然而这样不能保证非树边的两端点不同时属于独立集。

考虑添加一维状态,记 $f(i, j, k)$ 为 $i$ 的父结点属于$(j = 1)$或不属于$(j = 0)$独立集, 覆盖 $i$ 与其父结点 $e_i$ 连边的非树边 $e^{'} = (u_i, v_i)$ 顶端结点 $i_i$ 属于 $(j = 1)$ 或不属于 $(j = 0)$ 独立集的情况下,$i$ 子树内最大的独立集大小。当 $e_i$ 不属于环时,$k = 0$。转移时枚举 $i$ 是否属于独立集即可,注意当 $i$ 是 $e_i^{'}$ 的底端结点 $v_i$ 且 $k = 1$ 时,不能选 $i$。状态转移方程如下($Ch_i$,表示 $i$ 的子结点集合):

\begin{itemize}
    \item 当 $j = 1$ 或 $k = 1 \wedge i = v_i$ 时:
    \[f(i, j, k)=\sum_{c \in \mathrm{Ch}_{i}} f\left(c, 0,\I(e_{c}^{\prime}=e_{i}^{\prime}) \cdot k\right)\]
    \item 否则
    \[f(i, j, k)=\max \left\{\sum_{c \in \mathrm{Ch}_{i}} f\left(c, 0,\I\left(e_{c}^{\prime}=e_{i}^{\prime}\right) \cdot k\right), 1+\sum_{c \in \mathrm{Ch}_{i}} f\left(c, 1,\I(e_{c}^{\prime}=e_{i}^{\prime}) \cdot k+\I\left(u_{c}=i\right)\right)\right\}\]
\end{itemize}

用 $O(m)$ 时间预处理所有 $u_i, v_i$ 之后就可以用上述 $O(n)$ 的动态规划求出最大独立集了, 时间复杂度 $O(n + m)$。

笔者在研究上述算法之后,思考这种算法能否进行扩展。经过分析,笔者发现,将该算法做一些简单的修改之后,不仅能处理仙人掌,还能处理每条边所属的简单环个数不多的图。

我们把这样的图称为“k-仙人图”,即每条边最多属于 $k$ 个简单环的图,其中 $k$ 的值比较小。

求解k-仙人图 $G = (V, E)$ 的独立集问题时,同样先取一个点为根对图进行DFS,得到DFS树 $T$。接下来对每个点 $i$,记 $C_i$ 为覆盖 $i$ 与其父结点的连边 $e_i$ 的非树边集合,则有一个性质:

\begin{theorem}
    在k-仙人图中,对于任意的 $i \in V$,$|C_i| \le k$。
\end{theorem}
\begin{proof}
    对于每个 $(u, v) \in C_i$,$(u, v)$ 和 $T$ 中从 $u$ 到 $v$ 的路径都对应了一个包含 $e_i$ 的简单环,因此必然存在 $|C_i|$ 个包含 $e_i$ 的简单环。

    由于包含 $e_i$ 的简单环个数不超过 $k$,故 $|C_i| \le k$。
\end{proof}

接下来,我们利用这个性质设计动态规划。在状态中,需要记录 $C_i$ 中所有边的顶端是否属于独立集,转移方式类似。具体地,记 $f(i, j, S)$ 为 $i$ 的父结点属于 $(j = 1)$ 或不属于 $(j = 0)$ 独立集,且 $C_i$ 内的边的顶端结点构成的集合 $U_i$ 中属于独立集的结点集合为 $S$ 的情況下,$T$ 中以 $i$ 为根的子树内最大的独立集大小,则状态转移方程如下:
\begin{itemize}
    \item 当 $j = 1$ 或 $S \cup \{i\}$ 不是独立集时:
    \[f(i, j, S)=\sum_{c \in \mathrm{Ch}_{i}} f\left(c, 0, S \cap U_{i}\right)\]
    \item 否则:
    \[f(i, j, S)=\max \left\{\sum_{c \in \mathrm{Ch}_{i}} f\left(c, 0, S \cap U_{i}\right), 1+\sum_{c \in \mathrm{Ch}_{i}} f\left(c, 1,(S \cup\{i\}) \cap U_{i}\right)\right\}\]
\end{itemize}

和之前的算法类似,当 $k$ 视为常数时,该算法的复杂度为 $O(n)$。

事实上,“k-仙人图”这个条件过于宽松,只要满足覆盖每条树边的非树边数目均不超过$k$(甚至只要 $\sum_{v \in V}2^{|C_v|}$ 不大),这个算法的效率都是很高的。

该算法可以拓展到更复杂的问题,如独立集计数。

\begin{example}[子集计数问题测试点7,8]
    给定无向图 $G=(V, E)$,$|V| = n$,$|E| = m$。求有多少个子集 $V^{'} \subseteq V$ 满足 $|V^{'}| = k$ 且 $\forall (u, v) \in E$,$u \notin V^{'} \vee v \notin V^{'}$。由于答案可能很大,只需输出答案对 $p$ 取模的结果。提交答案题。
\end{example}

本例中仅讨论测试点7,8。这两个测试点满足 $G$ 是连通图数据规模如下表:

\begin{center}
    \begin{table}[!h]
        \centering
        \begin{tabular}{c|c|c|c|c}
            \hline 
            测试点编号 & $n = $ & $m = $ & $k = $ & $p = $\\
            \hline
            7 & 4998 & 5022 & 666 & 1000000009\\
            \hline
            8 & 11986 & 12011 & 1098 & 1000000007\\
            \hline
        \end{tabular}            
    \end{table}
\end{center}

问题求的是 $G$ 中大小为 $k$ 的独立集个数。由于 $G$ 是连通图,且 $m - n$ 很小,可以发现 $G$ 可以通过往一个树中加入 $m - n + 1$ 条边得到。

如果构出 $G$ 的DFS树之后,暴力枚举每条非树边的一端点是否选取,然后用树形动态规划统计独立集个数,可以较快通过测试点7,但测试点8需要运行的时间太长,需要优化。根据本节中提到的思想,每条边所属的简单环的个数不多,因此可以采用上述算法解决。

记
\begin{itemize}
    \item $C_i$ 为满足 $e \in E^{'}$,$u_e$ 为 $i$ 的祖先(不含 $i$,$v_e$ 在以 $i$ 为根的子树内的 $u_e$ 集合
    \item $f(i, j, S, s)$ 为满足 $C_i$ 中属于独立集的结点集合为 $S$ 的前提下,$i$ 的前 $j$ 个子树中,大小为 $s$ 的独立集个数
    \item $g(i, j, S, s)$ 为满足 $C_i$ 中属于独立集的结点集合为 $S$ 的前提下,$i$ 以及 $i$ 的前 $j$ 个子树中,大小为 $s$ 的独立集个数
    \item $F(i, S, s) = f(i, t_i, S, s)$,$G(i, S, s) = g(i, t_i, S, s)$,这里  $t_i$ 为 $i$ 的子节点数
\end{itemize}

对于 $f(i, j, S, s)$,设 $i$ 和 $i$ 的前 $j - 1$ 个子树内共有 $s_1$ 个点,第 $j$ 个子树内有 $s_2$ 个点,第 $j$ 个子节点为 $c_j$,则
\[f(i, j, S, s)=\sum_{x=\max \left\{s-s_{1}, 0\right\}}^{\min \left\{s, s_{2}\right\}} f(i, j-1, S, s-x) \cdot G\left(c_{j}, S \cap C_{c_{j}}, x\right),\quad 0 \leq s \leq s_{1}+s_{2}\]

对于 $g(i, j, S, s)$,如果存在边 $e \in E^{'}$ 使得 $u_e \in S$ 且 $v_e = i$,那么 $i$ 不能属于独立集,有
\[g(i, j, S, s) = f(i, j, S, s)\]
否则,存在包含 $i$ 的独立集,这样的独立集个数记为 $g^{'}(i, j, S, s)$,则
\[g^{'}(i, j, S, s)=\sum_{x=\max \left\{s-s_{1}, 0\right\}}^{\min \left\{s-1, s_{2}\right\}} g^{'}(i, j-1, S, s-x) \cdot F\left(c_{j},(S \cup\{i\}) \cap C_{c_{j}}, x\right), \quad 0 \leq s \leq s_{1}+s_{2}\]
所以
\[g(i, j, S, s) = g^{'}(i, j, S, s) + f(i, j, S, s), \quad 0 \leq s \leq s_{1}+s_{2}\]
边界为 $f(i, 0, S, 0) = g^{'}(i, 0, S, 1) = 1$,未定义的状态值均为0。答案就是 $G(r, \varnothing, k)$。

可以只计算满足 $s \le k$ 的状态,复杂度 $O(k\sum_{v \in V}2^{|C_v|})$,其中 $|C_v|$ 通常在12以内。整个计算都在模 $p$ 意义下进行即可,内存可以动态分配。

值得注意的是,选取不同的点 $r \in V$ 当根,以及用不同的顺序进行DFS,运行效率是不同的。可以选择一个根 $r$ 进行DFS,使得
\[\sum_{v \in V}2^{|C_v|}\]
最小,然后再执行上述算法,以降低时间复杂度。经过实测,测试点7的状态数约为 $5\times 10^5$,可以在1s内通过该测试点;测试点8的状态数约为 $10^8$,可以在 $\SI{2}{min}$ 内通过该测试点。


\section{总结}
NP-Hard问题的算法优化方法数不胜数,本文仅仅提到了若干种独立集问题的优化算法,这些方法解决的问题相类似,但思想各有区别——针对普通的最优化问题(如最大独立集),可以用带最优性剪枝的搜素算法减少枚举量;针对计数或有额外约束的问题(如独立集计数),可以用状态压缩动态规划,通过优化状态数来提高运行效率;针对可 “增广” 的图以及具有明显阶段性的图,又可以用多项式复杂度的算法来高效完成。

同时,本文对几种算法在随机情况下的运行时间进行了分析和比较,让大家对独立集问题求解的效率有更进一步的认识。


% \nocite{*}
% \printbibliography[heading=bibintoc, title=\ebibname]

% \appendix
% \appendixpage
% \addappheadtotoc




\end{document}
