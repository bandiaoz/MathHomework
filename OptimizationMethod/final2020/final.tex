% \documentclass[12pt]{article}
\documentclass[12pt]{ctexart}
\usepackage[utf8]{inputenc}

\usepackage[english]{babel}
\usepackage[dvips]{epsfig}
\usepackage{amsmath}
\usepackage{amssymb}
\usepackage{amsfonts}
\usepackage{amsthm}
\usepackage{amsbsy}
\usepackage{amsgen}
\usepackage{amscd}
\usepackage{amsopn}
\usepackage{amstext}
\usepackage{amsxtra}
\usepackage{mathrsfs}
\usepackage{enumitem}
\usepackage{graphicx}
\usepackage{verbatim}
\usepackage{epstopdf}
\usepackage{float}
\usepackage[all,cmtip]{xy}
\usepackage{accents}
\usepackage{sseq}
\usepackage{url}
\usepackage{hyperref}
\usepackage{makeidx}
\usepackage{siunitx}
\usepackage{xcolor}

%%%%%%%%% 版面设置 %%%%%%%%%%%%%%%%%%%%%%%%%%%%%%%%%%%%%%
\usepackage{geometry}
\usepackage{titlesec}
\usepackage{fancyhdr}\pagestyle{empty}
\titleformat*{\section}{\large\bfseries}

%
\geometry{
	a4paper,
	total={170mm,240mm},
	left=20mm,
	top=30mm,
}

%Bitte nicht einstellen
\renewcommand{\figurename}{Abbildung}
\renewcommand{\tablename}{Tabelle}
\pagestyle{fancyplain}
\headheight 35pt
\lhead{\name}
\chead{\textbf{\Large \Title}}
\rhead{\due\\\today}
\lfoot{}
\cfoot{}
\rfoot{\small\thepage}
\headsep 1.5em

%%%%%%%%%%%%%%%%%%%%%%%%%%%%%%%%%%%%%%%%%%%%%%%%%%%%%%

\newtheorem{thm}{Theorem}[section]

% 定义解题环境
\theoremstyle{remark}
\newtheorem{remark}[thm]{Remark}
% \newtheorem{remark}[thm]{Remark}
\newtheorem{observation}[thm]{Observation}

\theoremstyle{definition}
\newtheorem{problem}{\text{}}
\newtheorem{Problem}{\text{}}
\newtheorem*{solution}{解}
\newtheorem*{Answer}{Answer}

%%%%%%%%%%%%%%%%%%%%%%%%%%%%%%%%%%%%%%%%%%%%%%%%%%%%%%%%%%%%%%%%%%
\newcommand\name{陈景龙22120307}
\newcommand\due{-}
\newcommand{\emptyline}{\vspace{0.6\baselineskip}}

\usepackage{array,tabularx}

\usepackage{zhnumber} % change section number to chinese
\renewcommand\thesection{\zhnum{section}}
\renewcommand\thesubsection{\arabic{section}}


\newcommand\Title{2020最优化方法final}

\newcommand\subject{\operatorname{subject\ to}}

\begin{document}

\section{(16分)}
\begin{problem}
    你知道几种求解无约束优化问题 $\min f(x)$的迭代算法?请列出三种及其相应的搜索方向的迭代公式.
\end{problem}

\begin{problem}
    取初始点 $x^{(0)} = (1, 1)^t$,采用牛顿法求解下面的无约束优化问题:
    \[\min f(x) = 2x_1^2 + x_2^2 - 4x_1 + 2x_2\]
    写出迭代步骡,并解释说明最终得到的迭代点就是最优解.
\end{problem}

\section{(18分)}
\begin{problem}
    考虑约束优化问题(P1):\[\begin{cases}
        \min \quad &x_1x_2\\
        \subject \quad &2x_1 - x_2 - 3 = 0
    \end{cases}\]
    \begin{enumerate}
        \item 给定 $\bar{x} = (\frac{3}{4}, -\frac{3}{2})^t$,利用约束优化问题局部解的一阶必要条件和二阶充分条件判断 $\bar{x}$ 是否是(P1)的局部最优解?
        \item 定义外罚函数为 \[G(x, c) = x_1x_2 + \frac{c}{2}(2x_1 - x_2 - 3)^2\]试用外罚函数法求解(P1),并说明产生的序列趋向点 $\bar{x}$.
    \end{enumerate}
\end{problem}

\section{(20分)}
\begin{problem}
    考虑下面的线性规划问题(P2):\[\begin{cases}
        \max \quad &2x_1 - x_2 + x_3\\
        \subject \quad &3x_1 + x_2 + x_3 \le b_1\\
        &x_1 - x_2 + 2x_3 \le b_2\\
        &x_1 + x_2 - x_3 \le b_3\\
        &x_1 \ge 0, x_2 \ge 0, x_3 \ge 0
    \end{cases}\]
    利用单纯形法求解(P2)得到如下最优单纯形表:

    \begin{center}\begin{tabularx}{30em}%
        {|*{8}{>{\centering\arraybackslash}X|}}
        \hline
        基 & $x_1$ & $x_2$ & $x_3$ & $x_4$ & $x_5$ & $x_6$ & RHS \\ \hline
         & () & () & () & () & () & () & () \\ \hline
        $x_4$ & () & () & () & () & $-1$ & $-2$ & $10$ \\ \hline
        $x_1$ & () & () & () & () & $1 / 2$ & $1 / 2$ & $15$ \\ \hline
        $x_2$ & () & () & () & () & $-1 / 2$ & $1 / 2$ & $5$ \\ \hline
    \end{tabularx}\end{center}

    试回答下面的问题:\begin{enumerate}
        \item 确定 $b_1, b_2, b_3$ 的值,并把最优表补充完整.
        \item 写出(P2)的对偶问题并根据给出的最优表求其对偶问题的最优解.
    \end{enumerate}
\end{problem}

\section{(28分)}
\begin{problem}
    设 $S \subseteq \mathbb{R}^n$,函数 $f: S \to R$ 二阶连续可微,考虑约束优化问题(P3)\[\begin{cases}
        \min \quad &f(x)\\
        \subject \quad &x \in S
    \end{cases}\]
    \begin{enumerate}
        \item 写出函数 $f$ 是凸函数的定义,并列出你所知道的判定函数 $f$ 是凸函数的充要条件.约束优化问题(P3)在什么条件下是凸规划?对于凸规划,你知道有什么好的性质?
        \item 设 $f(x_1, x_2) = (x_2 - x_1^2)^2, S = \left\{(x_1, x_2) \mid -1 < x_1 < 1, -1 < x_2 < 1\right\}$,判断函数 $f(x_1, x_2)$ 是否为 $S$ 上的凸函数?说明理由.
        \item 考虑如下优化问题(P4):\[\begin{cases}
            \min \quad &x_1^2 - 5x_1 + 4x_2\\
            \subject \quad &2 + x_1 - x_2 \ge 0\\
            &x_1 - 2 \le 0\\
            &x_1 - (x_2 - 3)^2 + 2 \ge 0
        \end{cases}\]
        (P4)是否为凸规划?说明理由.根据最优性条件求(P4)的最优解.
    \end{enumerate}
\end{problem}

\section{(18分)}
\begin{problem}
    设 $Q \in \mathbb{R}^{n \times n}$ 对称设定,$b \in \mathbb{R}^n$ 且 $b\neq 0$,考虑非线性规划问题(P5):\[\begin{cases}
        \min \quad &\frac{1}{2}x^tQx\\
        \subject \quad &x \ge b
    \end{cases}\]
    试回答下面的问题:\begin{enumerate}
        \item 写出(P5)的 Lagrange 对偶规划. 
        \item 设 $x^*$是(P5)的最优解,证明 $x^*$ 与 $x^* - b$ 关于 $Q$ 共轭.
    \end{enumerate}
\end{problem}

\end{document}