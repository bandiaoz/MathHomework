% \documentclass[12pt]{article}
\documentclass[12pt]{ctexart}
\usepackage[utf8]{inputenc}

\usepackage[english]{babel}
\usepackage[dvips]{epsfig}
\usepackage{amsmath}
\usepackage{amssymb}
\usepackage{amsfonts}
\usepackage{amsthm}
\usepackage{amsbsy}
\usepackage{amsgen}
\usepackage{amscd}
\usepackage{amsopn}
\usepackage{amstext}
\usepackage{amsxtra}
\usepackage{mathrsfs}
\usepackage{enumitem}
\usepackage{graphicx}
\usepackage{verbatim}
\usepackage{epstopdf}
\usepackage{float}
\usepackage[all,cmtip]{xy}
\usepackage{accents}
\usepackage{sseq}
\usepackage{url}
\usepackage{hyperref}
\usepackage{makeidx}
\usepackage{siunitx}
\usepackage{xcolor}
\usepackage{physics}

%%%%%%%%% 版面设置 %%%%%%%%%%%%%%%%%%%%%%%%%%%%%%%%%%%%%%
\usepackage{geometry}
\usepackage{titlesec}
\usepackage{fancyhdr}\pagestyle{empty}
\titleformat*{\section}{\large\bfseries}

%
\geometry{
	a4paper,
	total={170mm,240mm},
	left=20mm,
	top=30mm,
}

%Bitte nicht einstellen
\renewcommand{\figurename}{Abbildung}
\renewcommand{\tablename}{Tabelle}
\pagestyle{fancyplain}
\headheight 35pt
\lhead{\name}
\chead{\textbf{\Large \Title}}
\rhead{\due\\\today}
\lfoot{}
\cfoot{}
\rfoot{\small\thepage}
\headsep 1.5em

%%%%%%%%%%%%%%%%%%%%%%%%%%%%%%%%%%%%%%%%%%%%%%%%%%%%%%

\newtheorem{thm}{Theorem}[section]

% 定义解题环境
\theoremstyle{remark}
\newtheorem{remark}[thm]{Remark}
\newtheorem{theorem}{Theorem}
\newtheorem{observation}[thm]{Observation}

\theoremstyle{definition}
\newtheorem{problem}{\text{}}
\newtheorem{Problem}{\text{Problem}}
\newtheorem*{solution}{解}
\newtheorem*{Answer}{Answer}
\newtheorem{example}{Example} 

%%%%%%%%%%%%%%%%%%%%%%%%%%%%%%%%%%%%%%%%%%%%%%%%%%%%%%%%%%%%%%%%%%
\newcommand\name{陈景龙22120307}
\newcommand\due{-}
\newcommand{\emptyline}{\vspace{0.6\baselineskip}}

\newcommand\Title{最优化方法第6次作业}


\begin{document}

\begin{problem}
    写出下列原问题的对偶问题
    \[\begin{cases}
        \max \quad &4x_1 - 3x_2 + 5x_3 \\
        s.t. \quad &3x_1 + x_2 + 2x_3 \le 15\\
        &-x_1 + 2x_2 - 7x_3 \ge 3\\
        &x_1 + x_3 = 1\\
        &x_1, x_2, x_3 \ge 0
    \end{cases}\]
\end{problem}
\begin{solution}
    对偶问题如下
    \[\begin{cases}
        \min \quad &15w_1 + 3w_2 + w_3\\
        s.t. \quad &3w_1 - w_2 + w_3 \ge 4\\
        &w_1 + 2w_2 \ge -3\\
        &2w_1 - 7w_2 + w_3 \ge 5\\
        &w_1 \ge 0, w_2 \le 0, w_3\ \text{free}
    \end{cases}\]
\end{solution}

\begin{problem}
    写出下列原问题的对偶问题
    \[\begin{cases}
        \min \quad &-4x_1 - 5x_2 - 7x_3 + x_4 \\
        s.t. \quad &x_1 + x_2 + 2x_3 - x_4 \ge 1\\
        &2x_1 - 6x_2 + 3x_3 + x_4 \le -3\\
        &x_1 + 4x_2 + 3x_3 + 2x_4 = -5\\
        &x_1, x_2, x_4 \ge 0
    \end{cases}\]
\end{problem}
\begin{solution}
    对偶问题如下
    \[\begin{cases}
        \min \quad &w_1 - 3w_2 - 5w_3\\
        s.t. \quad &w_1 + 2w_2 + w_3 \le -4\\
        &w_1 - 6w_2 + 4w_3 \le -5\\
        &2w_1 + 3w_2 + 3w_3 = -7\\
        &-w_1 + w_2 + 2w_3 \le 1\\
        &w_1 \ge 0, w_2 \le 0, w_3 \ \text{free}
    \end{cases}\]
\end{solution}

\begin{problem}
    给定原问题
    \[\begin{cases}
        \min \quad &4x_1 + 3x_2 + x_3 \\
        s.t. \quad &x_1 - x_2 + x_3 \ge 1\\
        &x_1 + 2x_2 - 3x_3 \ge 2\\
        &x_1, x_2, x_3 \ge 0
    \end{cases}\]
    已知对偶问题的最优解 $(w_1, w_2) = (\frac{5}{3}, \frac{7}{3})$,利用对偶性质求原问题的最优解。
\end{problem}
\begin{solution}
    对偶问题如下:
    \[\begin{cases}
        \max \quad &w_1 + 2w_2\\
        s.t. \quad &w_1 + w_2 \ge 4\\
        &-w_1 + 2w_2 \ge 3\\
        &w_1 - 3w_2 \ge 1\\
        &w_1 \ge 0, w_2 \ge 0
    \end{cases}\]
    得到对偶问题的最优解为 $w_1 = \frac{5}{3} > 0, w_2 = \frac{7}{3} > 0$,由互补松弛定理可得,$x_3 = 0$,从而
    \[\begin{cases}
        x_1 - x_2 + x_3 &= 1\\
        x_1 + 2x_2 - 3x_3 &= 2\\
        x_3 &= 0
    \end{cases}\]
    解得 $x^* = (\frac{4}{3}, \frac{1}{3}, 0), f_{min}^* = \frac{19}{3}$
\end{solution}

\begin{problem}
    给定下列线性规划问题
    \[\begin{cases}
        \max \quad &10x_1 + 7x_2 + 30x_3 + 2x_4 \\
        s.t. \quad &x_1 - 6x_3 + x_4 \le -2\\
        &x_1 + x_2 + 5x_3 - x_4 \le -7\\
        &x_2, x_3, x_4 \le 0
    \end{cases}\]
    \begin{enumerate}
        \item 写出上述原问题的对偶问题
        \item 用图解法求对偶问题的最优解
        \item 利用对偶问题的最优解及对偶性质求原问题的最优解和目标函数的最优值
    \end{enumerate}
\end{problem}
\begin{solution}
    \begin{enumerate}
        \item 对偶问题如下 \[\begin{cases}
            \min \quad &-2w_1 - 7w_2\\
            s.t. \quad &w_1 + w_2 = 10\\
            &w_2 \le 7\\
            &-6w_1 + 5w_2 \le 30\\
            &w_1 - w_2 \le 2\\
            &w_1 \ge 0, w_2 \ge 0
        \end{cases}\]
        \item 对偶问题的可行域是直线 $w_1 + w_2 = 10$ 上的一线段,最优解为 $(w_1, w_2) = (3, 7)$,最优值为 $f_{min} = -55$
        \item 由于对偶问题的最优解中,$w_1 > 0, w_2 > 0$,以及对偶问题约束的后两个条件没有取到等号,由互补松弛定理可得
        \[\begin{cases}
            x_1 - 6x_3 + x_4 = -2\\
            x_1 + x_2 + 5x_3 - x_4 = -7\\
            x_3 = 0\\
            x_4 = 0
        \end{cases}\]
        解得 $x^* = (-2, -5, 0, 0), f_{max} = -55$。
    \end{enumerate}
\end{solution}

\begin{problem}
    给定线性规划问题
    \[\begin{cases}
        \min \quad &5x_1 + 21x_3 \\
        s.t. \quad &x_1 - x_2 + 6x_3 \ge b_1\\
        &x_1 + x_2 + 2x_3 \ge 1\\
        &x_1, x_2, x_3 \ge 0
    \end{cases}\]
    其中 $b_1$ 是某一个正数,已知这个问题的一个最优解为 $(x_1, x_2, x_3) = (\frac{1}{2}, 0, \frac{1}{4})$。
    \begin{enumerate}
        \item 写出对偶问题。
        \item 求对偶问题的最优解。
    \end{enumerate}
\end{problem}
\begin{solution}
    \begin{enumerate}
        \item 对偶问题如下$\begin{cases}
            \max \quad &b_1w_1 + w_2\\
            s.t. \quad &w_1 + w_2 \le 5\\
            & -w_1 + w_2 \le 0\\
            &6w_1 + 2w_2 \le 21\\
            &w_1 \ge 0, w_2 \ge 0
        \end{cases}$
        \item 原问题中 $x_1 > 0, x_3 > 0$,由互补松弛定理可得\[\begin{cases}
            w_1  + w_2 = 5\\
            6w_1 + 2w_2 = 21
        \end{cases}\]
        解得 $w_1 = \frac{11}{4}, w_2 = \frac{9}{4}, g_{max} = f_{min} = \frac{31}{4}$。
    \end{enumerate}
\end{solution}

\end{document}