% \documentclass[12pt]{article}
\documentclass[12pt]{ctexart}
\usepackage[utf8]{inputenc}

\usepackage[english]{babel}
\usepackage[dvips]{epsfig}
\usepackage{amsmath}
\usepackage{amssymb}
\usepackage{amsfonts}
\usepackage{amsthm}
\usepackage{amsbsy}
\usepackage{amsgen}
\usepackage{amscd}
\usepackage{amsopn}
\usepackage{amstext}
\usepackage{amsxtra}
\usepackage{mathrsfs}
\usepackage{enumitem}
\usepackage{graphicx}
\usepackage{verbatim}
\usepackage{epstopdf}
\usepackage{float}
\usepackage[all,cmtip]{xy}
\usepackage{accents}
\usepackage{sseq}
\usepackage{url}
\usepackage{hyperref}
\usepackage{makeidx}
\usepackage{siunitx}
\usepackage{xcolor}
\usepackage{physics}

%%%%%%%%% 版面设置 %%%%%%%%%%%%%%%%%%%%%%%%%%%%%%%%%%%%%%
\usepackage{geometry}
\usepackage{titlesec}
\usepackage{fancyhdr}\pagestyle{empty}
\titleformat*{\section}{\large\bfseries}

%
\geometry{
	a4paper,
	total={170mm,240mm},
	left=20mm,
	top=30mm,
}

%Bitte nicht einstellen
\renewcommand{\figurename}{Abbildung}
\renewcommand{\tablename}{Tabelle}
\pagestyle{fancyplain}
\headheight 35pt
\lhead{\name}
\chead{\textbf{\Large \Title}}
\rhead{\due\\\today}
\lfoot{}
\cfoot{}
\rfoot{\small\thepage}
\headsep 1.5em

%%%%%%%%%%%%%%%%%%%%%%%%%%%%%%%%%%%%%%%%%%%%%%%%%%%%%%

\newtheorem{thm}{Theorem}[section]

% 定义解题环境
\theoremstyle{remark}
\newtheorem{remark}[thm]{Remark}
\newtheorem{theorem}{Theorem}
\newtheorem{observation}[thm]{Observation}

\theoremstyle{definition}
\newtheorem{problem}{\text{}}
\newtheorem{Problem}{\text{Problem}}
\newtheorem*{solution}{解}
\newtheorem*{Answer}{Answer}
\newtheorem{example}{Example} 

%%%%%%%%%%%%%%%%%%%%%%%%%%%%%%%%%%%%%%%%%%%%%%%%%%%%%%%%%%%%%%%%%%
\newcommand\name{陈景龙22120307}
\newcommand\due{-}
\newcommand{\emptyline}{\vspace{0.6\baselineskip}}

\newcommand\Title{最优化方法第11次作业}

\newcommand\subject{\operatorname{subject\ to}}

\begin{document}
\begin{problem}
    用基于精确线搜索的最速下降法求解下述无约束优化问题\[\min f(x_1, x_2) = x_1^2 + x_2^2 - x_1 + 2x_2\]
    其中初始点为 $y_0 = (0, 0)^t$,精度为 $\varepsilon = 0.001$.
    \Answer $\nabla f(x) = (2x_1 - 1, 2x_2 + 2)^t$
    第一次迭代
    \begin{itemize}
        \item $d^{(0)} = -\nabla f(x^{(0)}) = (1, -2)^t$
        \item $\lambda = \underset{\lambda \ge 0}{\arg \min}\varphi(\lambda) = \underset{\lambda \ge 0}{\arg \min} f(x^{(0)} + \lambda d^{(0)}) = \frac{1}{2}$
        \item $x^{(1)} = (\frac{1}{2}, -1)^t$
    \end{itemize}
    $d^{(1)} = (0, 0)^t, \norm{d} = 0 \le \varepsilon$,所以 $x^* = (\frac{1}{2}, -1)^t, f^* = 0$.
\end{problem}

\begin{problem}
    考虑二次函数 $f(x) = \frac{1}{2}x^tAx + b^tx + c$,其中 $A \in \mathbb{R}^{n \times n}$ 为对称正定矩阵,设函数的极小值点为 $x^*$,并设 $x_0 = x^* + ts$,其中 $s$ 为函数 $f$ 的Hesse矩阵的某特征值的特征向量.若以 $x_0$ 为初始点,试证明基于精确线搜索的最速下降法与牛顿法等价.
    \begin{proof}v
        $\nabla f(x) = Ax + b, \nabla^2 f(x) = A$

        使用牛顿法:$x^{(1)} = x^{(0)} - (\nabla^2 f(x))^{-1}\nabla f(x) = x^{(0)} - A^{-1}(Ax^{(0)} + b) = -A^{-1}b$
        
        使用最速下降法:$d^{(0)} = -\nabla f(x^{(0)}) = -Ax^{(0)} - b, f(x^{(1)}) = \frac{1}{2}x^{(1)t}Ax^{(1)} + b^tx^{(1)} + c$, 令 $f(x^{1})$ 最小可得 $x^{(1)} = -A^{-1}b$
        
        故最速下降法与牛顿法等价.
    \end{proof}
\end{problem}

\begin{problem}
    用共轭梯度法求解下列问题\[\min \frac{1}{2}x_1^2 + x_2^2\]取初始点 $x^{(1)} = (4, 4)^t$ 
    \Answer $f(x) = \frac{1}{2}x_1^2 + x_2^2, \nabla f(x) = (x_1, 2x_2)^t$.

    第一次迭代:$g_1 = (4, 8)^t, d^{(1)} = -g_1 = (-4, -8)^t, x^{(2)} = x^{(1)} + \lambda_1d^{(1)} = (\frac{16}{9}, -\frac{4}{9})^t$

    第二次迭代:$g_2 = (\frac{16}{9}, -\frac{8}{9}), d^{(2)} = -g_2 + \beta_1d^{(1)} = (-\frac{160}{81}, \frac{40}{81}), x^{(3)} = x^{(2)} + \lambda_2d^{(2)} = (0, 0)^t$

    第三次迭代:$g_3 = (0, 0), \norm{g_3} = 0$,故 $\bar{x} = (0, 0), f^* = 0$.
\end{problem}

\begin{problem}
    总结最速下降法、牛顿法及共轭梯度法的基本思想和计算步骤.
    \Answer 带精确线搜索的最速下降法 \begin{enumerate}
        \item 给定初始点 $x^{(1)} \in E^n$,允许误差 $\varepsilon > 0$,置 $k = 1$.
        \item 取搜索方向:$d^{(k)} = -\nabla f(x^{(k)})$
        \item 若 $\norm{d^{(k)}} \le \varepsilon$,则停止计算;否则,从 $x^{(k)}$ 出发,沿 $d^{(k)}$ 进行一维搜索,求 $\lambda_k$,使 \[f(x^{(k)} + \lambda_k d^{(k)}) = \min_{\lambda \ge 0} f(x^{(k)} + \lambda d^{(k)})\]
        \item 令 $x^{(k + 1)} = x^{(k)} + \lambda_k d^{(k)}$,置 $k := k + 1$,返回2
    \end{enumerate}

    牛顿法计算步骤:\begin{enumerate}
        \item 给定初始点 $x^{(0)} \in E^n$,允许误差 $\varepsilon > 0$,置 $k = 0$
        \item 若 $\norm{\nabla f(x^{(k)})} < \varepsilon$,则停止计算;
        \item $x^{(k + 1)} = x^{(k)} - \left(\nabla^2f(x^{(k)})\right)^{-1}\nabla f(x^{(k)})$,置 $k:=k + 1$,返回2.
    \end{enumerate}

    FR共轭梯度法(二次凸函数)\begin{enumerate}
        \item 给定初始点 $x^{(1)}$,置 $k = 1$
        \item 计算 $g_k = \nabla f(x^{(k)})$,若 $\norm{g_k} = 0$,则停止计算,否则进行下一步
        \item 令 $d^{(k)} = -g_k + \beta_{k - 1}d^{(k - 1)}$,其中,当 $k = 1$ 时,$\beta_0 = 0$,当 $k > 1$ 时,$\beta_{k - 1} = \frac{d^{(k - 1)t}Ag_k}{d^{(k - 1)t}Ad^{(k - 1)}} = \frac{\norm{g_k}^2}{\norm{g_{k - 1}}^2}$
        \item 令 $x^{(k + 1)} = x^{(k)} + \lambda_kd^{(k)}$,其中 $\lambda_k = \frac{g_k^tg_k}{d^{(k)t}Ad^{(k)}}$
        \item 若 $k = n$,则停止计算,否则置 $k = k + 1$,返回步骤2.
    \end{enumerate}
\end{problem}

\begin{problem}
    $\min x_1^2 + 2x_2^2 - 2x_1x_2 + 2x_2 + 2$,取初始点 $x^{(1)} = (0, 0)^t$.
    \Answer $\nabla f(x) = (2x_1 - 2x_2, 4x_2 - 2x_1 + 2)^t$
    
    共轭梯度法:
    
    第一次迭代:$g_1 = (0, 2)^t, d^{(1)} = -g_1 = (0, -2)^t, x^{(2)} = x^{(1)} + \lambda_1d^{(1)} = (0, -\frac{1}{2})^t$

    第二次迭代:$g_2 = (1, 0)^t, d^{(2)} = -g_2 + \beta_1d^{(1)} = (-1, -\frac{1}{2})^t, x^{(3)} = x^{(2)} + \lambda_2d^{(2)} = (-1, -1)^t$

    第三次迭代:$g_3 = (0, 0)^t, \norm{g_3} = 0$,故 $\bar{x} = (-1, -1)^t, f^* = 1$

    牛顿法:

    $x^{(2)} = x^{(1)} - G_1^{-1}g_1 = (-1, -1)^t, \norm{g_2} = 0$,故 $\bar{x} = (-1, -1)^t$.
\end{problem}

\begin{problem}
    $\min 2x_1^2 + 2x_1x_2 + 5x_2^2$,取初始点 $x^{(1)} = (2, -2)^t$.
    \Answer $\nabla f(x) = (4x_1 + 2x_2, 2x_1 + 10x_2)^t$

    共轭梯度法:

    第一次迭代:$g_1 = (4, -16)^t, d^{(1)} = -g_1 = (-4, 16)^t, x^{(2)} = x^{(1)} + \lambda d^{(1)} = (\frac{57}{37}, -\frac{6}{37})$

    第二次迭代:$g_2 = (\frac{216}{37}, \frac{54}{37})^t, d^{(2)} = -g_2 + \beta d^{(1)}, x^{(3)} = x^{(2)} + \lambda_2 d^{(2)} = (0, 0)^t$

    第三次迭代:$g_3 = (0, 0), \norm{g_3} = 0$,故 $\bar{x} = (0, 0)^t$.

    牛顿法:

    $x^{2} = x^{1} - G_1^{-1}g_1 = (0, 0), \norm{g_2} = 0$,故 $\bar{x} = (0, 0)^t$.
\end{problem}

\begin{problem}
    $\min x_1^2 + 2x_2^2 + 2$,取 $x_0 = (1, 1)^t$,用共轭梯度法.

    \Answer $\nabla f(x) = (2x_1, 4x_2)^t$

    第一次迭代:$g_1 = (2, 4)^t, d^{(1)} = (-2, -4)^t, x^{(2)} = (\frac{4}{9}, -\frac{1}{9})^t$

    第二次迭代:$g_2 = (\frac{8}{9}, -\frac{4}{9})^t, d^{(2)} = -g_2 + \beta_1d^{(1)} = \frac{20}{81}(-4, 1)^t, x^{(3)} = x^{(2)} + \lambda_2d^{(2)} = (0, 0)^t$

    第三次迭代:$g_3 = (0, 0), \norm{g_3} = 0$,故最优点 $\bar{x} = (0, 0)$.
\end{problem}

\begin{problem}
    证明:如果非零向量 $p_0, \dots, p_l$ 满足 $p_i^tAp_j = 0, \forall i \neq j$,其中 $A$ 为对称正定矩阵,则这些向量线性无关.

    \begin{proof}
        设 $\alpha_0p_0 + \cdots + \alpha_lp_l = 0$

        左乘 $p_i^tA$ 可得 $\alpha_0p_i^tAp_0 + \cdots + \alpha_lp_i^tAp_l = 0$,得到 $\alpha_i = 0$. 故 $p_0, \dots, p_l$ 线性无关.
    \end{proof}
\end{problem}

\end{document}