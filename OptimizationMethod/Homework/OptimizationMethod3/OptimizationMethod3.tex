% \documentclass[12pt]{article}
\documentclass[12pt]{ctexart}
\usepackage[utf8]{inputenc}

\usepackage[english]{babel}
\usepackage[dvips]{epsfig}
\usepackage{amsmath}
\usepackage{amssymb}
\usepackage{amsfonts}
\usepackage{amsthm}
\usepackage{amsbsy}
\usepackage{amsgen}
\usepackage{amscd}
\usepackage{amsopn}
\usepackage{amstext}
\usepackage{amsxtra}
\usepackage{mathrsfs}
\usepackage{enumitem}
\usepackage{graphicx}
\usepackage{verbatim}
\usepackage{epstopdf}
\usepackage{float}
\usepackage[all,cmtip]{xy}
\usepackage{accents}
\usepackage{sseq}
\usepackage{url}
\usepackage{hyperref}
\usepackage{makeidx}
\usepackage{siunitx}
\usepackage{xcolor}

%%%%%%%%% 版面设置 %%%%%%%%%%%%%%%%%%%%%%%%%%%%%%%%%%%%%%
\usepackage{geometry}
\usepackage{titlesec}
\usepackage{fancyhdr}\pagestyle{empty}
\titleformat*{\section}{\large\bfseries}

%
\geometry{
	a4paper,
	total={170mm,240mm},
	left=20mm,
	top=30mm,
}

%Bitte nicht einstellen
\renewcommand{\figurename}{Abbildung}
\renewcommand{\tablename}{Tabelle}
\pagestyle{fancyplain}
\headheight 35pt
\lhead{\name}
\chead{\textbf{\Large \Title}}
\rhead{\due\\\today}
\lfoot{}
\cfoot{}
\rfoot{\small\thepage}
\headsep 1.5em

%%%%%%%%%%%%%%%%%%%%%%%%%%%%%%%%%%%%%%%%%%%%%%%%%%%%%%

\newtheorem{thm}{Theorem}[section]

% 定义解题环境
\theoremstyle{remark}
\newtheorem{remark}[thm]{Remark}
% \newtheorem{remark}[thm]{Remark}
\newtheorem{observation}[thm]{Observation}

\theoremstyle{definition}
\newtheorem{problem}{\text{}}
\newtheorem{Problem}{\text{}}
\newtheorem*{solution}{解}
\newtheorem*{Answer}{Answer}

%%%%%%%%%%%%%%%%%%%%%%%%%%%%%%%%%%%%%%%%%%%%%%%%%%%%%%%%%%%%%%%%%%
\newcommand\name{陈景龙22120307}
\newcommand\due{-}
\newcommand{\emptyline}{\vspace{0.6\baselineskip}}

\newcommand\Title{最优化方法第3次作业}


\begin{document}


\begin{remark}
    Farkas 引理:设 $\boldsymbol{A}$ 为 $m \times n$ 矩阵,$\boldsymbol{c}$ 为 $n$ 维列向量,则 $\boldsymbol{A}\boldsymbol{x}\le \boldsymbol{0},\boldsymbol{c}^T\boldsymbol{x} > 0$ 有解的充要条件是 $\boldsymbol{A}^T\boldsymbol{y} =\boldsymbol{c}, \boldsymbol{y} \ge \boldsymbol{0}$ 无解。
\end{remark}

\begin{remark} 
    Gordan定理:给定矩阵 $\boldsymbol{A} \in \mathbb{R}^{m\times n}$,下列两个系统只有一个有解
    \begin{gather*}
        \boldsymbol{A}x < 0\\
        \boldsymbol{y} \ge 0, \boldsymbol{y} \neq 0, \boldsymbol{A}^T\boldsymbol{y} = 0
    \end{gather*}
\end{remark}

\begin{problem}
    设 $A$ 是 $m\times n$ 矩阵,$B$ 是 $l \times n$ 矩阵,$c\in\mathbb{R}^n$,证明下列两个系统恰有一个有解:
    \begin{itemize}
        \item 系统1:$Ax \le 0, Bx = 0, c^Tx > 0$,对某些 $x\in\mathbb{R}^n$
        \item 系统2:$A^Ty + B^Tz = c, y \ge 0$,对某些 $y \in \mathbb{R}^m$ 和 $z\in\mathbb{R}^l$
    \end{itemize}  
\end{problem}
\begin{solution}
    证系统 1 有解,即
    \[\begin{bmatrix}
        A\\
        B\\
        -B
    \end{bmatrix}x\le 0, c^Tx>0\]
    有解,则根据 Farkas定理,有
    \[\begin{pmatrix}
        A^T & B^T & -B^T
    \end{pmatrix}\begin{bmatrix}
        y\\
        z_1\\
        z_2
    \end{bmatrix}=c,\begin{bmatrix}
        y\\
        z_1\\
        z_2
    \end{bmatrix}\ge 0\]
    无解,即 $A^Ty + B^Tz_1 - B^Tz_2 = c, y\ge 0, z_1\ge 0, z_2\ge 0$无解,即 $A^T + B^Tz = c, y\ge 0$ 无解。

    反之,若 $A^T + B^Tz = c, y\ge 0$ 有解,即 $A^Ty + B^Tz_1 - B^Tz_2 = c, y\ge 0, z_1\ge 0, z_2\ge 0$ 有解,根据 Farkas定理,有 
    \[\begin{bmatrix}
        A\\
        B\\
        -B
    \end{bmatrix}x\le 0, c^Tx>0\]
    无解,即 $Ax \le 0, Bx = 0, c^Tx > 0$ 无解。
\end{solution}

\begin{problem}
    证明:设 $f(x)$ 是凸集 $S$ 上的凸函数,对每一个实数 $c$,则集合 \[S_c=\left\{x|x\in S, f(x) \le c\right\}\] 是一个凸集。
\end{problem}
\begin{solution}
    $\forall x_1, x_2 \in S_c, z = \lambda x_1 + (1 - \lambda) x_2$,有 
    \begin{align*}
        f(z) &= f(\lambda x_1 + (1 - \lambda) x_2) \\
        &\le \lambda f(x_1) + (1 - \lambda)f(x_2)\\
        &\le \lambda c + (1 - \lambda)c \\
        &= c
    \end{align*}

    得到 $f(z) \le c$,所以 $z \in S_c$,则集合 $S_c$ 是一个凸集。 
\end{solution}

\begin{problem}
    用定义验证下列集合是凸集。
    \[S = \left\{(x_1, x_2) | x_1^2 + x_2^2 \le 10\right\}\]
\end{problem}
\begin{solution}
    任取 $x,y\in S$,满足 $x=(x_1, x_2),y=(y_1, y_2),x_1^2 + x_2^2 \le 10, y_1^2 + y_2^2 \le 10$。

    \[z = \lambda x + (1 - \lambda)y = \left(\lambda x_1 + (1 - \lambda) y_1, \lambda x_2 + (1 - \lambda) y_2\right)\]
    \begin{align*}
        &z_1^2 + z_2^2\\
        =& \lambda^2 x_1^2 + (1-\lambda)y_1^2 + 2\lambda x_1(1-\lambda)y_1\\
        &+\lambda^2 x_2^2 + (1-\lambda)y_2^2 + 2\lambda x_2(1-\lambda)y_2\\
        =& \lambda (x_1^2 + x_2^2) + (1-\lambda)^2(y_1^2 + y_2^2) + 2\lambda(1 - \lambda)(x_1y_1 + x_2y_2)\\
        \le & 10\lambda^2 + 10(1-\lambda)^2 + \lambda(1 - \lambda)(x_1^2 + y_1^2 + x_2^2 + y_2^2)\\
        \le & 10\lambda^2 + 10(1-\lambda)^2 + \lambda(1 - \lambda)\cdot 20\\
        =& 10
    \end{align*}

    所以 $z$ 也在集合内,集合 $S$ 是一个凸集。
\end{solution}

\begin{problem}
    证明下列集合 $S$ 是凸集:
    \[S = \left\{x | x = Ay, y \ge 0\right\}\]
    其中 $A$ 是 $n\times m$ 矩阵,$x \in \mathbb{R}^n, y\in\mathbb{R}^m$。
\end{problem}
\begin{solution}
    $\forall x^{(1)}, x^{(2)} \in S$,$\lambda x^{(1)} + (1 - \lambda)x^{(2)}= A \left[\lambda y_1 + (1 - \lambda)y_2\right]$,而 $\lambda y_1 + (1 - \lambda)y_2 \ge 0$,则 $\lambda x^{(1)} + (1 - \lambda)x^{(2)} \in S$,所以集合 $S$ 是凸集。
\end{solution}

\begin{problem}
    设$A$ 是 $m\times n$ 矩阵,$c\in\mathbb{R}^m$,则下列两个系统中恰有一个有解:
    \begin{itemize}
        \item 系统1: $Ax\le 0,x\ge 0,c^Tx>0$,对某些 $x\in\mathbb{R}^n$。
        \item 系统2: $A^Ty\ge c,y\ge 0$,对某些 $y\in\mathbb{R}^m$。
    \end{itemize}
\end{problem}
\begin{solution}
    证系统1有解,即
    \[\begin{bmatrix}
        A\\
        -I
    \end{bmatrix}x\le 0, c^Tx > 0\]
    有解,则根据 Farkas 定理,有
    \[(A^T - I)\begin{bmatrix}
        y\\
        u
    \end{bmatrix} = c, \begin{bmatrix}
        y\\
        u
    \end{bmatrix}\ge 0\]
    无解,即 $A^Ty - u = c, y\ge 0, u\ge 0$ 无解,即 $A^Tt\ge c, y\ge 0$ 无解。

    反之,若 $A^Ty\ge c, y\ge 0$ 有解,即
    \[A^Ty - u = c, y\ge 0, y\ge 0\]
    有解,即
    \[(A^T - I)\begin{bmatrix}
        y\\
        u
    \end{bmatrix} = c, \begin{bmatrix}
        y\\
        u
    \end{bmatrix} \ge 0\]
    有解,根据 Farkas 定理,有
    \[\begin{bmatrix}
        A\\
        -I
    \end{bmatrix}x\le 0, c^Tx>0\]
    无解,即 $Ax\le 0, x\ge 0, c^Tx>0$无解。
\end{solution}

\begin{problem}
    证明下列不等式组无解:
    \[\begin{cases}
        x_1 + 3x_2 < 0 \\
        3x_1 - x_2 < 0 \\
        17x_1 + 11x_2 > 0
    \end{cases}\]
\end{problem}
\begin{solution}
    $A = \begin{bmatrix}
        1 & 3\\
        3 & -1\\
        -17 & -11
    \end{bmatrix}$,即证 $Ax < 0$ 无解。

    根据 Gordan 定理,只需证明 $A^Ty = 0, y \ge 0, y \neq 0$ 有解,对系数矩阵 $A^T$ 做初等行变换:
    \[\begin{bmatrix}
        1 & 3 & -17\\
        3 & -1 & -11
    \end{bmatrix} \longrightarrow \begin{bmatrix}
        1 & 3 & -17\\
        0 & -10 & 40
    \end{bmatrix} \longrightarrow \begin{bmatrix}
        1 & 0 & -5\\
        0 & 1 & -4
    \end{bmatrix}\]
    所以 $A^Ty = 0, y\ge 0, y \neq 0$ 有解,根据 Gordan 定理,原来的不等式组无解。
\end{solution}

\begin{problem}
    判别下列函数是否为凸函数。
    \[f(x_1, x_2) = (x_1 - x_2)^2 + 4x_1x_2 + e^{x_1 + x_2}\]
\end{problem}
\begin{solution}
    \begin{align*}
        \partial f_{x_1} &= 2(x_1 - x_2) + 4x_2 + e^{x_1 + x_2}\\
        \partial f_{x_2} &= -2(x_1 - x_2) + e^{x_1 + x_2}\\
        \partial f_{x_1, x_1} &= 2 + e^{x_1 + x_2}\\
        \partial f_{x_1, x_2} &= 2 + e^{x_1 + x_2}\\
        \partial f_{x_2, x_2} &= 2 + e^{x_1 + x_2}\\
        \nabla^2f(x) &= \begin{bmatrix}
            2 + e^{x_1 + x_2} & 2 + e^{x_1 + x_2}\\
            2 + e^{x_1 + x_2} & 2 + e^{x_1 + x_2}
        \end{bmatrix} = (2 + e^{x_1 + x_2}) \begin{bmatrix}
            1 & 1\\
            1 & 1
        \end{bmatrix}
    \end{align*}

    $\nabla^2f(x)$ 矩阵是一个半正定矩阵,因此 $f(x)$ 是凸函数。

    
\end{solution}

\begin{problem}
    设 $f(x_1, x_2) = 10 - 2(x_2 - x_1^2)^2$,
    \[S = \left\{(x_1, x_2) | -11 \le x_1 \le 1, -1\le x_2 \le 1\right\}\]
    $f(x_1, x_2)$ 是否为 $S$ 上的凸函数?
\end{problem}
\begin{solution}
    \begin{align*}
        \partial f_{x_1} &= 8x_1(x_2 - x_1^2)\\
        \partial f_{x_2} &= -4(x_2 - x_1^2)\\
        \partial f_{x_1, x_1} &= 8(x_2 - 3x_1^2)\\
        \partial f_{x_1, x_2} &= 8x_1\\
        \partial f_{x_2, x_2} &= -4\\
        \nabla^2f(x) &= \begin{bmatrix}
            8(x_2 - 3x_1^2) & 8x_1 \\
            8x_1 & -4
        \end{bmatrix}
    \end{align*}
    $\nabla^2f(x)$ 不是半正定矩阵,因此 $f(x_1, x_2)$ 不是 $S$ 上的凸函数。
\end{solution}

\end{document}