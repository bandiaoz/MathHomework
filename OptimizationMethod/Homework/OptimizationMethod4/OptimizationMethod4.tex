% \documentclass[12pt]{article}
\documentclass[12pt]{ctexart}
\usepackage[utf8]{inputenc}

\usepackage[english]{babel}
\usepackage[dvips]{epsfig}
\usepackage{amsmath}
\usepackage{amssymb}
\usepackage{amsfonts}
\usepackage{amsthm}
\usepackage{amsbsy}
\usepackage{amsgen}
\usepackage{amscd}
\usepackage{amsopn}
\usepackage{amstext}
\usepackage{amsxtra}
\usepackage{mathrsfs}
\usepackage{enumitem}
\usepackage{graphicx}
\usepackage{verbatim}
\usepackage{epstopdf}
\usepackage{float}
\usepackage[all,cmtip]{xy}
\usepackage{accents}
\usepackage{sseq}
\usepackage{url}
\usepackage{hyperref}
\usepackage{makeidx}
\usepackage{siunitx}
\usepackage{xcolor}

%%%%%%%%% 版面设置 %%%%%%%%%%%%%%%%%%%%%%%%%%%%%%%%%%%%%%
\usepackage{geometry}
\usepackage{titlesec}
\usepackage{fancyhdr}\pagestyle{empty}
\titleformat*{\section}{\large\bfseries}

%
\geometry{
	a4paper,
	total={170mm,240mm},
	left=20mm,
	top=30mm,
}

%Bitte nicht einstellen
\renewcommand{\figurename}{Abbildung}
\renewcommand{\tablename}{Tabelle}
\pagestyle{fancyplain}
\headheight 35pt
\lhead{\name}
\chead{\textbf{\Large \Title}}
\rhead{\due\\\today}
\lfoot{}
\cfoot{}
\rfoot{\small\thepage}
\headsep 1.5em

%%%%%%%%%%%%%%%%%%%%%%%%%%%%%%%%%%%%%%%%%%%%%%%%%%%%%%

\newtheorem{thm}{Theorem}[section]

% 定义解题环境
\theoremstyle{remark}
\newtheorem{remark}[thm]{Remark}
% \newtheorem{remark}[thm]{Remark}
\newtheorem{observation}[thm]{Observation}

\theoremstyle{definition}
\newtheorem{problem}{\text{}}
\newtheorem{Problem}{\text{}}
\newtheorem*{solution}{解}
\newtheorem*{Answer}{Answer}

%%%%%%%%%%%%%%%%%%%%%%%%%%%%%%%%%%%%%%%%%%%%%%%%%%%%%%%%%%%%%%%%%%
\newcommand\name{陈景龙22120307}
\newcommand\due{-}
\newcommand{\emptyline}{\vspace{0.6\baselineskip}}

\newcommand\Title{最优化方法第4次作业}


\begin{document}

\begin{problem}
    设 $\boldsymbol{x}^{(0)} = (x_1^{(0)}, x_2^{(0)}, \cdots, x_n^{(0)})^T$ 是 $\boldsymbol{A}\boldsymbol{x} = \boldsymbol{b}$ 的一个解,其中 $\boldsymbol{A} = (\boldsymbol{p}_1, \boldsymbol{p}_2, \cdots, \boldsymbol{p}_n)$ 是 $m\times n$ 矩阵,$\boldsymbol{A}$ 的秩为 $m$。证明 $\boldsymbol{x}^{(0)}$ 是基本解的充要条件为 $\boldsymbol{x}^{(0)}$ 的非零分量 $x_{i_1}^{(0)}, x_{i_2}^{(0)}, \cdots, x_{i_s}^{(0)}$ 对应的列 $\boldsymbol{p}_{i_1}, \boldsymbol{p}_{i_2}, \cdots, \boldsymbol{p}_{i_s}$ 线性无关。
\end{problem}
\begin{solution}
    必要性:设 $\boldsymbol{x}^{(0)} = \begin{bmatrix}
        \boldsymbol{B}^{-1}\boldsymbol{b} \\
        \boldsymbol{0}
    \end{bmatrix}$ 是基本解,$\boldsymbol{B} = \begin{bmatrix}
        \boldsymbol{p}_{B_1} & \boldsymbol{p}_{B_2} & \cdots & \boldsymbol{p}_{B_m} 
    \end{bmatrix}$,则 $x^{(0)}$ 非零分量对应的列 $\left\{\boldsymbol{p}_{i_1}, \boldsymbol{p}_{i_2}, \cdots, \boldsymbol{p}_{i_s}\right\} \subset \left\{\boldsymbol{p}_{B_1}, \boldsymbol{p}_{B_2}, \cdots  \boldsymbol{p}_{B_m} \right\}$。由于 $\boldsymbol{p}_{B_1}, \boldsymbol{p}_{B_2}, \cdots  \boldsymbol{p}_{B_m}$ 线性无关,因此 $\boldsymbol{p}_{i_1}, \boldsymbol{p}_{i_2}, \cdots, \boldsymbol{p}_{i_s}$ 线性无关。

    充分性:设 $\boldsymbol{x}^{(0)}$ 的非零分量对应的列 $\boldsymbol{p}_{i_1}, \boldsymbol{p}_{i_2}, \cdots, \boldsymbol{p}_{i_s}$ 线性无关,矩阵 $\boldsymbol{A}$ 的秩为 $m$,因此,$s \le m$,可以将 $\boldsymbol{p}_{i_1}, \boldsymbol{p}_{i_2}, \cdots, \boldsymbol{p}_{i_s}$ 扩充成一组基。于是 $\boldsymbol{x}^{(0)}$ 可记作:$\begin{bmatrix}
        \boldsymbol{x}_B^{(0)} \\
        \boldsymbol{x}_N^{(0)}
    \end{bmatrix} = \begin{bmatrix}
        \boldsymbol{B}^{-1}\boldsymbol{b} \\
        \boldsymbol{0}
    \end{bmatrix}$,$\boldsymbol{x}^{(0)}$ 是基本解。
\end{solution}

\begin{problem}
    设 $S = \left\{\boldsymbol{x}|\boldsymbol{A}\boldsymbol{x} \ge \boldsymbol{b}\right\}$,其中 $\boldsymbol{A}$ 是 $m\times n$ 矩阵,$m > n$,$\boldsymbol{A}$ 的秩为 $n$。证明 $\boldsymbol{x}^{(0)}$ 是 $S$ 的极点的充要条件是 $\boldsymbol{A}$ 和 $\boldsymbol{b}$ 可作如下分解:
    \[\boldsymbol{A} = \begin{bmatrix}
        \boldsymbol{A}_1 \\
        \boldsymbol{A}_2
    \end{bmatrix}, \boldsymbol{b} = \begin{bmatrix}
        \boldsymbol{b}_1 \\
        \boldsymbol{b}_2
    \end{bmatrix}\]
    其中 $\boldsymbol{A}_1$ 有 $n$ 个行,且 $\boldsymbol{A}_1$ 的秩为 $n$,$\boldsymbol{b}_1$ 是 $n$ 维列向量,使得 $\boldsymbol{A}_1\boldsymbol{x}^{(0)} = \boldsymbol{b}_1, \boldsymbol{A}_2\boldsymbol{x}^{0} \ge \boldsymbol{b}_2$。
\end{problem}
\begin{solution}
    必要性:设 $x^{(0)}$ 是 $S$ 的极点,用反证法,设 $A, b$ 在点 $x^{(0)}$ 分解如下:
    \[A = \begin{bmatrix}
        A_1 \\ 
        A_2
    \end{bmatrix}, b = \begin{bmatrix}
        b_1 \\ 
        b_2
    \end{bmatrix}, A_1x^{(0)} = b_1, A_2x^{(0)} > b_2\]
    $A_1$ 的秩 $R(A_1) < n$。$A_1x = b_1$ 的同解线性方程组记作 \[\hat{A_1}x = \hat{b_1}\]
    $\hat{A_1}$ 是行满秩矩阵,$R(\hat{A_1}) = R(A_1) < n$。不妨假设 $\hat{A_1}$ 的前 $R(\hat{A_1})$ 个列线性无关,记作 $\hat{A_1} = \begin{bmatrix}
        B & N
    \end{bmatrix}$,其中 $B$ 是可逆矩阵,相应地记 \[x = \begin{bmatrix}
        x_B \\
        x_N
    \end{bmatrix},x_B = B^{-1}\hat{b_1} - B^{-1}Nx_N\]
    $A_1x = b_1$ 的解为 
    \begin{equation}
        x = \begin{bmatrix}
            x_B\\
            x_N
        \end{bmatrix} = \begin{bmatrix}
            B^{-1}\hat{b_1} - B^{-1}Nx_N \\
            x_N
        \end{bmatrix}
    \end{equation}
    其中,$x_N$ 是自由未知量,是 $n - R(A_1)$ 维向量,$S$ 的极点
    \begin{equation}
        x^{(0)} = \begin{bmatrix}
            x_B^{(0)}\\
            x_N^{(0)}
        \end{bmatrix} = \begin{bmatrix}
            B^{-1}\hat{b_1} - B^{-1}Nx_N^{(0)} \\
            x_N^{(0)}
        \end{bmatrix}
    \end{equation}
    由于 $A_2x^{(0)} > b_2$,则存在 $x_N^{(0)}$ 的 $\delta$ 邻域 $N_s(x_N^{(0)})$,使得当 $x_N\in N_s(x_N^{(0)})$ 时,解(1)同时满足 $A_1x = b_1$ 和 $A_2x \ge b_2$,在过 $x_N^{(0)}$ 的直线上取不同点 $x_N^{(1)}, x_N^{(2)} \in N_s(x_N^{(0)})$,使 $\lambda x_N^{(1)} + (1 - \lambda)x_N^{(2)} = x_N^{(0)}, \lambda \in (0, 1)$,代入(2)式得到
    \begin{align*}
        \boldsymbol{x}^{(0)} &=\left[\begin{array}{c}
        \boldsymbol{B}^{-1} \hat{\boldsymbol{b}}_{1}-\boldsymbol{B}^{-1} \boldsymbol{N}\left(\lambda \boldsymbol{x}_{N}^{(1)}+(1-\lambda) \boldsymbol{x}_{N}^{(2)}\right) \\
        \lambda \boldsymbol{x}_{N}^{(1)}+(1-\lambda) \boldsymbol{x}_{N}^{(2)}
        \end{array}\right] \\
        &=\lambda\left[\begin{array}{c}
        \boldsymbol{B}^{-1} \hat{\boldsymbol{b}}_{1}-\boldsymbol{B}^{-1} \boldsymbol{N} \boldsymbol{x}_{N}^{(1)} \\
        \boldsymbol{x}_{N}^{(1)}
        \end{array}\right]+(1-\lambda)\left[\begin{array}{c}
        \mathbf{B}^{-1} \hat{\boldsymbol{b}}_{1}-\boldsymbol{B}^{-1} \boldsymbol{N} \boldsymbol{x}_{N}^{(2)} \\
        \boldsymbol{x}_{N}^{(2)}
        \end{array}\right]
    \end{align*}
    这样,可将 $x^{(0)}$ 表示成集合 $S$ 中两个不同点的凸组合,矛盾。

    再证充分性。设在点 $x^{(0)}, A, b$ 可作如下分解(其中 $A_1$ 是 $n$ 阶方阵)
    \[A = \begin{bmatrix}
        A_1 \\
        A_2
    \end{bmatrix}, b = \begin{bmatrix}
        b_1 \\ 
        b_2
    \end{bmatrix}, A_1x^{(0)} = b_1, A_2x^{(0)} \ge b_2, R(A_1) = n\]
    又设存在 $x_{(1)}, x_{(2)} \in S$ 使得
    \begin{equation}
        x^{(0)} = \lambda x^{(1)} + (1 - \lambda)x^{(2)}, \lambda \in (0, 1)
    \end{equation}
    用可逆矩阵 $A_1$ 乘(3)式两端,得
    \begin{equation}
        A_1x^{(0)} = \lambda A_1x^{(1)} + (1 - \lambda)A_1x^{(2)}
    \end{equation}
    由于 $A_1x^{(0)} = b_1, A_1x^{(1)} \ge b_1, A_1x^{(2)}\ge b_1, \lambda > 0, (1 - \lambda) > 0$,代入(4)式,得
    \[b_1 = A_1x^{(0)} = \lambda A_1x^{(1)} + (1 - \lambda)A_1x^{(2)} \ge \lambda b_1 + (1 - \lambda)b_1 = b_1\]
    因此
    \[\lambda A_1x^{(1)} + (1 - \lambda)A_1x^{(2)} = \lambda b_1 + (1 - \lambda)b_1\]
    整理得
    \[\lambda (A_1x^{(1)} - b_1) + (1 - \lambda)(A_1x^{(2)} - b_1) = 0\]
    由于 $\lambda > 0, 1 - \lambda > 0, A_1x^{(1)} - b_1 \ge 0, A_1x^{(2)} - b_1 \ge 0$,因此 $A_1x^{(0)} = A_1x^{(1)} = A_1x^{(2)} = b_1$,左乘 $A_1^{-1}$,得 $x_{(0)} = x_{(1)} = x_{(2)}$,因此 $x^{(0)}$ 是极点。
\end{solution}

\begin{problem}
    设函数 $f:\mathbb{R}^n \to \mathbb{R}$,矩阵 $\boldsymbol{A} \in \mathbb{R}^{n \times n}$ 对称,向量 $\boldsymbol{b} \in \mathbb{R}^n$。
    \begin{enumerate}
        \item 写出函数 $f$ 是凸函数的定义,并列出至少两个判定函数 $f$ 是凸函数的充要条件;
        \item 设 $f(x_1, x_2) = (x_2 - x_1^2)^2, S = \left\{(x_1, x_2)|-1 < x_1 < 1, -1 < x_2 < 1\right\}$,判断函数 $f(x_1, x_2)$ 是否为 $S$ 上的凸函数?说明理由;
        \item 证明 $f(x) = \frac{1}{2}x^TAx + b^Tx$ 为严格凸函数的充要条件是其 $Hessian$ 阵 $A$ 正定。
    \end{enumerate}
\end{problem}
\begin{solution}
    \begin{enumerate}
        \item 函数 $f$ 的定义域 $S$ 是非空凸集,对于 $\forall x_1, x_2 \in S, 0 \le \lambda \le 1$,都有 \[f(\lambda x_1 + (1 - \lambda)x_2) \le \lambda f(x_1) + (1 - \lambda)f(x_2)\] 
            充要条件:\begin{enumerate} 
                \item 函数 $f$ 的定义域 $S$ 是非空开凸集,函数 $f(X)$ 可微,$\forall x_1, x_2 \in S,f(x_2) \ge f(x_1) + \nabla f(x_1)^T(x_2 - x_1)$。
                \item 函数 $f$ 的定义域 $S$ 是非空开凸集,函数 $f(X)$ 二次可微 $\forall x \in S$, $f(x)$ 在 $x$ 处的 Hessian 矩阵 $\nabla^2f(x)$ 半正定。
            \end{enumerate}
        \item $\forall \boldsymbol{x}, \boldsymbol{y} \in S, 0\le \lambda \le 1$
            \begin{gather*}
                \boldsymbol{z} = \lambda \boldsymbol{x} + (1 - \lambda) \boldsymbol{y} \\
                z_1 = \lambda x_1 + (1 - \lambda) y_1\\
                z_2 = \lambda x_2 + (1 - \lambda) y_2\\
                -1 = \lambda \cdot (-1) + (1 - \lambda) \cdot (-1) \le z_1 \le \lambda \cdot 1 + (1 - \lambda) \cdot 1 = 1 \\
                -1 = \lambda \cdot (-1) + (1 - \lambda) \cdot (-1) \le z_2 \le \lambda \cdot 1 + (1 - \lambda) \cdot 1 = 1
            \end{gather*}
            $\boldsymbol{z} \in S$,所以集合 $S$ 是一个凸集。

            \begin{align*}
                \partial_{x_1x_1}f &= 4(3x_1^2 - x_2)\\
                \partial_{x_1x_2}f &= -4x_1 \\
                \partial_{x_2x_2}f &= 2\\
                \nabla^2 f &= \begin{pmatrix}
                    4(3x_1^2 - x_2) & -4x_1 \\
                    -4x_1 & 2
                \end{pmatrix} 
            \end{align*}
            
            $\nabla^2 f$ 矩阵不是半正定矩阵,所以函数 $f(x_1, x_2)$ 不是 $S$ 上的凸函数。
        
        \item 必要性:对 $\forall \boldsymbol{x} \in E^n,\exists \delta > 0$ 使当 $\lambda \in (0, \delta)$,有 $\bar{\boldsymbol{x}} + \lambda \boldsymbol{x} \in S$。
        $\Rightarrow f(\bar{\boldsymbol{x}} + \lambda\boldsymbol{x}) > f(\bar{\boldsymbol{x}}) + \lambda \nabla f(\bar{\boldsymbol{x}})^T\boldsymbol{x}$

        \[f(\bar{x} + \lambda x) = f(\bar{x}) + \lambda\nabla f(\bar{x})^T x + \frac{1}{2}\lambda^2x^T\nabla^2f(\bar{x})x + o(\|\lambda x\|^2)\]
        可得\[\frac{1}{2}\lambda^2x^T\nabla^2f(\bar{x})x + o(\|\lambda x\|^2) > 0\]
        两边除以 $\lambda^2$,令 $\lambda\to 0$,得 $x^T\nabla^2f(\bar{x})x > 0$,所以 $\nabla^2f(x)$ 正定。

        充分性:$\nabla^2f(x)$ 在任意点 $x\in S$ 正定,对 $\forall x, \bar{x} \in S$,\[f(x) = f(\bar{x}) + \nabla f(\bar{x})^T(x - \bar{x}) + \frac{1}{2}(x - \bar{x})\nabla^2 f(\xi)(x - \bar{x})\]
        其中 $\xi = \lambda\bar{x} + (1 - \lambda)x, \lambda \in (0, 1)$。因为 $S$ 是凸集,所以 $\xi \in S$,又 $\nabla^2f(x)$ 正定,得 $\frac{1}{2}(x - \bar{x})^T\nabla^2 f(\xi)(x - \bar{x}) > 0\Rightarrow f(x) > f(\bar{x}) + \nabla f(\bar{x})^T(x - \bar{x})$。所以函数 $f(x)$ 严格凸。
    \end{enumerate}
\end{solution}

\end{document}