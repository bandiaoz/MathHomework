\section{Introduction}
\begin{note}
    给定二次型 $f(X) = X^T AX$,若对 $\forall X \neq 0$,都有 $f(X) = X^T AX > 0$ 成立,则称 $f(X)$ 为正定二次型,$A$ 为正定矩阵.

    对于 $n$ 阶实对称矩阵 $A$,下列命题等价:
    \begin{itemize}
        \item $A^T AX$ 是正定二次型(或 $A$ 是正定矩阵)
        \item $A$ 的 $n$ 个顺序主子式都大于 0
        \item $A$ 的 $n$ 个特征值都大于 0
        \item 存在可逆矩阵 $P$,使得 $A = P^T P$
    \end{itemize}
\end{note}

\begin{note}
    给定二次型 $f(X) = X^T AX$,若对 $\forall X \neq 0$,都有 $f(X) = X^T AX \ge 0$ 成立,则称 $f(X)$ 为半正定二次型,$A$ 为半正定矩阵.

    对于 $n$ 阶实对称矩阵 $A$,下列命题等价:
    \begin{itemize}
        \item $A^T AX$ 是半正定二次型(或 $A$ 是半正定矩阵)
        \item $A$ 的所有主子式都大于等于 0,而且至少有一个等于 0
        \item $A$ 的 $n$ 个特征值都大于等于 0,而且至少有一个等于 0
    \end{itemize}
\end{note}

\begin{note}
    设 $S \subseteq E^n$,若对 $\forall x, y \in S, \forall \lambda \in [0, 1]$,都有 $\lambda x + (1 - \lambda)y \in S$,则称 $S$ 为凸集.
    \begin{itemize}
        \item $x_1, \dots, x_k \in S, \lambda_1 + \cdots + \lambda_k = 1$,称 $\sum_{i = 1}^k \lambda_ix_i$ 为 $x_1, \dots, x_k$ 的凸组合
        \item $S$ 是非空凸集,$x\in S$,若由 $x = \lambda x_1 + (1 - \lambda)x_2$,其中 $\lambda \in (0, 1), x_1, x_2 \in S$,必推出 $x = x_1 = x_2$,则称 $x$ 是 $S$ 的极点.
        \item $S$ 是 $E^n$ 中的闭凸集,$d \in E^n, d \neq 0$,如果对 $\forall x \in S$,有 $\left\{x + \lambda d | \lambda > 0\right\} \subset S$,则称向量 $d$ 为 $S$ 的方向.若 $S$ 的方向 $d$ 不能表示为集合的两个不同方向的\textbf{正线性组合},则称 $d$ 为 $S$ 的极方向.
    \end{itemize}
\end{note}

\begin{note}
    设 $S = \left\{x \mid Ax = b, x \ge 0\right\}$ 为非空多面集,则有
    \begin{itemize}
        \item 极点集非空,且存在有限个极点
        \item 极方向集合为空集的充要条件是 $S$ 有界;若 $S$ 无界,则存在有限个极方向
        \item (多面集表示定理)$x \in S$ 的充要条件是 \begin{gather*}
            x = \sum_{j = 1}^k \lambda_j x^{(j)} + \sum_{j = 1}^l \mu_j d^{(j)}\\
            \lambda_j \ge 0, \forall j = 1, \dots, k\\
            \sum_{j = 1}^k \lambda_j = 1\\
            \mu_j \ge 0, \forall j = 1, \dots, l
        \end{gather*}
    \end{itemize}
\end{note}

\begin{note}
    凸集分离定理:
    \begin{itemize}
        \item 设 $S$ 为 $E^n$ 的闭凸集,$y\notin S$,则存在唯一的 $\overline{x} \in S$,使得\[ \|y - \overline{x}\| = \sideset{}{}{\operatorname{inf}}_{x \in S}\|y - x\| > 0 \]
        $\overline{x}$ 是这一最小距离点 $\Leftrightarrow (y - \overline{x})^T(\overline{x} - x) \ge 0, \forall x \in S$.

        \item 设 $S$ 是 $E^n$ 的非空闭凸集,$y\notin S$,则存在非零向量 $p$ 以及数 $\varepsilon > 0$,使得对 $\forall x \in S$,有 $p^T y \ge \varepsilon + p^T x$
        
        \item 设 $S$ 是 $E^n$ 的非空凸集,$y\in \partial S$,则存在非零向量 $p$,使得对 $\forall x \in clS$($S$ 的闭包,由 $S$ 的内点和边界点组成),有 $p^Ty \ge p^Tx$.
        
        \item 设 $S_1$ 和 $S_2$ 是 $E^n$ 的两个非空凸集,$S_1 \cap S_2 = \varnothing$,则存在非零向量,使得\[p^Ty \ge p^T x \quad \text{其中}\forall y \in S_1, \forall x \in S_2\]
    \end{itemize}
\end{note}

\begin{note}
    两个系统恰有一个有解:
    \begin{itemize}
        \item Farkas 引理:设 $A$ 为 $m\times n$ 矩阵,$c$ 为 $n$ 维列向量,则 $Ax \le 0, c^Tx > 0$ 有解的充分条件是 $A^Ty = c, y\ge 0$ 无解.
        \item Gordan定理:给定矩阵 $\boldsymbol{A} \in \mathbb{R}^{m\times n}$,下列两个系统只有一个有解
        \begin{gather*}
            \boldsymbol{A}x < 0\\
            \boldsymbol{y} \ge 0, \boldsymbol{y} \neq 0, \boldsymbol{A}^T\boldsymbol{y} = 0
        \end{gather*}
    \end{itemize}
\end{note}

\begin{note}
    设 $S$ 是 $E^n$ 中的非空凸集,$f(x)$ 是定义在 $S$ 上的实函数,如果对于每一对 $x_1, x_2 \in S$ 及每一个 $\lambda, 0\le \lambda \le 1$ 都有 \[f(\lambda x_1 + (1 - \lambda)x_2) \le \lambda f(x_1) + (1 - \lambda)f(x_2)\]
    则称函数 $f(x)$ 为 $S$ 上的凸函数.

    凸函数的根本重要性:设 $S$ 是 $E^n$ 中的非空凸集,$f$ 是定义在 $S$ 上的凸函数,则 $f$ 在 $S$ 上的局部极小点事整体极小点,且极小点的集合是凸集.
\end{note}

\begin{note}
    $f(x)$ 是凸集 $S$ 上的凸函数,对每一个实数 $c$,则集合 $S_c = \left\{x | x \in S, f(x) \le c\right\}$ 是凸集.
\end{note}

\begin{note}
    凸函数的判别:
    \begin{itemize}
        \item (一阶充要条件)设 $S$ 是 $E^n$ 中的非空开凸集,$f(x)$ 是定义在 $S$ 上的\textbf{可微}函数,则 $f(x)$ 为凸函数的充要条件是对任意两点 $x_1, x_2 \in S$,有 \[f(x_2) \ge f(x_1) + \nabla f(x_1)^T(x_2 - x_1)\]
        $f(x)$ 为严格凸函数的充要条件是对任意互不相同两点 $x_1, x_2 \in S$,有 \[f(x_2) > f(x_1) + \nabla f(x_1)^T(x_2 - x_1)\]

        几何意义:$f(x)$ 是凸函数当且仅当任意点处的切线增量不超过函数的增量.
        \item (二阶充要条件)设 $S$ 是 $E^n$ 中的非空开凸集,$f(x)$ 是定义在 $S$ 上的\textbf{二次可微}函数,则 $f(x)$ 为凸函数的充要条件是对任意 $x \in S$,$f(x)$ 在 $x$ 处的Hessian矩阵 $\nabla ^2f(x)$ 是半正定的.
        
        $f(x)$ 为严格凸函数的充要条件是对任意 $x \in S$,$f(x)$ 在 $x$ 处的Hessian矩阵 $\nabla ^2f(x)$ 是正定的.
    \end{itemize}
\end{note}

\begin{note}
    设 $f(x)$ 是定义在凸集 $S$ 上的可微凸函数,若 $\exists x^* \in S$,使对 $\forall x \in S$,都有 \[\nabla f(x^*)^T(x - x^*) \ge 0\]
    则 $x^*$ 是 $f(x)$ 在凸集 $S$ 上的全局极小点.
\end{note}

\begin{note}
    凸规划:求凸函数在凸集上的极小点.
    \begin{align*}
        \min \quad &f(x)\\ 
        s.t. \quad &g_{i}(x) \le 0, i=1, \cdots, m \\  
        &h_{j}(x)=0, j=1, \cdots, l 
    \end{align*}
    若 $f(x)$ 是凸函数,$g_i(x)$ 是凸函数,$h_j(x)$ 是线性函数,则原问题为凸规划.
    \begin{itemize} 
        \item 凸规划的局部极小点就是整体极小点
        \item 且极小点的集合为凸集
    \end{itemize}
\end{note}
