\section{罚函数法}
\textbf{外点罚函数法、内点罚函数法}、乘子罚函数法。

\subsection{外点罚函数法}
\begin{note}
    对于有约束优化问题 
    \[
        \begin{cases}
            \min \quad &f(x)\\
            \subject \quad &g_i(x) \ge 0, i = 1, \dots, m\\
            &h_j(x) = 0, j = 1, \dots, l
        \end{cases}
    \]
    \begin{itemize}
        \item 通过引入罚项转换为无约束优化问题 
        \begin{align*}
            \min F(x, \sigma) &= f(x) + \sigma p(x)\\
            &= f(x) + \sigma \left(\sum_{i = 1}^m \varphi(g_i(x)) + \sum_{j = 1}^l \psi(h_j(x))\right)
        \end{align*}
        \item 其中 $\sigma$ 是很大的正数,$\varphi(y), \psi(y)$ 是连续函数。
        \item 一般取 $\varphi(g_i(x)) = \left(\max\left\{0, -g_i(x)\right\}\right)^\alpha$,$\psi(h_j(x)) = |h_j(x)|^\beta$,通常 $\alpha = \beta = 2$。
    \end{itemize}
\end{note}

\begin{note}
    外点罚函数法步骤:\begin{enumerate}
        \item 给定初始点 $x^{(0)}$,初始罚因子 $\sigma_1 > 0(\sigma_1 = 1)$,放大系数 $c > 1$,允许误差 $\varepsilon > 0$,置 $k = 1$.
        \item 以 $x^{(k - 1)}$ 为初始点,求解无约束问题 $\min f(x) + \sigma_k p(x)$ 设其极小点为 $x^{(k)}$.
        \item 若 $\sigma_kp(x^{(k)}) < \varepsilon$,则停止计算,得到点 $x^{(k)}$;否则令 $\sigma_{k + 1} = c\sigma_k$,置 $k := k + 1$,返回步骤 2.
    \end{enumerate}
\end{note}

\begin{theorem}
    对于由外点法所产生的序列 $\left\{x^{(k)}\right\}$,总有
    \begin{enumerate}
        \item $F\left(x^{(k + 1)}, \sigma_{k + 1}\right) \ge F\left(x^{(k)}, \sigma_k\right)$
        \item $p\left(x^{(k + 1)}\right) \le p\left(x^{(k)}\right)$
        \item $f\left(x^{(k + 1)}\right) \ge f\left(x^{(k)}\right)$
    \end{enumerate}
\end{theorem}

\begin{theorem}
    设 $x^*$ 是问题 (A) 的一个最优解,则对 $\forall k$,有$f(x^*) = F(x^*, \sigma^*) \ge F(x^{(k)}, \sigma_k) \ge f(x^{(k)})$。
\end{theorem}

\begin{note}
    外点罚函数法的重要特点:函数 $F(x, \sigma)$ 是在整个空间 $E^n$ 内进行优化,初始点可任意选择,且外点法也可用于非凸规划的最优化。

    缺点\begin{enumerate}
        \item 惩罚项 $\sigma p(X)$ 的二阶偏导数一般不存在;
        \item 外点法的中间结果不是可行解,不能作为近似解;
        \item 当点 $x^{(k)}$ 接近最优解时,罚因子 $\sigma_k$ 很大。可能使罚函数性质变坏,使搜索产生极大困难。
    \end{enumerate}
\end{note}

\begin{example}
    使用外点罚函数法求解优化问题
    \[
        \begin{cases}
            \min \quad &x_1 + x_2\\
            \subject \quad &x_1 - x_2^2 = 0    
        \end{cases}
    \]

    \answer 
    定义罚函数
    \[
        F(x, \sigma) = x_1 + x_2 + \sigma(x_1 - x_2^2)^2
    \]
    求偏导 
    \begin{align*}
        \partial_{x_1}F &= 1 + 2\sigma(x_1 - x_2^2)\\
        \partial_{x_2}F &= 1 - 4\sigma x_2(x_1 - x_2^2)
    \end{align*}
    令 $\partial_{x}F = 0$,解得
    \[
        \bar{x}_\sigma = \left(\frac{1}{4} - \frac{1}{\sigma}, -\frac{1}{2}\right)^t \to \left(\frac{1}{4}, -\frac{1}{2}\right)(\sigma \to \infty)
    \]
\end{example}

\begin{example}
    使用外点罚函数法求解优化问题
    \[
        \begin{cases}
            \min \quad &f(x) = x_1 + x_2\\
            \subject \quad &g_1(x) = -x_1^2 + x_2 \ge 0\\
            &g_2(x) = x_1 \ge 0
        \end{cases}
    \]

    \answer 定义外罚函数
    \[
        F(x, \sigma) = x_1 + x_2 + \sigma\left(\max(0, x_1^2 - x_2)\right)^2 + \sigma\left(\max(0, -x_1)\right)^2
    \]
    求偏导
    \begin{align*}
        \partial_{x_1}F &= 1 + 4\sigma x_1\max(0, x_1^2 - x_2) - 2\sigma\max(0, -x_1)\\
        \partial_{x_2}F &= 1 - 2\sigma x_2\max(0, x_1^2 - x_2)
    \end{align*}
    令 $\partial_{x}F = 0$,解得
    \[
        \bar{x}_\sigma = \left(-\frac{1}{2 + 2\sigma}, \frac{1}{(2 + 2\sigma)^2} - \frac{1}{2\sigma}\right)^t \to (0, 0)^t(\sigma \to \infty)
    \]
\end{example}

\begin{example}
    考虑下列问题:
    \[
        \begin{cases}
            \min &x_1x_2\\
            \subject &g(x) = -2x_1 + x_2 + 3 \ge 0.
        \end{cases}
    \]
    \begin{enumerate}
        \item 用二阶最优性条件证明点 $\bar{x} = \left(\frac{3}{4}, -\frac{3}{2}\right)$ 是局部最优解,并说明它是否为全局最优解。
        \item 定义障碍函数 
        \[
            G(x, r) = x_1x_2 - r\ln g(x)\text{,}
        \]
        试用内点法求解,并说明内点法产生的序列趋向点 $\bar{x}$.
    \end{enumerate}

    \answer \text{}
    \begin{enumerate}
        \item $L(x, \lambda) = x_1x_2 + \lambda(2x_1 - x_2 - 3)$, 由 KKT 条件
        \[
            \begin{cases}
                \lambda(2x_1 - x_2 - 3) &= 0\\
                \lambda &\ge 0\\
                \partial_{x_1}L = x_2 + 2\lambda &= 0\\
                \partial_{x_2}L = x_1 - \lambda &= 0\\
            \end{cases}
        \]
        解得 
        \[
            x = \left(\frac{3}{4}, -\frac{3}{2}\right)^t, \quad \lambda = \dfrac{3}{4}.
        \]
        由 \[\lambda > 0, \nabla g(x)^td = 0\]
        可得 
        \[d = (d_1, d_2)^t, \quad  d_1 - 2d_2 = 0,\ d \neq 0.\] 
        此时 
        \begin{align*}
            d^t\nabla^2_xL(x, \lambda)d &= \begin{pmatrix}
                2d_2 & d_2
            \end{pmatrix} \begin{pmatrix}
                0 & 1 \\ 1 & 0
            \end{pmatrix} \begin{pmatrix}
                2d_2 \\ d_2
            \end{pmatrix}\\
            &=4d_2^2\\
            &> 0
        \end{align*} 
        故矩阵 $\nabla^2L_x(x, \lambda)$ 在子空间 $G$ 上正定,故是极小值点。

        只有唯一的 KKT 点,故也是全局最优解。
        \item 令 
        \[
            \begin{cases}
                \partial_{x_1}G = x_2 + \dfrac{2r}{-2x_1 + x_2 + 3} = 0\\
                \partial_{x_2}G = x_1 - \dfrac{r}{-2x_1 + x_2 + 3} = 0\\
            \end{cases}.
        \] 
        可得 
        \[
            x = \frac{1}{8}(3 \mp \sqrt{9 - 16r}, -6 \pm 2\sqrt{9 - 16r})^t.
        \] 
        需在 $r > 0$ 时满足约束条件,故 
        \[
            x = \frac{1}{8}(3 + \sqrt{9 - 16r}, -6 - 2\sqrt{9 - 16r})^t.
        \] 
        令 $r \to 0$, 可得 $x = \left(\dfrac{3}{4}, -\dfrac{3}{2}\right)^t$ 是全局最优解。
    \end{enumerate}
\end{example}

\subsection{内点罚函数法}
基本思想:迭代总是从内点出发,并保持在可行域内部进行搜索.

\begin{note}
    障碍函数:\[G(x, r) = f(X) + rB(x)\]
    其中 $r$ 是很小的正数,$B(x)$ 定义在可行域内部,它满足两个条件:\begin{enumerate}
        \item $B(x)$ 是连续函数
        \item 当点 $x$ 趋向可行域边界时,$B(x) \to +\infty$
    \end{enumerate}
    两种最重要的形式:\[B(x) = \sum_{i = 1}^m \frac{1}{g_i(x)} \quad B(x) = -\sum_{i = 1}^m\ln g_i(x)\]
\end{note}

\begin{note}
    算法步骤:\begin{enumerate}
        \item 给定初始点 $x^{(0)} \in \operatorname{int} S$,允许误差 $\varepsilon > 0$,初始参数 $r_1$,缩小系数 $\beta\in (0, 1)$,置 $k=1$.
        \item 以 $x^{(k - 1)}$ 为初始点,求解下列问题\[\begin{cases}
            \min \quad &f(x) + r_kB(x)\\
            \subject \quad &x \in \operatorname{int} S
        \end{cases}\] 设其极小点为 $x^{(k)}$.
        \item 若 $r_kB(x^{(k)}) < \varepsilon$,则停止计算,得到点 $x^{(k)}$;否则,令 $r_{k + 1} = \beta r_k$,置 $k := k + 1$,返回 2.
    \end{enumerate}
\end{note}

\begin{theorem}
    对于由内点法所产生的序列 $\{x^{(k)}\}$,总有\begin{enumerate}
        \item $G(x^{(k + 1)}, r_{k + 1}) \le G(x^{(k)}, r_k)$
        \item $B(x^{(k + 1)}) \ge B(x^{(k)})$
        \item $f(x^{(k + 1)}) \le f(x^{(k)})$
    \end{enumerate}
\end{theorem}

\begin{theorem}
    设 $\{x^{(l)}\}$ 是由内点法产生的一个序列,则 $\{x^{(k)}\}$ 的任何收敛子序列的极限都是原问题的最优解。
\end{theorem}

\begin{note}
    求初始内点的迭代步骤:\begin{enumerate}
        \item 任取 $x^{(0)} \in E^n, r_0 > 0$(如取 $r_0 = 1$),置 $k := 0$.
        \item 令 $S_k = \left\{i \mid g_i(x^{(k)}) \le 0, 1 \le i \le m\right\}, T_k = \left\{i \mid g_i(x^{k}) > 0, 1 \le i \le m\right\}$
        \item 若 $S_k = \varnothing$,停止计算;否则,转 4.
        \item 构造函数 \[\widetilde{P}\left(x, r_{k}\right)=-\sum_{i \in S_{k}} g_{i}(x)+r_{k} \sum_{i \in T_{k}} \frac{1}{g_{i}(x)},\left(r_{k}>0\right)\]
        记 $\widetilde{R}_{k}=\left\{x \mid g_{i}(x)>0 ,i \in T_{k}\right\}$
        \item 以 $x^{(k)}$ 为初始点,在 $\tilde{R_k}$ 域内,求障碍函数 $\tilde{P}(x, r_k)$ 的极小点:\[\begin{cases}
            \min &\tilde{P}(x, r_k)\\
            \subject & x \in \tilde{R_k}
        \end{cases}\] 得 $x^{(k + 1)}$,转 6
        \item 令 $ 0 < r_{k + 1} < r_k$ (如取 $r_{k + 1} = \frac{1}{10}r_k$),置$k:=k+1$,转 2.
    \end{enumerate}
\end{note}

\begin{example}
    考虑约束优化问题:
    \[
        \begin{cases}
            \min \quad &x_1x_2\\
            \subject \quad &2x_1 - x_2 - 3 = 0
        \end{cases}
    \]
    \begin{enumerate}
        \item 给定 $\bar{x} = \left(\frac{3}{4}, -\frac{3}{2}\right)^t$, 利用约束优化问题局部解的一阶必要条件合二阶充分条件判断 $\bar{x}$ 是否是局部最优解?
        \item 定义外罚函数为
        \[
            G(x, c) = x_1x_2 + \dfrac{c}{2}\left(2x_1 - x_2 - 3\right)^2
        \]
        试用外罚函数法求解,并说明产生的序列趋向点。
    \end{enumerate}

    \answer \text{}
    \begin{enumerate}
        \item Lagrange 乘子函数为 $L(x, \mu) = x_1x_2 + \mu(2x_1 - x_2 - 3)$.
        
        由 KKT 条件可得 
        \[
            \begin{cases}
                2x_1 - x_2 - 3 &= 0\\
                \partial_{x_1}L = x_2 + 2\mu &= 0\\
                \partial_{x_2}L = x_1 - \mu &= 0
            \end{cases}
        \]
        解得
        \[
            x^* = \left(\dfrac{3}{4}, -\dfrac{3}{2}\right).
        \]
        由二阶最优性条件可得 
        \[
            \nabla h(x)^td = 0
        \]
        故
        \[
            d = \left(d_1, d_2\right) \neq 0,\quad 2d_1 - d_2 = 0
        \]
        则 Lagrange 乘子函数的 Hessian 矩阵在空间 $G$ 上 
        \begin{align*}
            d^t\nabla^2L_x(x, \mu)d &= \begin{pmatrix}
                d_1 & 2d_1
            \end{pmatrix}\begin{pmatrix}
                0 & 1 \\ 1 & 0
            \end{pmatrix}\begin{pmatrix}
                d_1 \\ 2d_1
            \end{pmatrix}\\
            &= 4d_1^2\\
            &> 0
        \end{align*}
        故 $\nabla^2L_x(x, \mu)$ 正定,$\bar{x} = \left(\dfrac{3}{4}, -\dfrac{3}{2}\right)$ 是局部最优解。
        \item 对 $x$ 求偏导可得 
        \begin{align*}
            \partial_{x_1}G &= x_2 + c(2x_1 - x_2 - 3) \cdot 2 = 0\\
            \partial_{x_2}G &= x_1 + c(2x_1 - x_2 - 3) \cdot (-1) = 0
        \end{align*}
        解得 
        \[
            x^* = \left(\dfrac{3c}{4c - 1}, \dfrac{-6c}{4c - 1}\right).
        \]
        令 $c \to +\infty$, 得
        \[
            x^* = \left(\dfrac{3}{4}, -\dfrac{3}{2}\right).
        \]
    \end{enumerate}
\end{example}


\begin{note}
    内点罚函数法优点:迭代总在可行域内进行,每一个中间结果都是可行解,可以作为近似解。

    内点罚函数法缺点:选取初始可行点较困难,且只适用于含不等式约束的非线性规划问题,否则没有严格内点。
\end{note}

\subsection{乘子罚函数法}
\subsubsection{等式约束优化的乘子罚函数法}
\begin{note}
    对于等式约束优化问题\[\begin{cases}
        \min &f(x)\\
        \subject &c_i(x) = 0, i \in \varepsilon    
    \end{cases}\]
    $P(x, \lambda, \pi)$ 为乘子罚函数,也称增广 Lagrange 函数。
    \[\min_xP(x, \lambda, \pi) := f(x) - \sum_{i \in \varepsilon}\lambda_ic_i(x) + \pi\sum_{i \in \varepsilon}c_i^2(x)\]
    对于凸规划,任意 $\pi > 0$ 下,原问题的最优解一定是乘子罚函数的最优解。

    并且要使 $c_i(x_k) \to 0$,可以使乘子 $\lambda_i^{(k)}$ 足够靠近最优 Lagrange 乘子 $\lambda_i^*$,而不一定需要罚因子 $\pi_k \to +\infty$.
\end{note}

\begin{note}
    乘子罚函数算法:\begin{enumerate}
        \item 给定 $\pi_0 > 0, \lambda^{(0)} = 0$,初始点 $x_{-1} \in \mathbb{R}^n$,增长因子 $\gamma > 1$ 和允许误差 $\varepsilon > 0$. 令 $k = 0$.
        \item 以 $x_{k - 1}$ 为初始点,用无约束优化方法计算函数 $P(x, \lambda^{(k)}, pi_k)$ 的最小值点 $x_k$.
        \item 若 $\max\{|c_i(x_k)| \mid i \in \varepsilon\} \le \varepsilon$,算法终止。否则,转下一步。
        \item 若 $\norm{c(x_k)}_\infty \ge \norm{c(x_{k - 1})}_\infty$,令 $\pi_{k + 1} = \gamma \pi_k, \lambda^{(k + 1)} = \lambda_k$,置 $\lambda^{(k + 1)} = \lambda^{(k)}$,转步 2. 否则,转步 5.
        \item 若 $\pi_k > \pi_{k - 1}$ 或 $\norm{c(x_k)}_\infty \le \frac{1}{4}\norm{c(x_{k - 1})}_\infty$,令 $\pi_{k + 1} = \pi_k$,根据 $\lambda_i^{(k + 1)} = \lambda_i^{(k)} - 2\pi_kc_i(x_k)$ 调整 $\lambda^{(k + 1)}$,置 $k = k + 1$,转步 2. 否则,令 $\pi_{k + 1} = \gamma \pi_k$,$\lambda^{(k + 1)} = \lambda^{(k)}$,置 $k = k + 1$,转步 2.
    \end{enumerate}
\end{note}



% \nocite{*}
% \printbibliography[heading=bibintoc, title=\ebibname]

% \appendix
% \appendixpage
% \addappheadtotoc

