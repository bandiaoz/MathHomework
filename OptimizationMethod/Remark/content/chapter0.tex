\section*{重点内容}
总成绩=平时成绩(30\%)+ 小论文(20\%)+期末考试成绩(50\%)

\begin{center}
    \begin{tabular}{|l|c|}
        \hline
        \begin{tabular}{l}
            最优化问题简介、发展史、分类 \\
            无约束优化理论 \\
            无约束优化的最优性条件
        \end{tabular} & \\ \hline
        \begin{tabular}{l}
            算法概述和基础 \\
            使用导数的最优化方法, 最速下降法,牛顿法 \\
        \end{tabular} & 重点 \\ \hline
        \begin{tabular}{l}
            牛顿法, 拟牛顿法
        \end{tabular} & 重点 \\ \hline
        \begin{tabular}{l}
            共轭梯度法
        \end{tabular} & 重点 \\ \hline
        \begin{tabular}{l}
            最小二乘法
        \end{tabular} &  \\ \hline
        \begin{tabular}{l}
            约束优化理论\\
            约束优化的最优性条件, Lagrange乘子
        \end{tabular} & 重点 \\ \hline
        \begin{tabular}{l}
            约束优化的最优性条件: 二阶条件,凸规划
        \end{tabular} & 重点 \\ \hline
        \begin{tabular}{l}
            凸规划的性质、对偶理论、鞍点定理
        \end{tabular} & 重点 \\ \hline
        \begin{tabular}{l}
            线性规划的基本性质,极点和基解
        \end{tabular} & \\ \hline
        \begin{tabular}{l}
            线性规划的单纯形方法
        \end{tabular} & \\ \hline
        \begin{tabular}{l}
            约束优化问题的罚函数法
        \end{tabular} & 重点 \\ \hline
        \begin{tabular}{l}
            增广Lagrangian方法/乘子罚函数方法
        \end{tabular} & \\ \hline
    \end{tabular}
\end{center}