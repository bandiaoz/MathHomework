\section{算法概述}
\begin{note}
    一类线搜索下降迭代算法的步骤:
    \begin{enumerate}
        \item 选定某一初始点 $x^{(0)}$,置 $k = 0$;
        \item 确定搜索方向 $d^{(k)}$;
        \item 从 $x^{(0)}$ 出发,沿方向 $d^{(k)}$ 求步长 $\lambda_k$,以产生下一个迭代点 $x^{(k + 1)}$;
        \item 检查 $x^{(k + 1)}$ 是否为极小点或近似极小点,若是,则停止迭代;否则,令 $k:= k + 1$,返回2.
    \end{enumerate}

    选取搜索方向是最关键的一步,各种算法的区别, 主要在于确定搜索方向的方法不同.
\end{note}

\begin{note}
    解集合:把满足某些条件的点集定义为解集合.当迭代点属于该集合时,停止迭代.
    
    常用的解集合:
    \begin{itemize}
        \item $\Omega = \left\{\bar{x}\ |\ \|\nabla f(\bar{x})\| = 0\right\}$
        \item $\Omega = \left\{\bar{x}\ |\ \bar{x} \text{为KKT点}\right\}$
        \item $\Omega = \left\{\bar{x}\ |\ \bar{x} \in S, f(\bar{x}) \le b\right\}$,其中 $b$ 是某个可接受的目标函数值.
    \end{itemize}
\end{note}

\begin{note}
    实用收敛准则
    \begin{itemize}
        \item $\|x^{(k + 1)} - x^{(k)}\| < \varepsilon$ 或者 $\frac{\|x^{(k + 1)} - x^{(k)}\|}{\|x^{(k)}\|} < \varepsilon$
        \item $f(x^{(k)}) - f(x^{(k + 1)}) < \varepsilon$ 或者 $\frac{f(x^{(k)}) - f(x^{(k + 1)})}{|f(x^{(k)})|} < \varepsilon$
        \item $\|\nabla f(x^{(k)})\| < \varepsilon$ (无约束最优化中)
    \end{itemize}
\end{note}

\begin{note}
    Q-收敛速率(Quotient)

    设序列 $\left\{\gamma^{(k)}\right\}$ 收敛于 $\gamma^*$,定义满足 \[\lim _{k \rightarrow+\infty} \frac{\left\|\gamma^{(k+1)}-\gamma *\right\|}{\left\|\gamma^{(k)}-\gamma *\right\|^{p}}=\beta<\infty\] 的非负数 $p$ 的上确界为序列 $\left\{\gamma^{(k)}\right\}$ 的收敛级.
    \begin{itemize}
        \item 若序列的收敛级为 $p$,则称序列是 $p$ 级收敛的.
        \item 若 $p = 1$ 且 $0 < \beta < 1$,则称序列是以收敛比 $\beta$ 线性收敛的.
        \item 若 $p > 1$,或者 $p = 1$ 且 $\beta = 0$,则称序列是超线性收敛的.
        \item 收敛级 $p$ 越大,序列收敛得越快;当收敛级 $p$ 相同时,收敛比 $\beta$ 越小,序列收敛得越快.
    \end{itemize}
\end{note}

\begin{note}
    R-收敛速率(Root):

    设点列 $\left\{x_k\right\}$ 收敛到 $x^*$.若存在 $\kappa>0, q \in(0,1)$ 使 $\|x_k - x^*\| \le \kappa q^k$,则称点列 $\left\{x_k\right\}$ R-线性收敛到 $x^*$;若存在 $\kappa > 0$ 和收敛到 $0$ 的正数列 $\left\{q_k\right\}$ 使 $\|x_k - x^*\| \le \kappa \prod_{i = 1}^k q_i$,则称点列 $\left\{x_k\right\}$ R-超线性收敛到 $x^*$.

    \begin{itemize}
        \item Q-(超)线性收敛 $\Longrightarrow$ R-(超)线性收敛
    \end{itemize}
\end{note}

\begin{note}
    算法的二次终止性:若某个算法对任意的\textbf{正定二次函数},从任意的初始点出发,都能经有限步迭代达到其极小点,则称该算法具有二次终止性.
    
    用二次终止性作为判断算法优劣的原因:
    \begin{enumerate}
        \item 正定二次函数具有某些较好的性质,因此一个好的算法应能够在有限步内达到其极小点.
        \item 对于一般的目标函数,若在其极小点处Hesse矩阵正定\[f(x) =f\left(x^{*}\right)+\nabla f\left(x^{*}\right)^{T}\left(x-x^{*}\right) 
        +\frac{1}{2}(x-x *)^{T} \nabla^{2} f\left(x^{*}\right)\left(x-x^{*}\right)+o\left(\left\|x-x^{*}\right\|\right)\]
        因此可以猜想,对正定二次函数好的算法,对于一般目标函数也应具有较好的性质.
    \end{enumerate}
\end{note}

\begin{note}
    单纯形算法的复杂度为指数时间复杂度.
\end{note}

\begin{note}
    对于线性规划问题 \[\min_{\lambda \ge 0} \quad f(x^{(k)} + \lambda d^{(k)})\] 如果求得的 $\lambda_k$,使得 \[f(x^{(k)} + \lambda_kd^{(k)}) = \min_\lambda f(x^{(k)} + \lambda d^{(k)})\]则称该一维搜索为精确一维搜索,$\lambda_k$ 为最优步长.否则,该一维搜索为非精确一维搜索.
\end{note}

\begin{note}
    一维搜索\begin{itemize}
        \item 精确线搜索\begin{itemize}
            \item 试探法:黄金分割法、Fibonacci法、二分法
            \item 函数逼近法:Newton法、割线法、抛物线法、三次插值法
        \end{itemize}
        \item 非精确线搜索:Armijo 步长规则、Goldstein 步长规则、Wolfe步长规则
    \end{itemize}
\end{note}

\begin{note}
    函数逼近法:牛顿法

    基本思想:在极小点附近用二阶 Taylor 多项式近似. \[\min \quad f(x)\]
    令 $\varphi(x)=f\left(x^{(k)}\right)+f^{\prime}\left(x^{(k)}\right)\left(x-x^{(k)}\right)+\frac{1}{2} f^{\prime \prime}\left(x^{(k)}\right)\left(x-x^{(k)}\right)^{2}$,又令 $\varphi^{\prime}(x)=f^{\prime}\left(x^{(k)}\right)+f^{\prime \prime}\left(x^{(k)}\right)\left(x-x^{(k)}\right)=0$,得 $\varphi(x)$ 的驻点,记 $x^{(k + 1)}$,则 $x^{(k+1)}=x^{(k)}-\frac{f^{\prime}\left(x^{(k)}\right)}{f^{\prime \prime}\left(x^{(k)}\right)}$.

    \begin{theorem}
        设 $f(x)$ 存在连续三阶导数,$\bar{x}$ 满足 $f^\prime(\bar{x}) = 0$,$f^{\prime\prime}(\bar{x}) \neq 0$,初点 $x^{(1)}$ 充分接近 $\bar{x}$,则牛顿法产生的序列 $\left\{x^{(k)}\right\}$ 至少以二级收敛速度收敛于 $\bar{x}$.
    \end{theorem}

    算法步骤:\begin{enumerate}
        \item 给定初始点 $x^{(0)}$,允许误差 $\varepsilon > 0$,置 $k = 0$.
        \item 若 $|f^\prime(x^{(k)})| < \varepsilon$,则停止计算,得点 $x^{(k)}$;否则转 3
        \item 计算点 $x^{(k + 1)} = x^{(k)} - \frac{f^\prime(x^{(k)})}{f^{\prime\prime}(x^{(k)}}$,置 $k = k + 1$,返回 2
    \end{enumerate}

    缺点:初始点选择十分重要.如果初始点靠近极小点,则可能很快收敛;如果初始点远离极小点,迭代产生的点列可能不收敛于极小点.
\end{note}

\begin{note}
    非精确搜索:
    \begin{itemize}
        \item Armijo 步长规则\begin{itemize}
            \item 设 $\beta > 0, \gamma \in (0, 1), \sigma \in (0, 1)$.取步长 $\lambda_k = \beta \gamma^{m_k}$,其中 $m_k$ 是满足下式的最小非负整数:\[f\left(x^{(k)}+\beta \gamma^{m} d^{(k)}\right) \leq f\left(x^{(k)}\right)+\sigma \beta \gamma^{m} \nabla f\left(x^{(k)}\right)^{T} d^{(k)}\]
            \item 根据目标函数的 Taylor 展开式,满足这种规则的步长一定存在.
        \end{itemize}
        \item Goldstein 步长规则\begin{itemize}
            \item 设 $\sigma \in (0, \frac{1}{2})$.取步长满足下式\begin{gather*}
                f\left(x^{(k)}+\lambda d^{(k)}\right) \leq f\left(x^{(k)}\right)+\sigma \lambda \nabla f\left(x^{(k)}\right)^{T} d^{(k)} \\
                f\left(x^{(k)}+\lambda d^{(k)}\right)>f\left(x^{(k)}\right)+(1-\sigma) \lambda \nabla f\left(x^{(k)}\right)^{T} d^{(k)}
            \end{gather*}
            \item 由于 $\lambda > 0$ 充分小时,第二式必不成立,故改规则在保证目标函数下降的前提下,使下一迭代点尽可能远离当前迭代点.
        \end{itemize}
        \item Wolfe 步长规则\begin{itemize}
            \item 设 $0 < \sigma_1 < \sigma_2 < 1$.取步长 $\lambda_k$ 满足下式\begin{gather*}
                f\left(x^{(k)}+\lambda d^{(k)}\right) \leq f\left(x^{(k)}\right)+\sigma_{1} \lambda \nabla f\left(x^{(k)}\right)^{T} d^{(k)} \\
                \nabla f\left(x^{(k)}+\lambda d^{(k)}\right)^{T} d^{(k)} \geq \sigma_{2} \nabla f\left(x^{(k)}\right)^{T} d^{(k)}
            \end{gather*}
            \item 该规则使函数 $f(x^{(k)} + \lambda d^{(k)})$ 的陡度在 $\lambda_k$ 点比在 $\lambda = 0$ 点有所减缓,从而使下一迭代点尽可能远离当前迭代点.
        \end{itemize}
    \end{itemize}
\end{note}
