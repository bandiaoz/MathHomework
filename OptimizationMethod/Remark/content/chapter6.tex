\section{非线性规划的最优性条件}
\subsection{无约束优化的最优性条件}
\begin{note}
    无约束优化问题的最优性条件\[\min f(x) \quad \text{s.t. } x \in E^n\]
    \begin{itemize}
        \item 定义:对$\min f(x)$,设 $\bar{x} \in E^n$ 是任给一点,$d\neq 0$ ,若存在 $\delta > 0$,使得对任意的 $\lambda \in (0, \delta)$,有 $f(\bar{x} + \lambda d) < f(\bar{x})$,则称 $d$ 为 $f(x)$ 在点 $\bar{x}$ 处的下降方向.
        \item 引理:设函数 $f(x)$ 在点 $\bar{x}$ 可微,若存在 $d \neq 0$ 使 $\nabla f(\bar{x})^td < 0$,则存在 $\delta > 0$,使对 $\forall \lambda\in (0, \delta)$,有 $f(\bar{x} + \lambda d) < f(\bar{x})$
    \end{itemize}
\end{note}

\begin{theorem}
    无约束优化问题的最优性条件\begin{itemize}
        \item 一阶必要条件:设函数 $f(x)$ 在点 $\bar{x}$ 处可微,若 $\bar{x}$ 是局部极小点,则 $\nabla f(\bar{x}) = 0$.
        \item 二阶必要条件:设函数 $f(x)$ 在 $\bar{x}$ 处二阶可微,若 $\bar{x}$ 是局部极小点,则 $\nabla f(\bar{x}) = 0$,且 Hessian 矩阵 $\nabla^2f(\bar{x})$ 是半正定的.
        \item 充分条件:设函数 $f(x)$ 在点 $\bar{x}$ 处二次可微,若梯度 $\nabla f(\bar{x}) = 0$ ,且 Hessian矩阵 $\nabla^2f(\bar{x})$ 正定,则 $\bar{x}$ 是严格局部极小点.
        \item 设函数 $f(x)$ 在点 $\bar{x}$ 的邻域内二次可微,若梯度 $\nabla f(\bar{x}) = 0$,且Hessian矩阵 $\nabla^2f(x)$ \textbf{在该邻域内}半正定,则 $\bar{x}$ 是局部极小点.特别地,对于邻域内的任意点 $x\neq \bar{x}$,若 $\nabla^2f(x)$ 是正定矩阵,则 $\bar{x}$ 是一个\textbf{严格}的局部极小点.
        \item 设 $f(x)$ 是定义在 $E^n$ 上的可微凸函数,$\bar{x} \in E^n$,则 $\bar{x}$ 为整体极小点的充要条件是 $\nabla f(\bar{x}) = 0$.
    \end{itemize}
\end{theorem}

\begin{note}
    驻点\begin{itemize}
        \item 定义:若 $f(x)$ 在点 $x^*$ 可微,并且 $\nabla f(x^*) = 0$.则 $x^*$ 称为 $f(x)$ 的一个驻点(平稳点),既不是极小点,也不是极大点的驻点称为鞍点.
    \end{itemize}
\end{note}

\subsection{约束优化问题的最优性条件}
\begin{note}
    约束优化问题的最优性条件
    \begin{itemize}
        \item 定义:对 $\underset{x \in E^n}{\min}f(x)$,设 $\bar{x} \in E^n$ 是任给一点,$d\neq 0$,若存在 $\delta > 0$,使得对任意的 $\lambda \in (0, \delta)$ 有 $f(\bar{x} + \lambda d) < f(\bar{x})$,则称 $d$ 为 $f(x)$ 在点 $\bar{x}$ 处的下降方向.\[F_0 = \left\{d\ |\ \nabla f(\bar{x})^td < 0\right\}\]称为点 $\bar{x}$ 处的\textbf{下降方向集}.
        \item 定义:设集合 $S \subset E^n$,$\bar{x} \in clS$,$d$ 为非零向量,若存在数 $\delta > 0$,使得对任意 $\lambda \in (0, \delta)$,都有 $\bar{x} + \lambda d \in S$ 则称 $d$ 为集合 $S$ 在 $\bar{x}$ 的\textbf{可行方向}. 
        
        $D=\{d \mid d \neq 0, \bar{x} \in c l S, \exists \delta>0, \forall \lambda \in(0, \delta), \bar{x}+\lambda d \in S\}$ 是 $\bar{x}$ 处的\textbf{可行方向锥}.
        \item 几何最优性条件:考虑问题\[\min f(x), \quad \text{s.t. } x \in S\]设 $S$ 是 $E^n$ 的非空集合,$\bar{x} \in S$,$f(x)$ 在 $\bar{x}$ 处可微,若 $\bar{x}$ 是局部最优解,则下降方向集和可行方向锥没有交集. 即 $F_0\cap D = \varnothing$.
    \end{itemize}
\end{note}

\begin{note}
    不等式约束优化问题的一阶最优性条件\[\begin{cases}
        \min \quad &f(x)\\
        \text{s.t.} \quad &g_i(x) \ge 0, \quad i = 1, \dots, m
    \end{cases}\]
    可行域 $S = \left\{x\ |\ g_i(x) \ge 0, i = 1, \dots, m\right\}$
    \begin{itemize}
        \item 定义:若一个可行点 $\bar{x}$($\bar{x} \in S$)使某个不等式约束$g_i(x) \ge 0$变成等式,即 $g_i(\bar{x}) = 0$,则该不等式约束称为关于可行点 $\bar{x}$ 的起作用约束(或等式约束);否则,若 $\bar{x}$ 使得某个 $g_i(\bar{x}) > 0$,则该不等式约束称为关于可行点 $\bar{x}$ 的不起作用约束(或松约束).
        
        记 $I = \left\{i \mid g_i(\bar{x}) = 0, \bar{x} \in S\right\}$ 为起作用约束(等式约束).
        \item $G_0 = \left\{d \mid \nabla g_i(\bar{x})^td > 0, i \in I\right\}$ 称为 $S$ 在点 $\bar{x}$ 处的\textbf{局部约束方向锥(内方向锥)}
        \item 几何最优性条件:设 $\bar{x} \in S$,$f(x)$ 和 $g_i(x)(i \in I)$ 在 $\bar{x}$ 处可微,$g_i(x)(i \notin I)$ 在 $\bar{x}$ 处连续,如果 $\bar{x}$ 是局部最优解,则下降方向集和内方向锥的交集为空,即 $F_0\cap G_0 = \varnothing$.
    \end{itemize}
\end{note}

\begin{theorem}
    Fritz John 条件:设 $\bar{x} \in S$, $f(x), g_i(x)(i \in I)$ 在 $\bar{x}$ 处可微,$g_i(x)(i\notin I)$ 在 $\bar{x}$ 处连续,若 $\bar{x}$ 是局部最优解,则存在不全为零的数 $w_0, w_i(i \in I)$,使得 \[\begin{cases}
        w_{0} \nabla f(\bar{x})-\sum_{i \in I} w_{i} \nabla g_{i}(\bar{x})=0 \\
        w_{0}, w_{i} \geq 0, \quad i \in I
    \end{cases}\] $\bar{x}$ 称为 Fritz John 点,即满足Fritz John条件的点.
\end{theorem}

\begin{theorem}
    KKT 条件:设 $\bar{x} \in S$, $f(x), g_i(x)(i \in I)$ 在 $\bar{x}$ 处可微,$g_i(x)(i\notin I)$ 在 $\bar{x}$ 处连续,若 $\bar{x}$ 是局部最优解,存在非负数 $w_i, i \in I$,使得\[\nabla f(\bar{x}) - \sum_{i \in I}w_i\nabla g_i(\bar{x}) = 0\]
    
    设 $\bar{x} \in S, f, g_i$ 在 $\bar{x}$ 可微,$\left\{\nabla g_i(\bar{x}) \mid i \in I\right\}$;线性无关,若 $\bar{x}$ 是局部最优解,则存在数 $w_i, i=1, 2, \dots, m$,使得\[\begin{array}{l}
        \nabla f(\bar{x})-\sum_{\mathrm{i}=1}^{m} w_{i} \nabla g_{i}(\bar{x})=0 \\
        w_{i} g_{i}(\bar{x})=0 \quad i=1,2, \cdots, m \\
        w_{i} \geq 0 \quad i=1,2, \cdots, m 
    \end{array}\]

    对于凸规划,则若在 $\bar{x}$ 点处KKT条件成立,则 $\bar{x}$ 为整体极小值点.
\end{theorem}

\subsection{一般约束问题的一阶最优性条件}
\begin{theorem}
    设 $\bar{x} \in S$,$f(x)$ 和 $g_i(x)(i\in I)$ 在 $\bar{x}$ 处可微,$g_i(x) (i \notin I)$ 在 $\bar{x}$ 处连续,$h_j(j = 1, \dots, l)$ 在 $\bar{x}$ 处连续可微,且 $\bar{x}$ 是 $\S = \{x \mid h(x) = 0\}$ 的正则点.如果 $\bar{x}$ 是问题(NP)的局部最优解,则在 $\bar{x}$ 处,有\[F_{0} \cap G_{0} \cap H_{0}=\varnothing\]其中
    \begin{align*}
        F_{0}&=\left\{d \mid \nabla f(\bar{x})^{T} d<0\right\} \\
        G_{0}&=\left\{d \mid \nabla g_{i}(\bar{x})^{T} d>0, i \in I\right\} \\
        H_{0}&=\left\{d \mid \nabla h_{j}(\bar{x})^{T} d=0, j=1, \cdots, l\right\}
    \end{align*}
\end{theorem}

\begin{theorem}
    若系统 $Ax < 0$,$Bx = 0$ 无解,则系统 $A^ty + B^tz = 0, y \ge 0$,且 $y \neq 0$ 或 $z \neq 0$ 有解.
\end{theorem}

\begin{note}
    一阶充分条件:\[\begin{cases}
        \min &f(x)\\
        \st &g_i(x) \ge 0 \quad i = 1, \dots, m\\
        &h_j(x) = 0, \quad j = 1, \dots, l
    \end{cases}\]
    $f$ 是凸函数,$g_i$ 是凹函数,$h_j$ 是线性函数,$S$ 为可行域,$\bar{x} \in S$,$I = \{i \mid g_i(\bar{x}) = 0\}$. $f$ 和 $g_i(i \in I)$ 在点 $\bar{x}$ 可微,$h_j$ 在点 $\bar{x}$ 连续,$g_i(i \notin I)$ 在点 $\bar{x}$ 连续,且在 $\bar{x}$ 处 KKT 条件成立,则 $\bar{x}$ 为整体极小点.
\end{note}
