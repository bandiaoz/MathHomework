\section{LP 基本性质}
LP:线性规划.

\subsection{LP标准形}
\begin{definition}
    LP 标准形:
    \begin{enumerate}
        \item 极小化型
        \item 约束方程为等式
        \item 所有的决策变量为非负值
        \item 约束方程的右端项系数为非负值
    \end{enumerate}
    \begin{align*}
        \min \quad &z = c^tx\\
        \subject \quad &Ax = b\\
        &x \succeq 0\\
        &c, x \in \R^n, b \in \R^m_+, A \in \R^{m \times n}
    \end{align*}
\end{definition}

\begin{note}
    非标准形LP模型转化为标准形LP模型
    \begin{itemize}
        \item 目标函数是极大值的转化
        \[
            \max c^tx \Longrightarrow \min -c^tx  
        \]
        \item 决策变量无约束转化为非负约束
        \[
            x = x^{+} - x^{-}, \quad x^{+}, x^{-} \ge 0
        \]
        \item 不等约束转化为等式约束(松弛变量)
        \begin{gather*}
            a^tx \le b \Longrightarrow a^tx + s = b, \quad s \ge 0\\
            a^tx \ge b \Longrightarrow a^tx - s = b, \quad s \ge 0
        \end{gather*}
        \item 决策变量有上下界的转换
        \begin{gather*}
            1 \le x_1 \le 6\\
            x_1^{'} = x_1 - 1 \ge 0,x_1^{'} + x_2 = 5, x_2 \ge 0
        \end{gather*}
        \item 带绝对值
        \[\begin{cases}
            x_1 = \frac{x + |x|}{2}\\
            x_2 = \frac{x - |x|}{2}
        \end{cases}\]
        则 $x_1, x_2 \ge 0$,$x = x_1 - x_2, |x| = x_1 + x_2$.
    \end{itemize}
\end{note}

\begin{example}
    \[
        \begin{cases}
            \max \quad &3x_1 - 2x_2 + 3\\
            \subject \quad &x_1 + x_2 \le 7\\
            &x_1 - x_2 + x_3 \ge 5\\
            &1 \le x_3 \le 6\\
            &x_1 \ge 0, x_2 \text{ free}
        \end{cases}  
    \]
    转化为LP模型.

    \answer
    \[
        \begin{cases}
            \min \quad &-3x_1 + 2(x_2^{+} - x_2^{-}) - (x_3^{'} + 1)\\
            \subject \quad &x_1 + (x_2^{+} - x_2^{-}) + x_4 = 7\\
            &x_1 - (x_2^{+} - x_2^{-}) + x_3^{'} - x_5 = 4\\
            &x_3^{'} + x_6 = 5\\
            &x_1, x_2^{+}, x_2^{-}, x_3^{'}, x_4, x_5, x_6 \ge 0
        \end{cases}    
    \]
\end{example}

\begin{example}
    \[
        \begin{cases}
            \max \quad &-|x| - |y|\\
            \subject \quad &x + y \ge 2\\
            &x \le 3
        \end{cases}  
    \]
    转换为LP模型.

    \answer
    \[
        \begin{cases}
            x_1 = \frac{x + |x|}{2}\\
            x_2 = \frac{x - |x|}{2}
        \end{cases},
        \begin{cases}
            y_3 = \frac{y + |y|}{2}\\
            y_4 = \frac{y - |y|}{2}
        \end{cases}
    \]
    则 
    \begin{gather*}
        x_1 \ge 0, x_2 \ge 0, x = x_1 - x_2, |x| = x_1 + x_2\\
        x_3 \ge 0, x_4 \ge 0, y = x_3 - x_4, |y| = x_3 + x_4
    \end{gather*}
    \[
        \begin{cases}
            \min \quad &(x_1 + x_2) + (x_3 + x_4)\\
            \subject \quad &x_1 - x_2 + x_3 - x_4 - x_5 = 2\\
            &x_1 - x_2 + x_6 = 3\\
            &x_i \ge 0, i = 1, \dots, 6
        \end{cases}  
    \]
\end{example}

\begin{example}
    将下列线性规划变成标准形式
    \[
        \begin{cases}
            \min \quad &3x_1 - x_2\\
            \subject \quad &x_1 + x_2 \le 9\\
            &1 \le x_1 \le 5\\
            &0 \le x_2 \le 6
        \end{cases}  
    \]

    \answer
    \[
        \begin{cases}
            \min \quad &3(x_1^{'} + 1) - x_2\\
            \subject \quad &x_1^{'} + x_2 + x_3 = 8\\
            &x_1^{'} + x_4 = 4\\
            &x_2 + x_5 = 6\\
            &x_1^{'}, x_2, x_3, x_4, x_5 \ge 0
        \end{cases}
    \]  
\end{example}

\subsection{LP的图解法}
\begin{note}
    线性不等式的几何意义——半平面.

    图解法步骤:
    \begin{enumerate}
        \item 作出LP问题的可行域
        \item 作出目标函数的等值线
        \item 移动等值线到可行域边界得到最优点
    \end{enumerate}
\end{note}

\begin{theorem}
    若LP问题存在最优解,则必在可行域的某个极点上找到。
\end{theorem}

\subsection{LP的基本性质}
\begin{definition}[可行解]
    满足LP模型的约束条件且满足非负条件的解是可行解.
\end{definition}

\begin{example}
    \[
        \begin{cases}
            \max \quad &z = 3x_1 + 2x_2\\
            \subject \quad &x_1 + 3x_2 \ge 6\\
            &x_1 - 2x_2 \le 4\\
            &x_1, x_2 \ge 0
        \end{cases}
    \]
    判断 $X = (5, 1)^t$,$X = (-1, 3)^t$,$X = (2, 1)^t$ 是否为可行解。

    \answer 只有 $X = (5, 1)^t$ 是可行解。
\end{example}

\begin{theorem}
    线性规划的可行域是凸集。
\end{theorem}

\begin{theorem}
    设线性规划的可行域非空,则
    \begin{enumerate}
        \item LP存在有限最优解的充要条件是对任意的 $j$,$cd^{(j)} \ge 0$,其中 $d^{(j)}$ 为可行域的极方向,$c$ 为目标函数的系数。
        \item 若LP存在有限最优解,则目标函数的最优值可在某个极点达到。
    \end{enumerate}
\end{theorem}

\begin{definition}[基矩阵和基变量]\label{def2-3}
    对LP问题
    \[
        \begin{cases}
            \min \quad &z = cx\\
            \subject \quad &Ax = b\\
            &x \ge 0
        \end{cases}  
    \]
    将 $A$ 按列分块为 $(P_1, \dots, P_n)$,则 $Ax = b$ 等价于 
    \[
        P_1x_1 + \cdots + P_nx_n = b.
    \]
    系数矩阵 $A$ 中任意 $m$ 列所组成的 $m$ 阶可逆子方阵 $B$,称为LP的一个基(矩阵),变量 $x_j$ 对应的 $P_j$ 包含在基 $B$ 中,则称 $x_j$ 为基变量,否则称为非基变量。

    基变量的个数最多为 $\binom{n}{m}$。
\end{definition}

\begin{definition}[基本解]\label{def2-4}
    接定义\ref{def2-3},设 $A = \begin{bmatrix} B & N \end{bmatrix}$,其中 $r(B) = m$,设 $x = \begin{bmatrix}x_B & x_N\end{bmatrix}$.

    由 $Ax = b$ 得
    \[
        Bx_B + Nx_N = b  
    \]
    解得
    \[
        x_B = B^{-1}b - B^{-1}Nx_N    
    \]
    令 $x_N = 0$,得
    \[
        x = \begin{bmatrix}
            B^{-1}b\\
            0
        \end{bmatrix}    
    \]
    称 $x$ 为LP的\textbf{基本解}。
\end{definition}

\begin{definition}[基本可行解]
    接定义\ref{def2-3},若 $B^{-1}b \ge 0$,则称 $x = \begin{bmatrix}
        B^{-1}b\\
        0
    \end{bmatrix}$ 为LP的\textbf{基本可行解},$B$ 称为可行基矩阵,$x_{B_1}, \dots, x_{B_m}$ 为一组可行基。

    若 $B^{-1}b > 0$,则称基本可行解是非退化的,否则称为退化的。
\end{definition}

\begin{example}
    优化问题的约束条件为 
    \[
        \begin{cases}
            x_1 + 3x_2 \ge 6\\
            x_1 - 2x_2 \le 4\\
            x_1, x_2 \ge 0
        \end{cases}    
    \]
    首先引入松弛变量,转化为等式约束
    \[
        \begin{cases}
            x_1 + 3x_2 - x_3 = 6\\
            x_1 - 2x_2 + x_4 = 4
        \end{cases}
    \]
    系数矩阵 
    \[A = \begin{pmatrix}
        1 & 3 & -1 & 0\\
        1 & -2 & 0 & 1
    \end{pmatrix}\]
    基矩阵为 
    \begin{gather*}
        B_1 = \begin{pmatrix}
            1 & 3\\
            1 & -2
        \end{pmatrix}, B_2 = \begin{pmatrix}
            1 & -1\\
            1 & 0
        \end{pmatrix}, B_3 = \begin{pmatrix}
            1 & 0\\
            1 & 1
        \end{pmatrix},\\
        B_4 = \begin{pmatrix}
            3 & -1\\
            -2 & 0
        \end{pmatrix}, B_5 = \begin{pmatrix}
            3 & 0\\
            -2 & 1
        \end{pmatrix}, B_6 = \begin{pmatrix}
            -1 & 0\\
            0 & 1
        \end{pmatrix}
    \end{gather*}
    对于 $B_1$ 求基解
    \[
        B_1^{-1}b = -\frac{1}{5}\begin{pmatrix}
            -2 & -3\\
            -1 & 1
        \end{pmatrix}\begin{pmatrix}
            6\\
            4
        \end{pmatrix} = \begin{pmatrix}
            \frac{24}{5}\\
            \frac{2}{5}
        \end{pmatrix}
    \]
    基本解为 $x^{(1)} = \left(\frac{24}{5}, \frac{2}{5}, 0, 0\right)^t$.
    
    同理可得
    \begin{gather*}
        x^{(1)} = (\frac{24}{5}, \frac{2}{5}, 0, 0)^t, x^{(2)} = (4, 0, -2, 0)^t, x^{(3)} = (6, 0, 0, -2)^t\\
        x^{(4)} = (0, -2, -12, 0)^t, x^{(5)} = (0, 2, 0, 8)^t, x^{(6)} = (0, 0, -6, 4)^t
    \end{gather*}
    只有 $x^{(1)}$ 和 $x^{(5)}$ 为基本可行解。
\end{example}

\begin{theorem}
    基本可行解与极点之间的关系:
    \begin{enumerate}
        \item 可行解 $\bar{x}$ 是基本可行解 $\Longleftrightarrow$ $\bar{x}$ 的非零分量所对应的 $A$ 的列向量线性无关。
        \item 设 $S$ 是LP的可行域,$\bar{x} \in S$,则 $\bar{x}$ 是 $S$ 的极点等价于 $\bar{x}$ 是LP的基本可行解。
    \end{enumerate}
\end{theorem}

\begin{theorem}
    基本可行解的存在性:
    \begin{enumerate}
        \item 如果LP有可行解,则一定存在基本可行解。
        \item 如果LP有最优解,则一定存在一个基本可行解是最优解。
        \item 如果LP问题有最优解,则要么最优解唯一,要么有无穷多最优解。
    \end{enumerate}
\end{theorem}