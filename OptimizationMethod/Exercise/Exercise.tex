% \documentclass[12pt]{article}
\documentclass[12pt]{ctexart}
\usepackage[utf8]{inputenc}

\usepackage[english]{babel}
\usepackage[dvips]{epsfig}
\usepackage{amsmath}
\usepackage{amssymb}
\usepackage{amsfonts}
\usepackage{amsthm}
\usepackage{amsbsy}
\usepackage{amsgen}
\usepackage{amscd}
\usepackage{amsopn}
\usepackage{amstext}
\usepackage{amsxtra}
\usepackage{mathrsfs}
\usepackage{enumitem}
\usepackage{graphicx}
\usepackage{verbatim}
\usepackage{epstopdf}
\usepackage{float}
\usepackage[all,cmtip]{xy}
\usepackage{accents}
\usepackage{sseq}
\usepackage{url}
\usepackage{hyperref}
\usepackage{makeidx}
\usepackage{siunitx}
\usepackage{xcolor}
\usepackage{physics}

%%%%%%%%% 版面设置 %%%%%%%%%%%%%%%%%%%%%%%%%%%%%%%%%%%%%%
\usepackage{geometry}
\usepackage{titlesec}
\usepackage{fancyhdr}\pagestyle{empty}
\titleformat*{\section}{\large\bfseries}

%
\geometry{
	a4paper,
	total={170mm,240mm},
	left=20mm,
	top=30mm,
}

%Bitte nicht einstellen
\renewcommand{\figurename}{Abbildung}
\renewcommand{\tablename}{Tabelle}
\pagestyle{fancyplain}
\headheight 35pt
\lhead{\name}
\chead{\textbf{\Large \Title}}
\rhead{\due\\\today}
\lfoot{}
\cfoot{}
\rfoot{\small\thepage}
\headsep 1.5em

%%%%%%%%%%%%%%%%%%%%%%%%%%%%%%%%%%%%%%%%%%%%%%%%%%%%%%

\newtheorem{thm}{Theorem}[section]

% 定义解题环境
\theoremstyle{remark}
\newtheorem{remark}[thm]{Remark}
\newtheorem{theorem}{Theorem}
\newtheorem{observation}[thm]{Observation}

\theoremstyle{definition}
\newtheorem{problem}{\text{}}
\newtheorem{Problem}{\text{Problem}}
\newtheorem*{solution}{解}
\newtheorem*{Answer}{Answer}
\newtheorem{example}{Example} 

%%%%%%%%%%%%%%%%%%%%%%%%%%%%%%%%%%%%%%%%%%%%%%%%%%%%%%%%%%%%%%%%%%
\newcommand\name{陈景龙22120307}
\newcommand\due{-}
\newcommand{\emptyline}{\vspace{0.6\baselineskip}}

\newcommand\Title{最优化习题}

\newcommand\subject{\operatorname{s.t.}}
\newcommand\dom{\operatorname{dom}} % 定义域

\renewcommand{\labelenumi}{(\arabic*)}

\begin{document}
\begin{Problem}
    设函数 $f:\mathbb{R}^n\to \mathbb{R}$. 矩阵 $A \in \mathbb{R}^{n\times n}$ 对称,向量 $b \in \mathbb{R}^n$.
    \begin{enumerate}
        \item 写出函数 $f$ 是凸函数的定义,并列出至少两个判定函数 $f$ 是凸函数的充要条件。
        \item 设 $f(x_1, x_2) = 10 - 2(x_2 - x_1^2)^2$, $S = \{(x_1, x_2) \mid -11 \le x_1 \le 1, -1 \le x_2 \le 1\}$. 判断函数 $f(x_1, x_2)$ 是否为 $S$ 上的凸函数?说明理由。
        \item 证明 $f(x) = \frac{1}{2}x^tAx + b^tx$,为严格凸函数的充要条件是其 Hessian 阵 $A$ 正定。
    \end{enumerate}

    \Answer \text{} \begin{enumerate}
        \item 定义:$\dom(f)$ 是凸集,并且对于 $\forall x, y \in \dom(f), 0 \le \lambda \le 1$, 都有 $f(\lambda x + (1 - \lambda)y) \le \lambda f(x) + (1 - \lambda)f(y)$.
        
        判定充要条件:
        \begin{enumerate}
            \item 函数 $f$ 是定义在非空开凸集 $S$ 上的可微函数,$\forall x, y \in \dom(f)$, 都有 $f(y) \ge f(x) + \nabla f(x)^t(y - x)$.
            \item 函数 $f$ 是定义在非空开凸集 $S$ 上的二次可微函数,$\forall x \in S$, 都有 $\nabla^2f(x) \succeq 0$.
        \end{enumerate}
        \item $\nabla^2f = \begin{pmatrix}
            8x_2 - 24x_1^2 & 8x_1 \\
            8x_1 & -4
        \end{pmatrix}$ 不是半正定矩阵,故函数 $f$ 不是凸函数。
        \item \begin{proof}\text{}
            \begin{enumerate}
                \item 严格凸 $\to$ 正定:
                
                由一阶条件可得 
                \[f(\bar{x} + \lambda x) > f(\bar{x}) + \lambda \nabla f(\bar{x})x.\] 
                并且
                \[f(\bar{x} + \lambda x) = f(\bar{x}) + \lambda \nabla f(\bar{x})^tx + \frac{1}{2}\lambda^2 x^t\nabla^2f(\bar{x})x + \lambda^2\norm{x}^2a,\ \underset{\lambda \to 0}{\lim} a = 0.\] 
                故 
                \[\frac{1}{2}\lambda^2 x^t\nabla^2f(\bar{x})x + \lambda^2\norm{x}^2a > 0.\] 
                两边除 $\lambda^2$, 令 $\lambda \to 0$ 得 $x^t\nabla^2f(\bar{x})x > 0$, 故 Hessian 阵 $A$ 正定。
                \item 正定 $\to$ 严格凸:
                
                $\forall x, \bar{x} \in \dom(f)$, 二阶 Taylar 展开可得:
                \[f(x) = f(\bar{x}) + \nabla f(\bar{x})^t(x - \bar{x}) + \frac{1}{2}(x - \bar{x})^t\nabla^2f(\xi)(x - \bar{x}).\]
                其中 $\xi = \lambda \bar{x} + (1 - \lambda)x, \lambda \in (0, 1).$

                因为 $\dom(f)$ 是凸集,故 $\xi \in \dom(f)$, 又 $\nabla^2f$ 正定,可得 
                \[\frac{1}{2}(x - \bar{x})\nabla^2f(\xi)(x - \bar{x}) > 0\]
                故 
                \[f(x) > f(\bar{x}) + \nabla f(\bar{x})^t(x - \bar{x})\]
                即 $f$ 是严格凸函数。
            \end{enumerate}
        \end{proof}
    \end{enumerate}
\end{Problem}


\begin{Problem}
    设 $S \subseteq \mathbb{R}^n$, 函数 $f: S \to \mathbb{R}$ 二阶连续可微。考虑约束优化问题 (P1):
    \[\begin{cases}
        \min & f(x)\\
        \subject & x \in S
    \end{cases}\]
    \begin{enumerate}
        \item 约束优化问题 (P1) 在什么条件下是凸规划?对于凸规划,你知道有什么好的性质?
        \item 考虑如下优化问题 (P2):\[\begin{cases}
            \min & x_2^2 + 4x_1 - 7x_2\\
            \subject & x_1 - x_2 \le 4\\
            &x_2 \le 3\\
            &6x_1 - x_1^2 + x_2 \ge 8
        \end{cases}\]
        (P2) 是否是凸规划?说明理由。根据最优性条件求 (P2) 的最优解。
    \end{enumerate}

    \Answer \text{}\begin{enumerate}
        \item 当 $f$ 是凸函数,不等约束 $g_i(x) \le 0$ 是凸函数,等式约束是线性函数时,(P1) 是凸规划。凸规划下,函数的局部极小点就是整体极小点,且极小点的集合是凸集。
        \item 优化目标 $f(x) = x_2^2 + 4x_1 - 7x_2$ 是凸函数,约束条件都是线性的,故 (P2) 是凸规划。
        \[\begin{cases}
            4 + \lambda_1 + (-6 + 2x_1)\lambda_3 &= 0\\
            2x_2 - 7 - \lambda_1 + \lambda_2 - \lambda_3 &= 0\\
            \lambda_1(x_1 - x_2 - 4) &= 0\\
            \lambda_2(x_2 - 3) &= 0\\
            \lambda_3(-6x_1 + x_1^2 - x_2 + 8) &= 0\\
            \lambda_1, \lambda_2, \lambda_3 &\ge 0
        \end{cases}\]
        解得 $\lambda_1 = \lambda_2 = 0, \lambda_3 = 2\sqrt{6} - 4$, $x^* = \left(2 - \dfrac{\sqrt{6}}{2}, \dfrac{3}{2} + \sqrt{6}\right)^t$.
    \end{enumerate}
\end{Problem}

\begin{Problem}
    用最速下降法,求解下列问题:
    \[\min \ x_1^2 - 2x_1x_2 + 4x_2^2 + x_1 - 3x_2\]
    取初始点 $x^{(1)} = (1, 1)^t$, 迭代两次。
    \Answer $f(x) = x_1^2 - 2x_1x_2 + 4x_2^2 + x_1 - 3x_2$, $\nabla f(x) = (2x_1 - 2x_2 + 1, -2x_1 + 8x_2 - 3)^t$. 
    
    第一次迭代 
    \begin{enumerate}
        \item $d^{(1)} = (-1, -3)^t$.
        \item $\underset{\lambda \ge 0}{\min} \varphi(\lambda) = \underset{\lambda \ge 0}{\min}f(x^{(1)} + \lambda d^{(1)})$, 得 $\lambda = -\frac{5}{31}$.
        \item $x^{(2)} = x^{(1)} + \lambda d^{(1)} = (\frac{26}{31}, \frac{16}{31})$.
    \end{enumerate}

    第二次迭代
    \begin{enumerate}
        \item $d^{(2)} = \frac{17}{31}(-3, 1)^t$
        \item $\underset{\lambda \ge 0}{\min} \varphi(\lambda) = \underset{\lambda \ge 0}{\min}f(x^{(2)} + \lambda d^{(2)})$, 得 $\lambda = \frac{5}{19}$.
        \item $x^{(3)} = x^{(2)} + \lambda d^{(2)} = \frac{1}{589}(239, 389)^t$. 
    \end{enumerate}
\end{Problem}

\begin{Problem}
    设 $Q \in \mathbb{R}^{n \times n}$ 对称正定,$b \in \mathbb{R}^n$ 且 $b\neq 0$, 考虑非线性规划问题 (P3):
    \[\begin{cases}
        \min &\frac{1}{2}x^tQx\\
        \subject &x \ge b.
    \end{cases}\]
    \begin{enumerate}
        \item 写出 (P3) 的 Lagrange 对偶规划。
        \item 设 $x^*$ 是 (P3) 的最优解,证明 $x^*$ 与 $x^* - b$ 关于 $Q$ 共轭。
    \end{enumerate}

    \Answer \text{} \begin{enumerate}
        \item $g(\lambda) = \underset{x}{\inf}\ \frac{1}{2}x^tQx + \lambda(b - x)$
        \item \begin{proof}
            由 Lagrange 对偶可得,$L(x, \lambda) = \frac{1}{2}x^tQx + \lambda^t(b - x)$
            \[\begin{cases}
                Qx - \lambda &= 0\\
                \lambda &\succeq 0\\
                \lambda_i(b_i - x_i) &= 0
            \end{cases}\]
            由 $\lambda = Qx$ 以及 $\lambda_i(b_i - x_i) = 0$ 可得,最优解 $x^*$ 满足 $x^tQ(b - x) = 0$, 即 $x^*$ 与 $x^* - b$ 关于 $Q$ 共轭。
        \end{proof}
    \end{enumerate}
\end{Problem}

\begin{Problem}
    考虑线性规划问题 (P4):
    \[\begin{cases}
        \max & 8x_1 - 9x_2 + 12x_3 + 4x_3 + 11x_5\\
        \subject &2x_1 - 3x_2 + 4x_3 + x_4 + 3x_5 \le 1\\
        &x_1 + 7x_2 + 3x_3 - 2x_4 + x_5 \le 1\\
        &5x_1 + 4x_2 - 6x_3 + 2x_4 + 3x_5 \le 22\\
        &x_1, x_2, x_3, x_4, x_5 \ge 0\text{.}
    \end{cases}\]
    \begin{enumerate}
        \item 写出 (P4) 的对偶规划。
        \item 利用对偶理论判断 $x^* = (0, 2, 0, 7, 0)$ 是否是 (P4) 的最优解,说明理由。
    \end{enumerate}

    \Answer \text{} \begin{enumerate}
        \item \[\begin{cases}
            \min &w_1 + w_2 + 22w_3\\
            \subject &2w_1 + w_2 + 5w_3 \ge 8\\
            &-3w_1 + 7w_2 + 4w_3 \ge -9\\
            &4w_1 + 3w_2 - 6w_3 \ge 12\\
            &w_1 - 2w_2 + 2w_3 \ge 4\\
            &3w_1 + w_2 + 3w_3 \ge 11\\
            &w_1, w_2, w_3 \ge 0
        \end{cases}\]
        \item 如果 $x^* = (0, 2, 0, 7, 0)^t$ 是最优解,那么由互补松弛定理可得 
        \[\begin{cases}
            -3w_1 + 7w_2 + 4w_3 &= -9\\
            w_1 - 2w_2 + 2w_3 &= 4\\
            w_2 &= 0
        \end{cases}\]
        解得对偶问题的最优解为 $(w_1, w_2, w_3) = \left(\dfrac{34}{10}, 0, \dfrac{3}{10}\right)$. 两个问题最优解相同,故 $x^* = (0, 2, 0, 7, 0)$ 是最优解。
    \end{enumerate}
\end{Problem}

\begin{Problem}
    考虑下列问题 (P5):
    \[\begin{cases}
        \min &x_1x_2\\
        \subject &g(x) = -2x_1 + x_2 + 3 \ge 0.
    \end{cases}\]
    \begin{enumerate}
        \item 用二阶最优性条件证明点 $\bar{x} = \left(\frac{3}{4}, -\frac{3}{2}\right)$ 是局部最优解,并说明它是否为全局最优解。
        \item 定义障碍函数 \[G(x, r) = x_1x_2 - r\ln g(x)\text{,}\] 试用内点法求解 (P5), 并说明内点法产生的序列趋向点 $\bar{x}$.
    \end{enumerate}

    \Answer \text{}
    \begin{enumerate}
        \item $L(x, \lambda) = x_1x_2 + \lambda(2x_1 - x_2 - 3)$, 由 KKT 条件
        \[\begin{cases}
            \lambda(2x_1 - x_2 - 3) &= 0\\
            \lambda &\ge 0\\
            \partial_{x_1}L = x_2 + 2\lambda &= 0\\
            \partial_{x_2}L = x_1 - \lambda &= 0\\
        \end{cases}\]
        解得 
        \[x = \left(\frac{3}{4}, -\frac{3}{2}\right)^t, \quad \lambda = \dfrac{3}{4}.\]
        由 \[\lambda > 0, \nabla g(x)^td = 0\]
        可得 
        \[d = (d_1, d_2)^t, \quad  d_1 - 2d_2 = 0,\ d \neq 0.\] 
        此时 
        \begin{align*}
            d^t\nabla^2_xL(x, \lambda)d &= \begin{pmatrix}
                2d_2 & d_2
            \end{pmatrix} \begin{pmatrix}
                0 & 1 \\ 1 & 0
            \end{pmatrix} \begin{pmatrix}
                2d_2 \\ d_2
            \end{pmatrix}\\
            &=4d_2^2\\
            &> 0
        \end{align*} 
        故矩阵 $\nabla^2L_x(x, \lambda)$ 在子空间 $G$ 上正定,故是极小值点。

        只有唯一的 KKT 点,故也是全局最优解。
        \item 令 
        \[\begin{cases}
            \partial_{x_1}G = x_2 + \dfrac{2r}{-2x_1 + x_2 + 3} = 0\\
            \partial_{x_2}G = x_1 - \dfrac{r}{-2x_1 + x_2 + 3} = 0\\
        \end{cases}.\] 可得 \[x = \frac{1}{8}(3 \mp \sqrt{9 - 16r}, -6 \pm 2\sqrt{9 - 16r})^t.\] 
        需在 $r > 0$ 时满足约束条件,故 
        \[x = \frac{1}{8}(3 + \sqrt{9 - 16r}, -6 - 2\sqrt{9 - 16r})^t.\] 
        令 $r \to 0$, 可得 $x = \left(\dfrac{3}{4}, -\dfrac{3}{2}\right)^t$ 是全局最优解。
    \end{enumerate}
\end{Problem}

\begin{Problem}
    考虑约束优化问题 (P6):
    \[\begin{cases}
        \min \quad &x_1x_2\\
        \subject \quad &2x_1 - x_2 - 3 = 0
    \end{cases}\]
    \begin{enumerate}
        \item 给定 $\bar{x} = \left(\frac{3}{4}, -\frac{3}{2}\right)^t$, 利用约束优化问题局部解的一阶必要条件合二阶充分条件判断 $\bar{x}$ 是否是 (P6) 的局部最优解?
        \item 定义外罚函数为
        \[G(x, c) = x_1x_2 + \dfrac{c}{2}\left(2x_1 - x_2 - 3\right)^2\]
        试用外罚函数法求解 (P6),并说明产生的序列趋向点。
    \end{enumerate}

    \Answer \text{}
    \begin{enumerate}
        \item Lagrange 乘子函数为 $L(x, \mu) = x_1x_2 + \mu(2x_1 - x_2 - 3)$.
        
        由 KKT 条件可得 
        \[\begin{cases}
            2x_1 - x_2 - 3 &= 0\\
            \partial_{x_1}L = x_2 + 2\mu &= 0\\
            \partial_{x_2}L = x_1 - \mu &= 0
        \end{cases}\]
        解得
        \[x^* = \left(\dfrac{3}{4}, -\dfrac{3}{2}\right).\]
        由二阶最优性条件可得 
        \[\nabla h(x)^td = 0\]
        故
        \[d = \left(d_1, d_2\right) \neq 0,\quad 2d_1 - d_2 = 0\]
        则 Lagrange 乘子函数的 Hessian 矩阵在空间 $G$ 上 
        \begin{align*}
            d^t\nabla^2L_x(x, \mu)d &= \begin{pmatrix}
                d_1 & 2d_1
            \end{pmatrix}\begin{pmatrix}
                0 & 1 \\ 1 & 0
            \end{pmatrix}\begin{pmatrix}
                d_1 \\ 2d_1
            \end{pmatrix}\\
            &= 4d_1^2\\
            &> 0
        \end{align*}
        故 $\nabla^2L_x(x, \mu)$ 正定,$\bar{x} = \left(\dfrac{3}{4}, -\dfrac{3}{2}\right)$ 是 (P6) 的局部最优解。
        \item 对 $x$ 求偏导可得 
        \begin{align*}
            \partial_{x_1}G &= x_2 + c(2x_1 - x_2 - 3) \cdot 2 = 0\\
            \partial_{x_2}G &= x_1 + c(2x_1 - x_2 - 3) \cdot (-1) = 0
        \end{align*}
        解得 
        \[x^* = \left(\dfrac{3c}{4c - 1}, \dfrac{-6c}{4c - 1}\right).\]
        令 $c \to +\infty$, 得
        \[x^* = \left(\dfrac{3}{4}, -\dfrac{3}{2}\right).\]
    \end{enumerate}
\end{Problem}

\begin{Problem}
    用单纯形法求解标准的线性规划问题得到下面的最优表:
    \[\begin{array}{|l|rccc|c|}
        \hline & x_{1} & x_{2} & x_{3} & x_{4} & \\
        \hline & 0 & 0 & \bar{c}_{3} & \bar{c}_{4} & \\
        \hline x_{2} & 0 & 1 & -1 & \alpha & 1 \\
        \hline x_{1} & 1 & 0 & 4 & \beta & 3 \\
        \hline
    \end{array}\]
    设 $\bar{c_3} = 0$, 求另一个不同于表中的最优解。

    \Answer 
    \[\begin{array}{|l|rccc|c|}
        \hline & x_{1} & x_{2} & x_{3} & x_{4} & \\
        \hline & 0 & 0 & 0 & \bar{c}_{4} & \\
        \hline x_{2} & \frac{1}{4} & 1 & 0 & \alpha + \frac{\beta}{4} & \frac{7}{4} \\
        \hline x_{3} & \frac{1}{4} & 0 & 1 & \frac{\beta}{4} & \frac{3}{4} \\
        \hline
    \end{array}\]
    故线性规划另一个解为 $x = \left(0, \dfrac{7}{4}, \dfrac{3}{4}, 0\right)^t$.
\end{Problem}

\end{document}