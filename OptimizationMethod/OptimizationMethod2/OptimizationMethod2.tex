% \documentclass[12pt]{article}
\documentclass[12pt]{ctexart}
\usepackage[utf8]{inputenc}

\usepackage[english]{babel}
\usepackage[dvips]{epsfig}
\usepackage{amsmath}
\usepackage{amssymb}
\usepackage{amsfonts}
\usepackage{amsthm}
\usepackage{amsbsy}
\usepackage{amsgen}
\usepackage{amscd}
\usepackage{amsopn}
\usepackage{amstext}
\usepackage{amsxtra}
\usepackage{mathrsfs}
\usepackage{enumitem}
\usepackage{graphicx}
\usepackage{verbatim}
\usepackage{epstopdf}
\usepackage{float}
\usepackage[all,cmtip]{xy}
\usepackage{accents}
\usepackage{sseq}
\usepackage{url}
\usepackage{hyperref}
\usepackage{makeidx}
\usepackage{siunitx}
\usepackage{xcolor}

%%%%%%%%% 版面设置 %%%%%%%%%%%%%%%%%%%%%%%%%%%%%%%%%%%%%%
\usepackage{geometry}
\usepackage{titlesec}
\usepackage{fancyhdr}\pagestyle{empty}
\titleformat*{\section}{\large\bfseries}

%
\geometry{
	a4paper,
	total={170mm,240mm},
	left=20mm,
	top=30mm,
}

%Bitte nicht einstellen
\renewcommand{\figurename}{Abbildung}
\renewcommand{\tablename}{Tabelle}
\pagestyle{fancyplain}
\headheight 35pt
\lhead{\name}
\chead{\textbf{\Large \Title}}
\rhead{\due\\\today}
\lfoot{}
\cfoot{}
\rfoot{\small\thepage}
\headsep 1.5em

%%%%%%%%%%%%%%%%%%%%%%%%%%%%%%%%%%%%%%%%%%%%%%%%%%%%%%

\newtheorem{thm}{Theorem}[section]

% 定义解题环境
\theoremstyle{remark}
\newtheorem{remark}[thm]{Remark}
% \newtheorem{remark}[thm]{Remark}
\newtheorem{observation}[thm]{Observation}

\theoremstyle{definition}
\newtheorem{problem}{\text{}}
\newtheorem{Problem}{\text{}}
\newtheorem*{solution}{解}
\newtheorem*{Answer}{Answer}

%%%%%%%%%%%%%%%%%%%%%%%%%%%%%%%%%%%%%%%%%%%%%%%%%%%%%%%%%%%%%%%%%%
\newcommand\name{陈景龙22120307}
\newcommand\due{-}
\newcommand{\emptyline}{\vspace{0.6\baselineskip}}

\newcommand\Title{最优化方法第2次作业}


\begin{document}

\begin{problem}
    求实值函数在任一点的梯度及 Hesse 矩阵 $f(x)=f(x_1, x_2)=(x_1-x_2)^2+4x_1x_2+e^{x1+x2}$。
\end{problem}

\begin{solution}
    $$\begin{aligned}
        \partial_1f(\boldsymbol{x}) &= 2(x_1-x_2)+4x_2+e^{x_1+x_2}\\
        \partial_2f(\boldsymbol{x}) &= -2(x_1-x_2)+4x_1+e^{x_1+x_2}\\
        \nabla f(\boldsymbol{x}) &= (2x_1+2x_2+e^{x_1+x_2},2x_1+2x_2+e^{x_1+x_2})^T\\
        \partial_{1,1}f(\boldsymbol{x}) &= 2+e^{x_1+x_2}\\
        \partial_{1,2}f(\boldsymbol{x}) &= 2+e^{x_1+x_2}\\
        \partial_{2,1}f(\boldsymbol{x}) &= 2+e^{x_1+x_2}\\
        \partial_{2,2}f(\boldsymbol{x}) &= 2+e^{x_1+x_2}\\
        \nabla^2f(\boldsymbol{x})&=\begin{pmatrix}
            2+e^{x_1+x_2} & 2+e^{x_1+x_2} \\
            2+e^{x_1+x_2} & 2+e^{x_1+x_2}
        \end{pmatrix}
    \end{aligned}$$
\end{solution}
  
\begin{problem}
    求二次函数在任一点的梯度以及 Hesse 矩阵 $f(\boldsymbol{x})=\frac{1}{2}\boldsymbol{x}^T\boldsymbol{A}\boldsymbol{x}+\boldsymbol{b}^T\boldsymbol{x}+c$,其中 $\boldsymbol{A}\in\mathbb{R}^{n\times n},\boldsymbol{b}\in \mathbb{R}^{n}, c\in \mathbb{R}$。
\end{problem}

\begin{solution}
    $$\begin{aligned}
        f(\boldsymbol{x}+\boldsymbol{\epsilon})-f(\boldsymbol{x}) =& \frac{1}{2}(\boldsymbol{x}+\boldsymbol{\epsilon})^T\boldsymbol{A}(\boldsymbol{x}+\boldsymbol{\epsilon})+\boldsymbol{b}^T(\boldsymbol{x}+\boldsymbol{\epsilon})+c\\
        &-\frac{1}{2}\boldsymbol{x}^T\boldsymbol{A}\boldsymbol{x}-\boldsymbol{b}^T\boldsymbol{x}-c\\
        =&\frac{1}{2}\boldsymbol{x}^T\boldsymbol{A}\boldsymbol{\epsilon}+\frac{1}{2}\boldsymbol{\epsilon}^T\boldsymbol{A}\boldsymbol{x}+\boldsymbol{b}^T\boldsymbol{\epsilon}+\frac{1}{2}\boldsymbol{\epsilon}^T\boldsymbol{A}\boldsymbol{\epsilon}\\
        =&(\frac{1}{2}\boldsymbol{x}^T\boldsymbol{A}+\frac{1}{2}\boldsymbol{x}^T\boldsymbol{A}^T+\boldsymbol{b}^T)\boldsymbol{\epsilon}+\frac{1}{2}\boldsymbol{\epsilon}^T\boldsymbol{A}\boldsymbol{\epsilon}
    \end{aligned}$$
    由 $f(\boldsymbol{x}+\boldsymbol{\epsilon})=f(\boldsymbol{x})+\nabla^Tf(\boldsymbol{x})\boldsymbol{\epsilon}+\frac{1}{2}\boldsymbol{\epsilon}^T\nabla^2f(\boldsymbol{x})\boldsymbol{\epsilon} + o(\|\boldsymbol{\epsilon}\|^2)$,可得
    $$\begin{aligned}
        \nabla f(\boldsymbol{x}) &= \frac{1}{2}(\boldsymbol{A}^T\boldsymbol{x}+\boldsymbol{A}\boldsymbol{x})+\boldsymbol{b}\\
        \nabla^2 f(\boldsymbol{x}) &= \frac{1}{2}(\boldsymbol{A} + \boldsymbol{A}^t)
    \end{aligned}$$
\end{solution}

\begin{problem}
    求向量值函数 $\boldsymbol{f}(\boldsymbol{x})$ 在任一点的导数(即 Jacobi 矩阵)。
    $$\boldsymbol{f}(\boldsymbol{x})=\boldsymbol{f}(x_1,x_2)=\begin{bmatrix}
        \sin x_1+\cos x_2\\
        e^{2x_1+x_2}\\
        2x_1^2+x_1x_2
    \end{bmatrix}$$
\end{problem}

\begin{solution}
    $$\begin{aligned}
        \boldsymbol{J}=\nabla_{\boldsymbol{x}} \boldsymbol{f}=\frac{\mathrm{d} \boldsymbol{f}(\boldsymbol{x})}{\mathrm{d} \boldsymbol{x}}&=\begin{bmatrix}
            \frac{\partial \boldsymbol{f}(\boldsymbol{x})}{\partial x_{1}}  & \frac{\partial \boldsymbol{f}(\boldsymbol{x})}{\partial x_{2}}
        \end{bmatrix}\\
        &=\begin{bmatrix}
            \frac{\partial f_{1}(\boldsymbol{x})}{\partial x_{1}} & \frac{\partial f_{1}(\boldsymbol{x})}{\partial x_{2}} \\
            \frac{\partial f_{2}(\boldsymbol{x})}{\partial x_{1}} & \frac{\partial f_{2}(\boldsymbol{x})}{\partial x_{2}} \\
            \frac{\partial f_{3}(\boldsymbol{x})}{\partial x_{1}} & \frac{\partial f_{3}(\boldsymbol{x})}{\partial x_{2}}
        \end{bmatrix}\\
        &=\begin{bmatrix}
            \cos x_1 & -\sin x_2\\
            2e^{2x_1+x_2} & e^{2x_1+x_2}\\
            4x_1+x_2 & x_1
        \end{bmatrix}
    \end{aligned}$$
\end{solution}

\begin{problem}
    用定义验证下列集合为凸集。
    $$S=\left\{(x_1, x_2) | x_1 + 2x_2 \ge 1, x_1 - x_2 \ge 1\right\}$$
\end{problem}

\begin{solution}
    任取集合中的两个点 $\boldsymbol{x} = (x_1, x_2)$ 和 $\boldsymbol{y} = (y_1, y_2)$,满足 
    $$\begin{aligned}
        \left\{\begin{matrix} 
            x_1 + 2x_2 &\ge 1 \\  
            x_1 - x_2 &\ge 1
        \end{matrix}\right.\\
        \left\{\begin{matrix}
            y_1 + 2y_2 &\ge 1 \\  
            y_1 - y_2 &\ge 1
        \end{matrix}\right. 
    \end{aligned}$$
    对于 $\forall \theta \in \left[0, 1\right]$,点 $\boldsymbol{z} = (z_1, z_2) = \theta\boldsymbol{x} + (1 - \theta)\boldsymbol{y} = (\theta x_1 + (1 - \theta) y_1, \theta x_2 + (1 - \theta) y_2)$,都满足
    $$\begin{aligned}
        z_1 + 2z_2 &= \theta x_1 + (1 - \theta) y_1 + 2\theta x_2 + 2(1 - \theta) y_2\\
        &=\theta (x_1 + 2x_2) + (1 - \theta)(y_1 + 2y_2)\\
        &\ge \theta + (1 - \theta)\\
        &\ge 1\\
        z_1 - z_2 &= \theta x_1 + (1 - \theta) y_1 - \theta x_2 - (1 - \theta) y_2\\
        &=\theta(x_1 - x_2) + (1 - \theta)(y_1 - y_2) \\
        &\ge \theta + (1 - \theta)\\
        &\ge 1
    \end{aligned}$$
    所以 $\boldsymbol{z}$ 也在集合 $S$ 中,所以集合 $S$ 是一个凸集。
\end{solution}

\end{document}